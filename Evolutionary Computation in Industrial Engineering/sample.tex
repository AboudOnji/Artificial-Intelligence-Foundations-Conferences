%----------------------------------------------------------------------------------------
%	PACKAGES AND THEMES
%----------------------------------------------------------------------------------------

\documentclass[aspectratio=169,xcolor=dvipsnames]{beamer}
\usetheme{SimpleDarkBlue}

\usepackage[spanish]{babel}
\usepackage{hyperref}
\usepackage{graphicx} % Allows including images
\usepackage{booktabs} % Allows the use of \toprule, \midrule and \bottomrule in tables
\usepackage{amsmath}
\usepackage{lettrine}
\setbeamertemplate{caption}[numbered]
\usepackage[dvipsnames,svgnames,x11names]{xcolor}% Para definir y usar colores (ej. en hipervínculos)
\usepackage{xurl}
\usepackage{hyperref}       % Para crear hipervínculos internos y externos
\usepackage{algorithm}
\usepackage{algorithmicx}
\usepackage{algpseudocode}
\hypersetup{
    colorlinks=true,        % Colorear los enlaces en lugar de usar recuadros
    linkcolor=blue,     % Color para enlaces internos (índice, referencias cruzadas)
    filecolor=blue, % Color para enlaces a archivos locales
    urlcolor=blue,      % Color para URLs
    citecolor=blue,     % Color para citas bibliográficas
}
%----------------------------------------------------------------------------------------

\usepackage{listings}
\usepackage{xcolor} % Para colores en listings
 \definecolor{codegreen}{rgb}{0,0.6,0}
 \definecolor{codegray}{rgb}{0.5,0.5,0.5}
 \definecolor{codepurple}{rgb}{0.58,0,0.82}
 \definecolor{backcolour}{rgb}{0.97,0.97,0.99}

\lstdefinestyle{MATLABStyle}{
  language=Matlab,
  basicstyle=\ttfamily\footnotesize,
  keywordstyle=\color{blue}\bfseries,
  commentstyle=\color{codegreen},
  stringstyle=\color{violet},
  numberstyle=\tiny\color{gray},
  breakatwhitespace=false,
  breaklines=true,
  captionpos=b,
  keepspaces=true,
  numbers=left,
  numbersep=5pt,
  showspaces=false,
  showstringspaces=false,
  showtabs=false,
  tabsize=2,
  frame=lines, % Añade un marco alrededor del código
  framerule=0.4pt, % Grosor del marco
  backgroundcolor=\color{backcolour} % Color de fondo suave
}
\lstset{style=MATLABStyle}
%----------------------------------------------------------------------------------------
%	TITLE PAGE
%----------------------------------------------------------------------------------------

\title{Computación Evolutiva y Machine Learning: Una Perspectiva de Ingeniería Industrial}
\subtitle{10º Congreso Internacional de Ingeniería - Instituto Tecnológico de México}

\author{Prof. D.Sc. BARSEKH-ONJI Aboud}

\institute
{
    Universidad Anáhuac México Sur \\
    Facultad de Ingeniería
}
\date{\today}

%----------------------------------------------------------------------------------------
%	PRESENTATION SLIDES
%----------------------------------------------------------------------------------------
% Poner esto en el preámbulo


%------------------------------------------------
 % Date, can be changed to a custom date

%----------------------------------------------------------------------------------------
%	PRESENTATION SLIDES
%----------------------------------------------------------------------------------------
% Poner esto en el preámbulo
\AtBeginSection
{
  \begin{frame}{Agenda}
    \tableofcontents[currentsection]
  \end{frame}
}
\begin{document}

\begin{frame}
    % Print the title page as the first slide
    \titlepage
\end{frame}

\begin{frame}{Objetivos de la Presentación}
    \begin{itemize}
        \item Introducir los conceptos fundamentales de la Computación Evolutiva (CE) y el Machine Learning (ML).
        \item Explorar cómo estas técnicas pueden transformar la práctica de la ingeniería industrial.
        \item Presentar ejemplos prácticos y casos de estudio relevantes.
        \item Discutir las implicaciones futuras y las oportunidades de investigación en este campo emergente.
    \end{itemize}
\end{frame}
%------------------------------------------------
\section{La Inteligencia Computacional y la Inteligencia artificial generativa}

\begin{frame}{Conceptos básicos}
    \begin{block}{Inteligencia Computacional (IC)}
    La IC es un subcampo de la Inteligencia Artificial que se enfoca en desarrollar algoritmos inspirados en procesos naturales para resolver problemas complejos. Incluye técnicas como:
    \begin{itemize}
        \item Computación Evolutiva (CE)
        \item Redes Neuronales Artificiales (RNA)
        \item Sistemas de Lógica Difusa
        \item Algoritmos de Enjambre
        \item Sistemas inmunológicos artificiales
    \end{itemize}
    \end{block}
\end{frame}
\begin{frame}{Conceptos básicos}

    \begin{alertblock}{Inteligencia Artificial Generativa (IAG)}
    La IAG se refiere a sistemas de IA capaces de generar contenido nuevo y original, como texto, imágenes o música, basándose en patrones aprendidos de datos existentes. Ejemplos incluyen:
    \begin{itemize}
        \item Modelos de lenguaje grande (LLM) como GPT-4, Gemini, ... tc.
        \item Redes Generativas Antagónicas (GANs)
    \end{itemize}
    \end{alertblock}
\end{frame}
\section{Introducción: La Nueva Frontera de la Optimización}
%------------------------------------------------

\begin{frame}
\frametitle{El Imperativo de la Optimización en Ingeniería Industrial}
La ingeniería industrial, en su esencia, es la disciplina de la \textbf{optimización}.
\begin{itemize}
    \item Es el motor que impulsa la productividad, la calidad y la competitividad en cualquier sector.
    \pause
    \item El ingeniero industrial es un \textbf{arquitecto de la eficiencia}, cuya misión es diseñar, mejorar e implementar sistemas integrados.
    \pause
    \item En el contexto actual de competencia feroz, globalización y recursos finitos, objetivos como:
    \begin{itemize}
        \item Optimización de procesos
        \item Reducción de costos
        \item Mejora continua
    \end{itemize}
    \pause
    \begin{alertblock}{¡Atención!}
        Estos objetivos han pasado de ser opcionales a \textbf{imperativos estratégicos} para la supervivencia y el crecimiento.
    \end{alertblock}
    
\end{itemize}
\end{frame}

\begin{frame}
\frametitle{Limitaciones de los Métodos de Optimización Clásicos}
Históricamente, la ingeniería industrial se ha apoyado en técnicas de optimización matemática y estadística (programación lineal, cálculo de gradientes, etc.).

\begin{columns} % T aligns the tops of the columns
\begin{column}{.5\textwidth}
    \begin{block}{Fortalezas}
        \begin{itemize}
        \item Extraordinariamente efectivos para problemas bien definidos.
        \item Ideales para relaciones lineales y funciones objetivo derivables.
    \end{itemize}
    \end{block}
    
\end{column}
\begin{column}{.5\textwidth}
    \begin{alertblock}{Debilidades}
        \begin{itemize}
        \item La realidad industrial rara vez es ordenada.
        \item Problemas clave (rutas, secuenciación, layout) son de optimización combinatoria \textbf{NP-duros}.
        \item El espacio de búsqueda crece exponencialmente, haciendo los métodos exactos computacionalmente \textbf{inviables}.
        \item Propensos a quedar atrapados en \textbf{óptimos locales}.
    \end{itemize}
    \end{alertblock}
    
\end{column}
\end{columns}
\end{frame}

\section{Terminología de los Problemas de Optimización}
%------------------------------------------------

\begin{frame}{Terminología de los Problemas de Optimización (Parte 1)}
    \begin{description}
        \item[\textbf{Función Objetivo (o de Costo/Aptitud):}] Es la función $f(x)$ que se desea minimizar o maximizar. \pause
        \item[\textbf{Variables de Decisión:}] Son las variables $x = (x_1, \dots, x_n)$ cuyos valores se buscan para optimizar la función objetivo. \pause
        \item[\textbf{Espacio de Búsqueda:}] El conjunto de todos los posibles valores que pueden tomar las variables de decisión. \pause
        \item[\textbf{Restricciones (Constraints):}] Condiciones que deben satisfacer las variables de decisión y que limitan el espacio de búsqueda a una \textbf{región factible}.
    \end{description}
\end{frame}


\begin{frame}{Terminología de los Problemas de Optimización (Parte 2)}
    \begin{description}
        \item[\textbf{Solución Factible:}] Un conjunto de valores para las variables de decisión que satisface todas las restricciones. \pause
        \item[\textbf{Solución Óptima:}] Una solución factible que produce el valor óptimo (mínimo o máximo) de la función objetivo. \pause
        \item[\textbf{Óptimo Local vs. Óptimo Global:}] Un \textbf{óptimo local} es una solución mejor que sus vecinas. Un \textbf{óptimo global} es la mejor solución en toda la región factible.
    \end{description}
\end{frame}
\subsection{Problema de Optimización: Maximización de Producción}
\begin{frame}
\frametitle{Problema de Optimización: Maximización de Producción}

\begin{itemize}
    \item Imagina una fábrica que produce dos tipos de productos: \textbf{Producto A} y \textbf{Producto B}.
    \item La ganancia por unidad del Producto A es de \textbf{\$40}.
    \item La ganancia por unidad del Producto B es de \textbf{\$30}.
    \item Se dispone de \textbf{2400 horas} de mano de obra y \textbf{3200 kg} de materia prima.
\end{itemize}

\vspace{0.5cm}
\textbf{Requisitos de producción por unidad:}
\begin{center}
\begin{tabular}{|l|c|c|c|}
\hline
\textbf{Recurso} & \textbf{Producto A} & \textbf{Producto B} & \textbf{Disponibilidad Semanal} \\
\hline
Mano de Obra (horas) & 2 & 3 & 2400 \\
Materia Prima (kg)   & 4 & 2 & 3200 \\
\hline
\end{tabular}
\end{center}


\end{frame}

\begin{frame}{Problema de Optimización: Maximización de Producción}
    \begin{block}{Formulación Matemática}
Sea:
\begin{itemize}
    \item $x_1$ = número de unidades del Producto A a producir.
    \item $x_2$ = número de unidades del Producto B a producir.
\end{itemize}

\textbf{Función Objetivo (a maximizar):}
\[
\text{Maximizar } Z = 40x_1 + 30x_2
\]

\textbf{Sujeto a las siguientes restricciones:}
\begin{align*}
    2x_1 + 3x_2 &\leq 2400 \quad (\text{Restricción de mano de obra}) \\
    4x_1 + 2x_2 &\leq 3200 \quad (\text{Restricción de materia prima}) \\
    x_1, x_2 &\geq 0 \quad (\text{No negatividad})
\end{align*}
\end{block}

\end{frame}

\begin{frame}{Problema de Optimización: Maximización de Producción}
\begin{figure}
    \centering
    \includegraphics[width=0.9\linewidth]{optimizacion_clasica_con_objetivo.png}
\end{figure}
\end{frame}
\begin{frame}{Tipos de Soluciones Óptimas}

\begin{figure}
        \centering
        \includegraphics[width=0.7\textwidth]{cresta.png}
        \caption{Espacio de Estados unidimensionales y tridimensionales}
        \label{fig:ejemplos_sol_optimas}
    \end{figure}
\end{frame}
\subsection{Modelo de Cantidad Económica de Pedido (EOQ) con Descuentos por Cantidad.}
\begin{frame}{Modelo de Cantidad Económica de Pedido (EOQ) con Descuentos por Cantidad.}
\begin{block}{}
    Imagina que eres el gerente de inventario de una empresa. Necesitas decidir cuántas unidades de un producto pedir a tu proveedor cada vez que haces un pedido.
\begin{itemize}
    \item Si pides muy poco (pedidos pequeños y frecuentes): Gastarás mucho en costos de envío y procesamiento de pedidos (costo por ordenar).
    \item Si pides demasiado (pedidos grandes y poco frecuentes): Gastarás mucho en costos de almacenamiento, seguro y capital inmovilizado (costo de mantener inventario).
\end{itemize}
\end{block}
\end{frame}

\begin{frame}{Modelo de Cantidad Económica de Pedido (EOQ) con Descuentos por Cantidad.}
\begin{block}{Escenario con descuentos}
    El modelo EOQ clásico encuentra el balance perfecto. Pero, ¿qué pasa si el proveedor te ofrece un descuento si compras más unidades?
\begin{itemize}
    \item Comprar más podría reducir el costo total gracias al precio más bajo por unidad.
    \item Sin embargo, también aumentaría tus costos de almacenamiento.

\end{itemize}
Esto crea un problema de optimización complejo. La ''mejor'' cantidad a pedir podría ser el punto de equilibrio de un nivel de precios, o podría ser la cantidad exacta para alcanzar el siguiente nivel de descuento. Estos son los múltiples mínimos locales y el mínimo global que buscamos.
\end{block}
\end{frame}

\begin{frame}{Modelo de Cantidad Económica de Pedido (EOQ) con Descuentos por Cantidad.}
\begin{figure}
    \centering
    \includegraphics[width=0.8\linewidth]{eoq_falla_clasica.png}
\end{figure}
\end{frame}


\begin{frame}
\frametitle{Presentación de Nuevos Paradigmas de IA}
En respuesta a estos desafíos, han surgido dos poderosas ramas de la Inteligencia Artificial (IA):
\vspace{1em}
\begin{columns}
\begin{column}{.5\textwidth}
    \begin{block}{Computación Evolutiva (CE)}
    \begin{itemize}
        \item Familia de metaheurísticas estocásticas inspiradas en la \textbf{evolución biológica} de Darwin.
        \item Trabaja con una ''población'' de soluciones que ''evolucionan'' mediante selección, cruzamiento y mutación.
        \item \textbf{Fortaleza:} Explora de manera robusta y paralela espacios de búsqueda vastos y complejos, encontrando soluciones de alta calidad para problemas intratables.
    \end{itemize}
    \end{block}
\end{column}
\begin{column}{.5\textwidth}
    \begin{alertblock}{Machine Learning (ML)}
    \begin{itemize}
        \item Dota a los sistemas de la capacidad de \textbf{aprender patrones e inferencias directamente de los datos}, sin ser programados explícitamente.
        \item Construye sus propios modelos a través de la ''experiencia'' para predecir, clasificar y descubrir estructuras ocultas en los datos.
    \end{itemize}
    \end{alertblock}
\end{column}
\end{columns}
\end{frame}

\begin{frame}
\frametitle{Tesis Central: La Sinergia Estratégica}
\begin{alertblock}{Idea Clave}
La verdadera revolución para la ingeniería industrial no reside en su aplicación aislada, sino en su \textbf{sinergia estratégica}.
\end{alertblock}
\vspace{1em}
\begin{itemize}
    \item La CE no solo resuelve problemas de optimización, sino que también \textbf{potencia y automatiza el diseño de los propios modelos de Machine Learning}.
    \item Se crea un ciclo virtuoso:
    \begin{enumerate}
        \item El \textbf{ML} extrae conocimiento y patrones predictivos de los datos de los sistemas industriales.
        \item La \textbf{CE} utiliza estas predicciones para optimizar las decisiones operativas (rutas, horarios, parámetros).
    \end{enumerate}
    \item Esta combinación dota al ingeniero de una capacidad sin precedentes para diseñar sistemas que no solo son eficientes, sino también \textbf{inteligentes, adaptativos y resilientes}.
\end{itemize}
\end{frame}
\section{El nuevo rol del Ingeniero Industrial}
\begin{frame}
\frametitle{El Nuevo Rol del Ingeniero Industrial}
\begin{columns}
\begin{column}{.5\textwidth}
    \begin{block}{Enfoque Tradicional:}
    \begin{itemize}
        \item Analizar un sistema estático.
        \item Aplicar técnicas para diseñar una configuración ''óptima'' bajo supuestos predefinidos.
        \item Resultado: Soluciones \textbf{frágiles} ante la dinámica y la incertidumbre del mundo real (fallos, fluctuaciones de demanda).
    \end{itemize}
    \end{block}
\end{column}
\begin{column}{.5\textwidth}
    \begin{alertblock}{Nuevo Paradigma (CE+ML):}
    \begin{itemize}
        \item El objetivo ya no es ''diseñar lo óptimo'', sino \textbf{''diseñar sistemas que se optimizan a sí mismos''}.
        \item El ML permite que el sistema \textit{aprenda} de eventos en tiempo real.
        \item La CE utiliza estos modelos actualizados para \textit{re-optimizar} continuamente las operaciones.
        \item El ingeniero evoluciona de ser un analista a ser el \textbf{arquitecto de un sistema ciberfísico inteligente}.
    \end{itemize}
    \end{alertblock}
\end{column}
\end{columns}
\end{frame}
\section{Introducción a la Computación Evolutiva}
%------------------------------------------------

\begin{frame}{¿Qué es la Computación Evolutiva (CE)?}
    \begin{block}{Inspiración: La Evolución Biológica}
    La Computación Evolutiva (CE) es un subcampo de la Inteligencia Artificial que se inspira directamente en los principios de la evolución biológica para desarrollar algoritmos de búsqueda y optimización.
    \end{block}
    
    \begin{alertblock}{Mecanismo General}
    Los algoritmos evolutivos operan sobre una \textbf{población} de soluciones candidatas, aplicando procesos de \textbf{selección}, \textbf{reproducción} y \textbf{variación} para mejorar iterativamente la calidad de las soluciones a lo largo de \textbf{generaciones}.
    \end{alertblock}
\end{frame}


%------------------------------------------------

\begin{frame}{Antecedentes Biológicos: Primeras Ideas}
    \begin{columns}[t]
        \column{.48\textwidth}
            \begin{block}{Lamarckismo (Principios del s. XIX)}
                \begin{itemize}
                    \item Propuso el primer mecanismo coherente de evolución.
                    \item Se basaba en la \textbf{'herencia de características adquiridas'}: los cambios en el entorno modificaban a los organismos, y estas modificaciones se transmitían a la descendencia.
                    \item Aunque incorrecto, fue una idea revolucionaria sobre la adaptación gradual.
                \end{itemize}
            \end{block}

        \column{.48\textwidth}
            \begin{alertblock}{Selección Natural (Darwin y Wallace, 1859)}
                 \begin{itemize}
                    \item Dentro de una población existe \textbf{variación natural}.
                    \item Los individuos con variaciones ventajosas tienen más probabilidades de sobrevivir y reproducirse (\textbf{'supervivencia del más apto'}).
                    \item Con el tiempo, este proceso conduce a la adaptación de las poblaciones.
                \end{itemize}
            \end{alertblock}
    \end{columns}
\end{frame}

%------------------------------------------------

\begin{frame}{La Pieza Faltante y la Síntesis Moderna}
    \begin{block}{El Problema de la Herencia}
    La principal debilidad de la teoría de Darwin era la falta de un mecanismo de herencia. La teoría de la 'herencia por mezcla' (que sugería que las características se promediaban) contradecía la selección natural.
    \end{block}
    
    \begin{alertblock}{La Genética Mendeliana}
    Sin que Darwin lo supiera, Gregor Mendel demostró que la herencia es \textbf{particulada}, no por mezcla. Los rasgos se heredan a través de 'factores' discretos (hoy llamados \textbf{genes}) que no se diluyen.
    \end{alertblock}
    
   
\end{frame}
\begin{frame}{La Pieza Faltante y la Síntesis Moderna}
    \begin{columns}[t]
        \column{0.48\textwidth}
        \begin{figure}
            \centering
            \includegraphics[width=0.58\linewidth]{darwin.jpg}
            \caption{Charles Darwin}
        \end{figure}
        \column{0.48\textwidth}
        \begin{figure}
            \centering
            \includegraphics[width=0.5\linewidth]{mendel.jpg}
            \caption{Gregor Mendel}
        \end{figure}
    \end{columns}
    \end{frame}

\begin{frame}{La Pieza Faltante y la Síntesis Moderna}

    \begin{block}{La Síntesis Moderna (Neo-Darwinismo)}
    Unificó la selección natural de Darwin con la genética mendeliana. La evolución es el resultado de la interacción de:
    \begin{itemize}
        \item \textbf{Variación Genética:} (Mutación y Recombinación).
        \item \textbf{Herencia:} Transmisión de genes.
        \item \textbf{Selección Natural:} Supervivencia diferencial.
    \end{itemize}
    \textbf{Este es el marco conceptual que inspira directamente a la Computación Evolutiva.}
    \end{block}
\end{frame}
\section{Principios en la biología}
\begin{frame}{Principios en la biología}
   \begin{figure}
       \centering
       \includegraphics[width=0.5\linewidth]{crossover_simple.jpg}
       \caption{Cruce (Crossover)}
       \label{fig:placeholder}
   \end{figure} 
\end{frame}

\begin{frame}{Principios en la biología}
   \begin{figure}
       \centering
       \includegraphics[width=0.8\linewidth]{Crossover-and-mutation.png}
       \caption{Cruce (Crossover) con Mutación (Mutation)}
       \label{fig:placeholder}
   \end{figure} 
\end{frame}
%------------------------------------------------
\section{Pioneros y Paradigmas Computacionales}
%------------------------------------------------

\begin{frame}{Pioneros y Paradigmas Fundacionales}
    \begin{block}{Primeros Vislumbres Computacionales}
    La idea de simular la evolución en una computadora es casi tan antigua como la computación misma. Visionarios como \textbf{Alan Turing} ya lo sugerían en 1950. En las décadas de 1950 y 1960, pioneros como Barricelli, Fraser y Box realizaron las primeras simulaciones de procesos evolutivos.
    \end{block}
    
    \begin{alertblock}{Los Tres Pilares de la Computación Evolutiva}
    En la década de 1960, surgieron de forma independiente tres líneas de investigación que se convertirían en los pilares del campo, cada una con un enfoque y una motivación distintos.
    \end{alertblock}
\end{frame}

%------------------------------------------------

\begin{frame}{Los Tres Paradigmas Fundacionales}
        \centering\includegraphics[width=0.2\textwidth]{Fogel.jpg}

            \begin{block}{Programación Evolutiva (EP)}
            \textbf{Lawrence J. Fogel (EE.UU.)}
                \begin{itemize}
                    \item \textbf{Motivación:} IA, crear sistemas que se adaptan y predicen.
                    \item \textbf{Enfoque:} Evolucionar el \textbf{comportamiento} (fenotipo) de las soluciones.
                    \item \textbf{Operador Clave:} \textbf{Mutación}.
                \end{itemize}
            \end{block}

\end{frame}

\begin{frame}{Los Tres Paradigmas Fundacionales}
    \centering\includegraphics[width=0.2\textwidth]{rechenberg.jpeg}


            \begin{alertblock}{Estrategias Evolutivas (ES)}
            \textbf{Rechenberg y Schwefel (Alemania)}
                 \begin{itemize}
                    \item \textbf{Motivación:} Optimización de problemas de \textbf{ingeniería} con parámetros de valor real.
                    \item \textbf{Concepto Clave:} \textbf{Auto-adaptación} (el algoritmo aprende y ajusta sus propios parámetros).
                \end{itemize}
            \end{alertblock}
            

\end{frame}

\begin{frame}{Los Tres Paradigmas Fundacionales}
        \centering\includegraphics[width=0.2\textwidth]{John_Henry_Holland.jpg}

           
            
            \begin{block}{Algoritmos Genéticos (GA)}
            \textbf{John H. Holland (EE.UU.)}
                 \begin{itemize}
                    \item \textbf{Motivación:} Entender teóricamente los principios de la \textbf{adaptación}.
                    \item \textbf{Enfoque:} Énfasis en la \textbf{codificación} de la solución (genotipo).
                    \item \textbf{Operador Clave:} \textbf{Cruce (Crossover)}.
                \end{itemize}
            \end{block}
\end{frame}

%------------------------------------------------



%------------------------------------------------
\subsection{Ejemplo de Aplicación: Optimización de Función}
%------------------------------------------------

\begin{frame}{Ejemplo: Paso a Paso de una Iteración (1+1)-ES}
    \begin{block}{Problema}
    Minimizar la función de Rosenbrock: $f(x_1, x_2) = 100(x_1^2 - x_2)^2 + (1-x_1)^2$
    \end{block}
    
    \begin{enumerate}
        \item \textbf{Inicialización (Padre):} Partimos de un punto inicial $\mathbf{x}^t = (-1.0 ; 1.0)$, con una aptitud (costo) de $f(\mathbf{x}^t) = 4.0$. \pause
        
        \item \textbf{Mutación:} Se genera un nuevo individuo (hijo) añadiendo ruido Gaussiano. \\
        Supongamos que el resultado es $\mathbf{x}^{t+1} = (-0.39; 1.57)$. \pause
        
        \item \textbf{Evaluación:} Se calcula la aptitud del nuevo individuo. \\
        $f(\mathbf{x}^{t+1}) = f(-0.39; 1.57) = 201.416$. \pause
        
        \item \textbf{Selección:} 
        \begin{alertblock}{}
        Como $f(\text{hijo}) > f(\text{padre})$, la nueva solución es peor. En una (1+1)-ES, el hijo se descarta y el padre sobrevive para la siguiente generación. El proceso se repite.
        \end{alertblock}
    \end{enumerate}
\end{frame}

%------------------------------------------------


%------------------------------------------------
\section{Algoritmos Genéticos (AG)}
%------------------------------------------------

\begin{frame}{Algoritmo Genético (AG)}
    \begin{block}{Concepto}
    Popularizados por John H. Holland, los Algoritmos Genéticos (AG) son modelos que simulan la \textbf{evolución genética}. A diferencia de EP y ES, el énfasis está en el \textbf{genotipo} (la codificación de la solución).
    \end{block}
    
    \begin{alertblock}{Operadores Principales}
    El proceso evolutivo en un AG es impulsado principalmente por:
        \begin{itemize}
            \item \textbf{Selección:} Modela la supervivencia del más apto.
            \item \textbf{Cruce (Crossover):} Modela la recombinación y la reproducción sexual. Es el operador de búsqueda principal.
        \end{itemize}
    \end{alertblock}
\end{frame}

%------------------------------------------------
\subsection{El Algoritmo Genético Canónico}
%------------------------------------------------

\begin{frame}{El Algoritmo Genético Canónico}
    \begin{block}{La Propuesta Original de Holland}
    El AG 'canónico' o clásico se define por las siguientes características específicas:
    \end{block}
    
    \begin{itemize}
        \item \textbf{Representación:} Se utiliza una \textbf{cadena de bits} (\textit{bitstring}) de longitud fija para codificar cada solución. \pause
        
        \item \textbf{Selección:} Se emplea la \textbf{selección proporcional a la aptitud} (a menudo implementada como una 'rueda de ruleta') para elegir a los padres. \pause
        
        \item \textbf{Cruce (Crossover):} El \textbf{cruce de un punto} se utiliza como el método principal para producir descendencia, combinando 'bloques de construcción' de los padres. \pause
        
        \item \textbf{Mutación:} La mutación (invertir un bit al azar) se considera un operador de fondo, de menor importancia, con el único fin de mantener la diversidad genética.
    \end{itemize}
\end{frame}
\begin{frame}
\frametitle{Ejemplo: Algoritmo Genético para la Mochila}

\begin{block}{Problema}
    Seleccionar los objetos más valiosos para una mochila con capacidad de \textbf{15 kg}.
    
    \begin{center}
        \begin{tabular}{|l|c|c|}
            \hline
            \textbf{Objeto} & \textbf{Peso (kg)} & \textbf{Valor (\$)} \\
            \hline
            1: Linterna & 4 & 5 \\
            2: Cuerda & 7 & 12 \\
            3: Comida & 5 & 8 \\
            4: Agua & 6 & 10 \\
            \hline
        \end{tabular}
    \end{center}
\end{block}
\begin{itemize}
    \item \textbf{Cromosoma:} Un vector binario de 4 bits. Un '1' significa llevar el objeto, un '0' es dejarlo.
\end{itemize}
\end{frame}

\begin{frame}
\frametitle{Ejemplo: Algoritmo Genético para la Mochila}

\begin{enumerate}
    \item \textbf{Inicialización (Población Inicial):} Creamos dos soluciones (padres) al azar. \\
    \textbf{Padre A:} \texttt{[1 0 1 1]} $\rightarrow$ Peso: 15kg, Valor: \$23. (\textit{Válido}) \\
    \textbf{Padre B:} \texttt{[0 1 1 1]} $\rightarrow$ Peso: 18kg, Valor: \$30. (\textit{Inválido, excede 15kg}) \pause
    
    \item \textbf{Crossover (Cruce de un punto):} Cruzamos a los padres después del segundo bit. \\
    \textbf{Hijo 1:} \texttt{[1 0} | \texttt{1 1]} $\rightarrow$ \texttt{[1 0 1 1]} \\
    \textbf{Hijo 2:} \texttt{[0 1} | \texttt{1 1]} $\rightarrow$ \texttt{[0 1 1 1]} \pause
    
    \item \textbf{Mutación:} Mutamos el tercer bit del \textbf{Hijo 2}. \\
    \textbf{Hijo 2 (mutado):} \texttt{[0 1 0 1]} \pause
    
    \item \textbf{Evaluación (Nueva Generación):} Calculamos la aptitud de los hijos. \\
    \textbf{Hijo 1:} \texttt{[1 0 1 1]} $\rightarrow$ Peso: 15kg, Valor: \$23. (\textit{Válido}) \\
    \textbf{Hijo 2 (mutado):} \texttt{[0 1 0 1]} $\rightarrow$ Peso: 13kg, Valor: \$22. (\textit{Válido}) \pause

    \item \textbf{Selección:}
    \begin{alertblock}{}
        El \textbf{Padre A} y el \textbf{Hijo 1} son las mejores soluciones válidas hasta ahora (ambas con valor \$23). El \textbf{Hijo 2} es también una buena solución válida. El \textbf{Padre B} (inválido) se descarta. Las soluciones válidas con mayor valor sobreviven para la siguiente generación.
    \end{alertblock}
\end{enumerate}

\end{frame}
\subsection{Aplicación del Algoritmo Genético}
\begin{frame}{Aplicación del Algoritmo Genético}
    \begin{figure}
        \centering
        \includegraphics[width=0.8\linewidth]{rastrigin_3d.png}
    \end{figure}
\end{frame}

\begin{frame}{Aplicación del Algoritmo Genético}
   \begin{columns}
      \column{0.48\textwidth}
      \begin{figure}
        \centering
        \includegraphics[width=1.2\linewidth]{rastrigin_falla_gradiente.png}
        \caption{Fallo de los métodos tradicionales}
    \end{figure}
      \column{0.48\textwidth}
      \begin{figure}
        \centering
        \includegraphics[width=1.2\linewidth]{rastrigin_exito_ga.png}
        \caption{Éxito del Algoritmo genético}
    \end{figure}
   \end{columns}
    
\end{frame}
%------------------------------------------------
\end{document}