\documentclass[aspectratio=169,xcolor=dvipsnames]{beamer}
\usetheme{SimpleDarkBlue}

\usepackage[spanish]{babel}
\usepackage{hyperref}
\usepackage{graphicx}
\usepackage{booktabs}
\usepackage{amsmath}
\usepackage{lettrine}
\setbeamertemplate{caption}[numbered]
\usepackage[dvipsnames,svgnames,x11names]{xcolor}
\usepackage{xurl}
\usepackage{hyperref}
\usepackage{algorithm}
\usepackage{algorithmicx}
\usepackage{algpseudocode}
\usepackage{adjustbox}
\hypersetup{
    colorlinks=true,
    linkcolor=blue,
    filecolor=blue,
    urlcolor=blue,
    citecolor=blue,
}
%----------------------------------------------------------------------------------------
\usepackage{listings}
\usepackage{xcolor}

\definecolor{codegreen}{rgb}{0,0.6,0}
\definecolor{codegray}{rgb}{0.5,0.5,0.5}
\definecolor{codepurple}{rgb}{0.58,0,0.82}
\definecolor{backcolour}{rgb}{0.97,0.97,0.99}

\lstdefinestyle{MATLABStyle}{
  language=Matlab,
  basicstyle=\ttfamily\footnotesize,
  keywordstyle=\color{blue}\bfseries,
  commentstyle=\color{codegreen},
  stringstyle=\color{violet},
  numberstyle=\tiny\color{gray},
  breakatwhitespace=false,
  breaklines=true,
  captionpos=b,
  keepspaces=true,
  numbers=left,
  numbersep=5pt,
  showspaces=false,
  showstringspaces=false,
  showtabs=false,
  tabsize=2,
  frame=lines,
  framerule=0.4pt,
  backgroundcolor=\color{backcolour}
}
\lstset{style=MATLABStyle}
%----------------------------------------------------------------------------------------
%   TITLE PAGE
%----------------------------------------------------------------------------------------
\title{Procesamiento de Lenguaje Natural (NLP)} 
\subtitle{Clase 4: Proyecto Integrador - Clasificador de Noticias} 
\author{Prof. D.Sc. BARSEKH-ONJI Aboud}
\institute{
    Faculty of Engineering \\
    Universidad Anáhuac México Sur
}
\date{\today}
%----------------------------------------------------------------------------------------
\AtBeginSection[]
{
  \begin{frame}{Agenda}
    \tableofcontents[currentsection]
  \end{frame}
}
%----------------------------------------------------------------------------------------
\begin{document}

%-------------------------------------------------
% DIAPOSITIVA DE TÍTULO
%-------------------------------------------------
\begin{frame}[plain]
    \titlepage
\end{frame}

%-------------------------------------------------
% DIAPOSITIVA DE AGENDA GENERAL
%-------------------------------------------------
\begin{frame}{Agenda de la Clase}
    \tableofcontents
\end{frame}

%----------------------------------------------------------------------------------------
\section{El Desafío Final: Construir Nuestro Modelo}
%----------------------------------------------------------------------------------------

\begin{frame}{¿Dónde nos quedamos?}
    \begin{block}{Resumen de la Clase 3}
        Aprendimos a usar una herramienta pre-entrenada (\textbf{VADER}) para realizar Análisis de Sentimientos. ¡Fue rápido y efectivo!
    \end{block}
    
    \begin{alertblock}{La Limitación}
        VADER es excelente para saber si un texto es positivo o negativo, pero... ¿Y si queremos clasificar texto en nuestras \textbf{propias categorías}?
        
        \begin{itemize}
            \item Por ejemplo, clasificar correos en ''Spam'' vs ''No Spam''.
            \item O clasificar noticias en ''Deportes'', ''Tecnología'', ''Finanzas''.
        \end{itemize}
    \end{alertblock}
    
    \vfill
    Para esto, no podemos usar un modelo genérico. Necesitamos \textbf{entrenar nuestro propio modelo} con nuestros propios datos y categorías.
\end{frame}

%-------------------------------------------------
\begin{frame}{Objetivo: El Proyecto Integrador}
    \begin{block}{Nuestra Misión}
        Construir, entrenar y evaluar un modelo de Machine Learning de principio a fin que pueda \textbf{clasificar reportes de fábrica en diferentes categorías} (ej. ''Calidad'', ''Mantenimiento'', ''Seguridad'').
    \end{block}
    
    \begin{center}
        \LARGE
        Se tiene que unir todo lo aprendido:
        \vspace{0.5cm}
        
        \textbf{Clase 1} (Limpieza) + \textbf{Clase 2} (Vectorización) $\rightarrow$ \textbf{Clase 4} (Entrenamiento y Predicción)
    \end{center}
\end{frame}

%----------------------------------------------------------------------------------------
\section{El Flujo de Trabajo de Machine Learning}
%----------------------------------------------------------------------------------------

\begin{frame}{El Mapa del Proyecto: 6 Pasos Clave}
    Todo proyecto de Machine Learning supervisado sigue un flujo de trabajo bien definido. Hoy lo aplicaremos al texto.
    
    \begin{enumerate}
        \item \textbf{Cargar los Datos:} Leer nuestro conjunto de datos (textos y sus etiquetas correctas).
        \pause
        \item \textbf{Dividir los Datos:} Separar los datos en un conjunto de \textbf{entrenamiento} (para enseñar al modelo) y uno de \textbf{prueba} (para evaluarlo).
        \pause
        \item \textbf{Preprocesar y Vectorizar:} Aplicar las técnicas de las clases 1 y 2 para convertir el texto de entrenamiento en una matriz TF-IDF.
        \pause
        \item \textbf{Entrenar el Modelo:} Alimentar al algoritmo de Machine Learning con la matriz numérica y las etiquetas para que aprenda los patrones.
        \pause
        \item \textbf{Probar el Modelo:} Usar el modelo entrenado para hacer predicciones sobre los datos de prueba (que nunca ha visto).
        \pause
        \item \textbf{Evaluar el Rendimiento:} Comparar las predicciones del modelo con las etiquetas reales para ver qué tan bien lo hizo.
    \end{enumerate}
\end{frame}

%-------------------------------------------------
\begin{frame}{¿Por qué dividir los datos en Entrenamiento y Prueba?}
    \begin{block}{Analogía del Examen}
        \begin{itemize}
            \item El \textbf{conjunto de entrenamiento} son los ejercicios que un alumno hace en clase con el profesor. Aprende de ellos.
            \item El \textbf{conjunto de prueba} es el examen final. Contiene preguntas que el alumno nunca ha visto, y su resultado nos dice si \textit{realmente} aprendió a generalizar o si solo memorizó las respuestas de la clase.
        \end{itemize}
    \end{block}
    
    \vfill
    
    \begin{alertblock}{Regla de Oro}
        \textbf{Nunca} se debe evaluar un modelo con los mismos datos con los que fue entrenado. Hacerlo nos daría una falsa sensación de un rendimiento perfecto.
    \end{alertblock}
    
    Típicamente se usa una división 70\% para entrenamiento y 30\% para prueba.
\end{frame}


%----------------------------------------------------------------------------------------
\section{Ejemplo Práctico en MATLAB: Clasificador de Noticias}
%----------------------------------------------------------------------------------------

\begin{frame}[fragile]{Paso 1 y 2: Cargar y Dividir los Datos}
    \begin{block}{Objetivo}
    Usaremos un conjunto de datos de MATLAB llamado \texttt{factoryReports.csv}. Contiene descripciones de eventos y su categoría.
    \end{block}
\begin{lstlisting}[language=Matlab]
% 1. Cargar los Datos
data = readtable('factoryReports.csv');
textos = data.Description;
categorias = categorical(data.Category); % Etiquetas a predecir

% 2. Dividir los datos en Entrenamiento (80%) y Prueba (20%)
cvp = cvpartition(categorias, 'Holdout', 0.20);
idxEntrenamiento = training(cvp);
textosEntrenamiento = textos(idxEntrenamiento);
categoriasEntrenamiento = categorias(idxEntrenamiento);

idxPrueba = test(cvp);
textosPrueba = textos(idxPrueba);
categoriasPrueba = categorias(idxPrueba);
\end{lstlisting}
    \begin{alertblock}{En MATLAB..}
    La función \texttt{cvpartition} es la herramienta estándar para dividir datos de forma segura.
    \end{alertblock}
\end{frame}

%-------------------------------------------------
\begin{frame}[fragile]{Paso 3: Preprocesar y Vectorizar}
    \begin{block}{Objetivo}
    Convertir el \textbf{texto de entrenamiento} en una matriz numérica TF-IDF. ¡Esto ya lo sabemos hacer!
    \end{block}
\begin{lstlisting}[language=Matlab]
% Preprocesamos los textos de ENTRENAMIENTO
documentosEntrenamiento = tokenizedDocument(lower(textosEntrenamiento));
documentosEntrenamiento = erasePunctuation(documentosEntrenamiento);
documentosEntrenamiento = removeStopWords(documentosEntrenamiento);
documentosEntrenamiento = normalizeWords(documentosEntrenamiento, 'Style', 'stem');

% Creamos la bolsa de palabras A PARTIR DE LOS DATOS DE ENTRENAMIENTO
bolsa = bagOfWords(documentosEntrenamiento);

% Creamos la matriz TF-IDF de ENTRENAMIENTO
matrizEntrenamiento = tfidf(bolsa);
\end{lstlisting}
    \begin{alertblock}{Punto Clave}
    El vocabulario (la `bolsa`) se crea \textbf{únicamente} con los datos de entrenamiento.
    \end{alertblock}
\end{frame}

%-------------------------------------------------
\begin{frame}[fragile]{Paso 4: Entrenar el Modelo Clasificador}
    \begin{block}{Objetivo}
    Alimentar a un algoritmo de Machine Learning con nuestros datos para que "aprenda". Usaremos un clasificador \textbf{SVM} (\textit{Support Vector Machine}), un modelo muy potente para texto.
    \end{block}
\begin{lstlisting}[language=Matlab]
% Entrenamos un modelo de clasificacion multiclase (SVM por defecto)
% Le damos la matriz numerica y las etiquetas correctas para que aprenda.
modeloClasificador = fitcecoc(matrizEntrenamiento, categoriasEntrenamiento);

disp('¡Modelo entrenado con exito!');
\end{lstlisting}
    \begin{alertblock}{En MATLAB..}
        La función \texttt{fitcecoc} (Fit multiclass ECOC) es un comando de alto nivel que entrena un clasificador robusto.
    \end{alertblock}
\end{frame}

%-------------------------------------------------
\begin{frame}[fragile]{Paso 5: Probar el Modelo}
    \begin{block}{Objetivo}
    Ahora, preprocesamos los datos de prueba y le pedimos al modelo que prediga sus categorías.
    \end{block}
\begin{lstlisting}[language=Matlab]
% 1. Preprocesamos los datos de PRUEBA (los que el modelo no ha visto)
documentosPrueba = tokenizedDocument(lower(textosPrueba));
documentosPrueba = erasePunctuation(documentosPrueba);
documentosPrueba = removeStopWords(documentosPrueba);
documentosPrueba = normalizeWords(documentosPrueba, 'Style', 'stem');

% 2. Vectorizamos usando LA MISMA 'bolsa' del entrenamiento
matrizPrueba = tfidf(bolsa, documentosPrueba);

% 3. Hacemos las predicciones
predicciones = predict(modeloClasificador, matrizPrueba);
\end{lstlisting}

\end{frame}
\begin{frame}[fragile]{Paso 5: Probar el Modelo}

    \begin{alertblock}{Punto Crítico}
    Al vectorizar los datos de prueba, debemos usar la \textbf{misma `bolsa` (vocabulario)} que creamos con los datos de entrenamiento para asegurar que las matrices sean compatibles.
    \end{alertblock}
\end{frame}

%-------------------------------------------------
\begin{frame}[fragile]{Paso 6: Evaluar el Rendimiento}
    \begin{block}{Objetivo}
    Comparar las \texttt{predicciones} del modelo con las \texttt{categoriasPrueba} (las respuestas correctas) para medir su rendimiento.
    \end{block}
\begin{lstlisting}[language=Matlab]
% Comparamos las predicciones con las etiquetas reales
correctas = sum(predicciones == categoriasPrueba);
total = numel(categoriasPrueba);
exactitud = correctas / total;

fprintf('La exactitud (accuracy) del modelo es: %.2f%%\n', exactitud * 100);

% Creamos una matriz de confusion para un analisis mas detallado
figure;
confusionchart(categoriasPrueba, predicciones);
title('Matriz de Confusion');
\end{lstlisting}
    La \textbf{exactitud} (accuracy) nos dice el porcentaje de veces que el modelo acertó.
\end{frame}

%----------------------------------------------------------------------------------------
\section{Análisis de Resultados}
%----------------------------------------------------------------------------------------

\begin{frame}{La Matriz de Confusión: ¿Dónde se equivoca?}
    La exactitud es un buen número, pero no cuenta toda la historia. La \textbf{matriz de confusión} nos muestra exactamente qué aciertos y errores cometió el modelo.

    
    \begin{block}{Cómo leerla:}
        \begin{itemize}
            \item \textbf{La diagonal principal (en azul oscuro):} Muestra los \textbf{aciertos}. Por ejemplo, el número en la casilla (Quality, Quality) es cuántas veces un reporte que \textit{era} de calidad fue \textit{predicho} como calidad.
            \item \textbf{Fuera de la diagonal (en azul claro):} Muestra los \textbf{errores}. Por ejemplo, el número en la casilla (Maintenance, Quality) es cuántas veces un reporte que \textit{era} de mantenimiento fue \textit{confundido} con uno de calidad.
        \end{itemize}
    \end{block}
\end{frame}

%----------------------------------------------------------------------------------------
\section{Conclusiones Finales del Curso}
%----------------------------------------------------------------------------------------

\begin{frame}{Resumen del estudio de NLP}
    \begin{block}{Los puntos analizados:}
        \begin{itemize}
            \item \textbf{Clase 1:} Empezamos con texto ''sucio'' y aprendimos a \textbf{limpiarlo y estandarizarlo}.
            \vfill
            \item \textbf{Clase 2:} Convertimos esas palabras limpias en \textbf{matrices numéricas (TF-IDF)}, el lenguaje de la IA.
            \vfill
            \item \textbf{Clase 3:} Usamos un modelo pre-entrenado (\textbf{VADER}) para realizar nuestra primera tarea práctica: el \textbf{Análisis de Sentimientos}.
            \vfill
            \item \textbf{Clase 4:} ¡Construimos nuestro \textbf{propio clasificador de texto desde cero}, siguiendo un flujo de trabajo profesional de Machine Learning!
        \end{itemize}
    \end{block}
    
   
\end{frame}

\end{document}