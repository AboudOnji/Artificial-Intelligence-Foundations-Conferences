\documentclass[aspectratio=169,xcolor=dvipsnames]{beamer}
\usetheme{SimpleDarkBlue}

\usepackage[spanish]{babel}
\usepackage{hyperref}
\usepackage{graphicx}
\usepackage{booktabs}
\usepackage{amsmath}
\usepackage{lettrine}
\setbeamertemplate{caption}[numbered]
\usepackage[dvipsnames,svgnames,x11names]{xcolor}
\usepackage{xurl}
\usepackage{hyperref}
\usepackage{algorithm}
\usepackage{algorithmicx}
\usepackage{algpseudocode}
\usepackage{adjustbox}
\hypersetup{
    colorlinks=true,
    linkcolor=blue,
    filecolor=blue,
    urlcolor=blue,
    citecolor=blue,
}
%----------------------------------------------------------------------------------------
\usepackage{listings}
\usepackage{xcolor}

\definecolor{codegreen}{rgb}{0,0.6,0}
\definecolor{codegray}{rgb}{0.5,0.5,0.5}
\definecolor{codepurple}{rgb}{0.58,0,0.82}
\definecolor{backcolour}{rgb}{0.97,0.97,0.99}

\lstdefinestyle{MATLABStyle}{
  language=Matlab,
  basicstyle=\ttfamily\footnotesize,
  keywordstyle=\color{blue}\bfseries,
  commentstyle=\color{codegreen},
  stringstyle=\color{violet},
  numberstyle=\tiny\color{gray},
  breakatwhitespace=false,
  breaklines=true,
  captionpos=b,
  keepspaces=true,
  numbers=left,
  numbersep=5pt,
  showspaces=false,
  showstringspaces=false,
  showtabs=false,
  tabsize=2,
  frame=lines,
  framerule=0.4pt,
  backgroundcolor=\color{backcolour}
}
\lstset{style=MATLABStyle}
%----------------------------------------------------------------------------------------
%   TITLE PAGE
%----------------------------------------------------------------------------------------
\title{Procesamiento de Lenguaje Natural (NLP)} 
\subtitle{Clase 3: Análisis de Sentimientos} 
\author{Prof. D.Sc. BARSEKH-ONJI Aboud}
\institute{
    Faculty of Engineering \\
    Universidad Anáhuac México Sur
}
\date{\today}
%----------------------------------------------------------------------------------------
\AtBeginSection[]
{
  \begin{frame}{Agenda}
    \tableofcontents[currentsection]
  \end{frame}
}
%----------------------------------------------------------------------------------------
\begin{document}

%-------------------------------------------------
% DIAPOSITIVA DE TÍTULO
%-------------------------------------------------
\begin{frame}[plain]
    \titlepage
\end{frame}

%-------------------------------------------------
% DIAPOSITIVA DE AGENDA GENERAL
%-------------------------------------------------
\begin{frame}{Agenda de la Clase}
    \tableofcontents
\end{frame}

%----------------------------------------------------------------------------------------
\section{De los Números a la Acción}
%----------------------------------------------------------------------------------------

\begin{frame}{¿Dónde nos quedamos?}
    \begin{block}{Resumen de la Clase 2}
        ¡Logramos el gran objetivo de la vectorización! Aprendimos a convertir texto en una matriz numérica usando \textbf{Bag-of-Words} y la ponderación \textbf{TF-IDF}.
    \end{block}
    
    \begin{center}
        \textbf{Tokens Limpios} \\
        \texttt{'bateri', 'dur', 'much', ...}
        \vspace{0.5cm}
        $\downarrow$ \quad \textit{Vectorización} \quad $\downarrow$
        \vspace{0.5cm}
        \textbf{Matriz Numérica (TF-IDF)} \\
        \begin{tabular}{c c c c}
           0.25 & 0.25 & 0 & ... \\
           0 & 0 & 0.35 & ... \\
        \end{tabular}
    \end{center}
    
    \vfill
    \begin{alertblock}{La Gran Pregunta}
        Ya tenemos nuestros datos en un formato que la máquina ''entiende''. \textbf{¿Ahora qué podemos hacer con ellos?}
    \end{alertblock}
\end{frame}

%-------------------------------------------------
\begin{frame}{Nuestra Primera Aplicación: Análisis de Sentimientos}
    \begin{block}{Objetivo de Hoy}
        Usar nuestros conocimientos de NLP para realizar una de las tareas más comunes y útiles: determinar automáticamente la \textbf{emoción} o \textbf{polaridad} de un texto.
    \end{block}

    \begin{center}
        \LARGE
        ¿Este comentario es Positivo 😄, Negativo 😠, o Neutral 😐?
    \end{center}
    
    \vfill
    
    Vamos a explorar cómo una máquina puede responder a esta pregunta, una tarea fundamental conocida como \textbf{Análisis de Sentimientos} (o minería de opiniones).
\end{frame}

%----------------------------------------------------------------------------------------
\section{¿Qué es el Análisis de Sentimientos?}
%----------------------------------------------------------------------------------------

\begin{frame}{Definición y Relevancia}
    \begin{block}{Definición}
        Es el proceso de usar NLP para identificar y extraer información subjetiva de un texto. En términos simples, es la tarea de \textbf{clasificar la polaridad de un texto}.
    \end{block}
    
    \textbf{¿Por qué es tan importante?}
    \begin{columns}[T]
        \begin{column}{0.5\textwidth}
            \textbf{Para las Empresas:}
            \begin{itemize}
                \item ¿Qué opinan los clientes de mi nuevo producto?
                \item Monitorear la reputación de la marca en redes sociales.
                \item Entender la satisfacción del cliente a partir de encuestas.
            \end{itemize}
        \end{column}
        \begin{column}{0.5\textwidth}
            \textbf{Para el Público:}
            \begin{itemize}
                \item Analizar la opinión pública sobre un tema político.
                \item Resumir críticas de películas o libros.
                \item Detectar ciberacoso o discurso de odio.
            \end{itemize}
        \end{column}
    \end{columns}
\end{frame}

%-------------------------------------------------
\begin{frame}{Enfoques para el Análisis de Sentimientos}
    Existen varias formas de abordar este problema. Hoy nos centraremos en la más directa:
    
    \begin{alertblock}{Enfoque Basado en Léxicos (Diccionarios)}
        \begin{itemize}
            \item \textbf{La Idea:} Se utiliza un diccionario pre-construido (léxico) donde cada palabra tiene una puntuación de polaridad.
            \item \textbf{Ejemplo de Léxico:}
                \begin{itemize}
                    \item ''excelente'': +0.9
                    \item ''bueno'': +0.6
                    \item ''malo'': -0.7
                    \item ''terrible'': -0.9
                \end{itemize}
            \item \textbf{¿Cómo funciona?} El algoritmo analiza un texto, busca las palabras en el léxico, y calcula una puntuación final sumando las polaridades.
        \end{itemize}
    \end{alertblock}
    
    \vfill
    \begin{block}{Otro Enfoque: Basado en Machine Learning}
        En este caso, en lugar de un diccionario, se entrena un modelo con miles de ejemplos de textos ya etiquetados como positivos o negativos.
    \end{block}
\end{frame}

%----------------------------------------------------------------------------------------
\section{El Algoritmo VADER}
%----------------------------------------------------------------------------------------

\begin{frame}{VADER: Una Herramienta Especializada}
    \begin{block}{VADER: \textit{Valence Aware Dictionary and sEntiment Reasoner}}
        Es un modelo basado en léxico que está específicamente \textbf{optimizado para analizar sentimientos en textos de redes sociales}.
    \end{block}
    
    \textbf{¿Qué lo hace especial?}
    \begin{itemize}
        \item \textbf{Considera la puntuación:} Reconoce que ''!!!'' intensifica un sentimiento.
        \item \textbf{Considera las mayúsculas:} Entiende que ''ODIO'' es más intenso que ''odio''.
        \item \textbf{Maneja negaciones:} Sabe que ''no es bueno'' invierte la polaridad de ''bueno''.
        \item \textbf{Reconoce modificadores:} Entiende que ''muy bueno'' es más positivo que solo ''bueno''.
    \end{itemize}
    
    \begin{alertblock}{En MATLAB..}
        ¡Tenemos una función lista para usar que implementa VADER! Se llama \texttt{vaderSentimentScores}.
    \end{alertblock}
\end{frame}

%-------------------------------------------------
\begin{frame}{Interpretando las Puntuaciones de VADER}
    La función `vaderSentimentScores` no devuelve una, sino \textbf{cuatro} puntuaciones para cada texto:
    
    \begin{itemize}
        \item \texttt{Positive}: La suma de las puntuaciones de las palabras positivas.
        \item \texttt{Negative}: La suma de las puntuaciones de las palabras negativas.
        \item \texttt{Neutral}: La proporción de palabras que no son ni positivas ni negativas.
        \item \texttt{Compound}: ¡La más importante! Es una puntuación \textbf{normalizada y agregada} de todas las demás, que va de -1 (muy negativo) a +1 (muy positivo).
    \end{itemize}
    
    \vfill
    
    \begin{alertblock}{Regla General para Clasificar con \texttt{Compound}}
        \begin{itemize}
            \item Si \texttt{Compound} $\geq$ 0.05 $\rightarrow$ \textbf{Positivo} 😄
            \item Si \texttt{Compound} $\leq$ -0.05 $\rightarrow$ \textbf{Negativo} 😠
            \item Si no, es $\rightarrow$ \textbf{Neutral} 😐
        \end{itemize}
    \end{alertblock}
\end{frame}

%----------------------------------------------------------------------------------------
\section{Ejemplo Práctico en MATLAB}
%----------------------------------------------------------------------------------------

\begin{frame}[fragile]{Puntuando Comentarios con VADER}
    \begin{block}{Objetivo}
    Usar la función \texttt{vaderSentimentScores} para clasificar automáticamente un conjunto de comentarios.
    \end{block}
\end{frame}
\begin{frame}[fragile]{Puntuando Comentarios con VADER}

\begin{lstlisting}[language=Matlab]
% 1. Datos de ejemplo
comentarios = [
    "Amo este producto, es absolutamente fantastico.",
    "Estoy seguro de que no me gusta.",
    "El servicio fue horrible, nunca volvere.",
    "El paquete llego a tiempo."
];

% 2. Tokenizar los comentarios
documentos = tokenizedDocument(comentarios);

% 3. Obtener las puntuaciones de sentimiento
puntuaciones = vaderSentimentScores(documentos);

% 4. Mostrar la tabla de resultados
disp(puntuaciones)
\end{lstlisting}
\end{frame}

%-------------------------------------------------
\begin{frame}[fragile]{Análisis del Resultado}
    \begin{block}{Salida del Código: La Tabla de Puntuaciones}
    MATLAB nos devuelve una tabla muy clara con las 4 puntuaciones para cada comentario.
\begin{lstlisting}
>> disp(puntuaciones)
    0.1027
   -0.2960
   -0.5423
    0
\end{lstlisting}
    \end{block}
    
    \pause
    
    \begin{alertblock}{La clave está en la columna `Compound`}
        Ahora, podemos usar nuestra regla para interpretar estos números.
    \end{alertblock}
\end{frame}

\begin{frame}{Interpretando el Resultado Final}
    \textbf{Vamos a clasificar cada comentario basándonos en la puntuación \texttt{Compound}:}
    
    \begin{itemize}
        \item \textbf{Comentario 1:} ''Amo este producto...''
        \begin{itemize}
            \item \texttt{Compound} = \textbf{0.1027} ($\geq$ 0.05) $\rightarrow$ \textbf{POSITIVO} 😄
        \end{itemize}
        \vfill
        \item \textbf{Comentario 2:} ''Estoy seguro...''
        \begin{itemize}
            \item \texttt{Compound} = \textbf{-0.2969} ($\leq$ -0.05) $\rightarrow$ \textbf{NEGATIVO} 😠
        \end{itemize}
        \vfill
        \item \textbf{Comentario 3:} ''El servicio fue horrible...''
        \begin{itemize}
            \item \texttt{Compound} = \textbf{-0.5423} ($\leq$ -0.05) $\rightarrow$ \textbf{NEGATIVO} 😠
        \end{itemize}
         \vfill
        \item \textbf{Comentario 4:} ''El paquete llego a tiempo.''
        \begin{itemize}
            \item \texttt{Compound} = \textbf{0} (entre -0.05 y 0.05) $\rightarrow$ \textbf{NEUTRAL} 😐
        \end{itemize}
    \end{itemize}
    
    \begin{alertblock}{Resultaos}
    Con solo unas pocas líneas de código, hemos construido un sistema funcional de análisis de sentimientos.
    \end{alertblock}
\end{frame}


%----------------------------------------------------------------------------------------
\section{Conclusiones y Próximos Pasos}
%----------------------------------------------------------------------------------------

\begin{frame}{Resumen y Conclusiones}
    \begin{itemize}
        \item \textbf{Aplicación Directa:} El Análisis de Sentimientos es una de las tareas más valiosas y directas del NLP.
        \vfill
        \item \textbf{Léxicos como Herramienta:} Los enfoques basados en léxicos (diccionarios) son una forma rápida y efectiva de analizar sentimientos sin necesidad de entrenar un modelo.
        \vfill
        \item \textbf{VADER:} Es una herramienta poderosa, especialmente para texto informal como el de redes sociales, y es muy fácil de usar en MATLAB.
        \vfill
        \item \textbf{Logro de Hoy:} Pasamos de tener datos numéricos a obtener \textbf{conclusiones accionables} sobre el contenido de un texto.
    \end{itemize}
\end{frame}

%-------------------------------------------------
\begin{frame}{Próximos Pasos}
    \begin{block}{La Próxima Frontera: Construir un Modelo propio}
        Usar una herramienta pre-entrenada como VADER es fantástico, pero... ¿y si quisiéramos clasificar texto en categorías que VADER no conoce? Por ejemplo, clasificar noticias en ''Deportes'', ''Tecnología'' o ''Negocios''.
        
        En la próxima y última clase, daremos el paso final: aprenderemos a \textbf{entrenar, probar y evaluar nuestro propio modelo de Machine Learning} para clasificar documentos desde cero.
    \end{block}

\end{frame}

\end{document}