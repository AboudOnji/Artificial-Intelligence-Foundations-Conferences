\documentclass[aspectratio=169,xcolor=dvipsnames]{beamer}
\usetheme{SimpleDarkBlue}

\usepackage[spanish]{babel} % Cambiado a español para acentos y textos automáticos
\usepackage{hyperref}
\usepackage{graphicx}
\usepackage{booktabs}
\usepackage{amsmath}
\usepackage{lettrine}
\setbeamertemplate{caption}[numbered]
\usepackage[dvipsnames,svgnames,x11names]{xcolor}
\usepackage{xurl}
\usepackage{hyperref}
\usepackage{algorithm}
\usepackage{algorithmicx}
\usepackage{algpseudocode}
\usepackage{adjustbox}
\hypersetup{
    colorlinks=true,
    linkcolor=blue, % Color más visible en temas oscuros
    filecolor=blue,
    urlcolor=blue,
    citecolor=blue,
}
%----------------------------------------------------------------------------------------
\usepackage{listings}
\usepackage{xcolor}

\definecolor{codegreen}{rgb}{0,0.6,0}
\definecolor{codegray}{rgb}{0.5,0.5,0.5}
\definecolor{codepurple}{rgb}{0.58,0,0.82}
\definecolor{backcolour}{rgb}{0.97,0.97,0.99}

\lstdefinestyle{MATLABStyle}{
  language=Matlab,
  basicstyle=\ttfamily\footnotesize,
  keywordstyle=\color{blue}\bfseries,
  commentstyle=\color{codegreen},
  stringstyle=\color{violet},
  numberstyle=\tiny\color{gray},
  breakatwhitespace=false,
  breaklines=true,
  captionpos=b,
  keepspaces=true,
  numbers=left,
  numbersep=5pt,
  showspaces=false,
  showstringspaces=false,
  showtabs=false,
  tabsize=2,
  frame=lines,
  framerule=0.4pt,
  backgroundcolor=\color{backcolour}
}
\lstset{style=MATLABStyle}
%----------------------------------------------------------------------------------------
%   TITLE PAGE
%----------------------------------------------------------------------------------------
\title{Procesamiento de Lenguaje Natural (NLP)} 
\subtitle{Clase 1: Introducción y Preprocesamiento de Texto} 
\author{Prof. D.Sc. BARSEKH-ONJI Aboud}
\institute{
    Faculty of Engineering \\
    Universidad Anáhuac México Sur
}
\date{\today}
%----------------------------------------------------------------------------------------
\AtBeginSection[]
{
  \begin{frame}{Agenda}
    \tableofcontents[currentsection]
  \end{frame}
}
%----------------------------------------------------------------------------------------
\begin{document}

%-------------------------------------------------
% DIAPOSITIVA DE TÍTULO
%-------------------------------------------------
\begin{frame}[plain] % 'plain' para quitar encabezado y pie de página
    \titlepage
\end{frame}

%-------------------------------------------------
% DIAPOSITIVA DE AGENDA GENERAL
%-------------------------------------------------
\begin{frame}{Agenda de la Clase}
    \tableofcontents
\end{frame}

%----------------------------------------------------------------------------------------
\section{Introducción al NLP}
%----------------------------------------------------------------------------------------

\begin{frame}{¿Qué es el Procesamiento de Lenguaje Natural?}
    \begin{block}{Definición}
        Es un campo de la Inteligencia Artificial (IA) que permite a las computadoras \textbf{comprender, interpretar y manipular el lenguaje humano}.
    \end{block}
    
    \begin{itemize}
        \item Busca cerrar la brecha entre la comunicación humana y la comprensión de las computadoras.
        \item No se trata solo de palabras, sino de entender el \textbf{contexto}, la \textbf{intención} y el \textbf{sentimiento}.
    \end{itemize}
    
    \vfill
    \begin{center}
        \LARGE
        Lenguaje Humano \quad $\Longleftrightarrow$ \quad Lenguaje de Máquina
    \end{center}
\end{frame}

%-------------------------------------------------
\begin{frame}{Aplicaciones en el Mundo Real}
    \begin{columns}[T] % 'T' para alinear por arriba
        \begin{column}{0.5\textwidth}
            \begin{itemize}
                \item \textbf{Traductores Automáticos:}\\ Google Translate, DeepL.
                \vfill
                \item \textbf{Asistentes Virtuales:}\\ Siri, Alexa, Google Assistant.
                \vfill
                \item \textbf{Análisis de Sentimientos:}\\ ¿Qué opinan los clientes de un producto en redes sociales?
            \end{itemize}
        \end{column}
        \begin{column}{0.5\textwidth}
            \begin{itemize}
                \item \textbf{Chatbots:}\\ Atención al cliente automatizada.
                \vfill
                \item \textbf{Clasificación de Texto:}\\ Organizar correos en spam/no spam.
                \vfill
                \item \textbf{Resumen de Textos:}\\ Crear un resumen de un largo artículo de noticias.
            \end{itemize}
        \end{column}
    \end{columns}
\end{frame}

%----------------------------------------------------------------------------------------
\section{El Desafío del Lenguaje}
%----------------------------------------------------------------------------------------

\begin{frame}{¿Por qué es tan difícil para una máquina?}
    El lenguaje humano es inherentemente complejo y ambiguo.
    
    \begin{alertblock}{El gran problema: La Ambigüedad}
        Consideremos la frase: \textit{''Vi a un hombre en la colina con un telescopio.''}
    \end{alertblock}
    
    \begin{itemize}
        \item ¿Quién tiene el telescopio?
        \begin{itemize}
            \item ¿Lo estaba usando yo para ver al hombre?
            \item ¿El hombre en la colina tenía el telescopio?
        \end{itemize}
        \vfill
        \item Los humanos usamos el \textbf{contexto} para resolver esto. Las máquinas necesitan ser entrenadas para hacerlo.
    \end{itemize}
    
    Otros desafíos incluyen la ironía, el sarcasmo, las jergas y la evolución constante del lenguaje.
\end{frame}

%----------------------------------------------------------------------------------------
\section{Preprocesamiento de Texto: La Limpieza de Datos}
%----------------------------------------------------------------------------------------

\begin{frame}{La Base de Todo Proyecto de NLP}
    \begin{block}{Analogía}
        El preprocesamiento de texto es como la \textit{mise en place} en la cocina: antes de poder cocinar, debes lavar, pelar y cortar los ingredientes.
    \end{block}
    
    \begin{itemize}
        \item El texto del mundo real es ''sucio'': contiene mayúsculas, puntuación, palabras irrelevantes, etc.
        \item Nuestro objetivo es \textbf{estandarizar y limpiar} el texto para que los algoritmos de IA puedan procesarlo de manera eficiente.
    \end{itemize}
    \vfill
    \begin{center}
        \textbf{Texto Crudo $\rightarrow$ Limpieza $\rightarrow$ Texto Procesado}
    \end{center}
\end{frame}

%-------------------------------------------------
\begin{frame}{Pasos Clave del Preprocesamiento}
    \begin{enumerate}
        \item \textbf{Tokenización:} Dividir el texto en unidades más pequeñas (palabras o ''tokens'').
        \pause
        \item \textbf{Normalización:} Convertir todo a minúsculas.
        \pause
        \item \textbf{Eliminación de \textit{Stop Words}:} Quitar palabras comunes sin significado (ej. ''el'', ''y'', ''un'').
        \pause
        \item \textbf{\textit{Stemming} (Derivación):} Reducir las palabras a su raíz.
    \end{enumerate}
    
    \vfill
    \begin{alertblock}{}
        Veremos cada uno de estos pasos en detalle y luego los aplicaremos en MATLAB.
    \end{alertblock}
\end{frame}

%-------------------------------------------------
\begin{frame}{1. Tokenización}
    \begin{block}{Objetivo}
        Segmentar un texto en sus componentes básicos, llamados \textbf{tokens}. Generalmente, un token es una palabra.
    \end{block}
    
    \textbf{Ejemplo:}
    \vfill
    \begin{center}
        \large
        ''El NLP es fascinante.''
        \vspace{1cm}
        $\downarrow$
        \vspace{1cm}
        \fbox{El} \quad \fbox{NLP} \quad \fbox{es} \quad \fbox{fascinante} \quad \fbox{.}
    \end{center}
 \begin{alertblock}{En MATLAB..}
    La función \texttt{'tokenizedDocument'} se encarga de esto y mucho más.
 \end{alertblock}
\end{frame}

%-------------------------------------------------
\begin{frame}{2. Normalización (a Minúsculas)}
    \begin{block}{Objetivo}
        Asegurar que la misma palabra se trate de igual forma, sin importar si está en mayúsculas o minúsculas.
    \end{block}
    
    \textbf{Ejemplo:}
    \begin{itemize}
        \item El algoritmo vería ''Hola'', ''hola'' y ''HOLA'' como tres palabras diferentes.
        \item Al convertir todo a minúsculas, las tres se convierten en \textbf{''hola''}.
    \end{itemize}
    \vfill
    \begin{center}
        \large
        \textbf{Hola} mundo $\rightarrow$ \textbf{hola} mundo
    \end{center}
\begin{alertblock}{En MATLAB..}
    usamos la función \texttt{'lower()'}
\end{alertblock}
    
\end{frame}

%-------------------------------------------------
\begin{frame}{3. Eliminación de \textit{Stop Words}}
    \begin{block}{Objetivo}
        Eliminar palabras extremadamente comunes que no aportan un significado semántico relevante al texto.
    \end{block}
    
    \begin{itemize}
        \item Ejemplos en español: \textit{el, la, y, a, un, pero, por, para...}
        \item Estas palabras son útiles para la gramática humana, pero a menudo son ''ruido'' para los modelos de IA.
    \end{itemize}
    
    \textbf{Ejemplo:}
    \vfill
    \begin{center}
        \large
        ''el perro \textbf{y} el gato corren \textbf{en} el parque''
        \vspace{0.5cm}
        $\downarrow$
        \vspace{0.5cm}
        ''perro gato corren parque''
    \end{center}
\begin{alertblock}{En MATLAB..}
    Usamos la función \texttt{'removeWords()'}.
\end{alertblock}
\end{frame}

%-------------------------------------------------


\begin{frame}{4. \textit{Stemming} (Derivación)}
    \begin{block}{Objetivo}
        Reducir una palabra a su raíz o ''stem'', aunque el resultado no sea siempre una palabra real. Es un método rápido y heurístico.
    \end{block}
    
    \begin{itemize}
        \item Ayuda a agrupar diferentes formas de una misma palabra.
        \item Por ejemplo, queremos que el modelo entienda que ''correr'', ''corriendo'' y ''corrió'' se refieren al mismo concepto.
    \end{itemize}
    
    \textbf{Ejemplo:}
    \vfill
    \begin{center}
        \large
        \begin{tabular}{l l}
             \textbf{aprend}iendo & $\rightarrow$ \textbf{aprend} \\
             \textbf{aprend}er & $\rightarrow$ \textbf{aprend} \\
        \end{tabular}
    \end{center}
\begin{alertblock}{En MATLAB..}
    La función \texttt{'normalizeWords'} con el estilo \texttt{'stem'} realiza esta tarea.
\end{alertblock}
\end{frame}

\begin{frame}{4. \textit{Stemming} (Derivación)}
    
    \begin{columns}[T]
        \begin{column}{0.5\textwidth}
            \begin{alertblock}{1. Stemming (Derivación)}
                \begin{itemize}
                    \item \textbf{Cómo funciona:} Corta el final de las palabras usando reglas. Es rápido y eficiente.
                    \item \textbf{Resultado:} La "raíz" no siempre es una palabra real.
                    \item \textbf{Ejemplo:}
                        \begin{itemize}
                            \item `aprendizaje` $\rightarrow$ `aprendizaj`
                            \item `aprendiendo` $\rightarrow$ `aprend`
                        \end{itemize}
                    \item \textbf{Uso:} \texttt{normalizeWords(doc, 'Style', 'stem')}
                \end{itemize}
            \end{alertblock}
        \end{column}
        
        \begin{column}{0.5\textwidth}
            \begin{block}{2. Lemmatization (Lematización)}
                \begin{itemize}
                    \item \textbf{Cómo funciona:} Usa un diccionario para encontrar la palabra raíz (lema). Es más preciso pero más lento.
                    \item \textbf{Resultado:} Siempre es una palabra real.
                    \item \textbf{Ejemplo:}
                        \begin{itemize}
                            \item `mejores` $\rightarrow$ `bueno`
                            \item `fui` $\rightarrow$ `ir`
                        \end{itemize}
                    \item \textbf{Uso:} \texttt{normalizeWords(doc, 'Style', 'lemma')}
                \end{itemize}
            \end{block}
        \end{column}
    \end{columns}
\end{frame}
\begin{frame}{4. \textit{Stemming} (Derivación)}

    \begin{alertblock}{Nuestra Elección para la Clase}
        Usaremos \textbf{Stemming} porque es ideal para tareas de clasificación: es muy rápido y efectivo para agrupar las palabras clave de un texto.
    \end{alertblock}
\end{frame}
%----------------------------------------------------------------------------------------
\section{Ejemplo Práctico en MATLAB}
%----------------------------------------------------------------------------------------

\begin{frame}[fragile]{Limpiando un Comentario de Principio a Fin}
    \begin{block}{Objetivo}
    Aplicar todos los pasos de preprocesamiento a una oración real usando el Text Analytics Toolbox™ de MATLAB.
    \end{block}
\begin{lstlisting}[language=Matlab]
textoOriginal = 'Los alumnos estan aprendiendo sobre el Procesamiento de Lenguaje Natural y están muy emocionados.';

doc = tokenizedDocument(lower(textoOriginal));

listaStopWords = ["y", "el", "la", "los", "las", "un", "una", "de", "sobre", "estan", "están", "."];
documento = removeWords(doc, listaStopWords);

documentoProcesado = normalizeWords(documento, 'Style', 'stem');

palabrasFinales = string(documentoProcesado.tokenDetails.Token);
\end{lstlisting}
\end{frame}

%-------------------------------------------------
\begin{frame}[fragile]{Análisis del Resultado}
    \begin{columns}[T]
        \begin{column}{0.4\textwidth}
            \textbf{Texto Original:}
            \begin{itemize}
                \item ''Los alumnos estan aprendiendo sobre el Procesamiento de Lenguaje Natural y estan muy emocionados.''
            \end{itemize}
        \end{column}
        
        \begin{column}{0.6\textwidth}
            \begin{alertblock}{Tokens Finales (Salida del Código)}
            \begin{verbatim}
'alumn'
'aprend'
'procesamiento'
'lenguaj'
'natural'
'muy'
'emocion'
            \end{verbatim}
            \end{alertblock}
        \end{column}
    \end{columns}
\end{frame}    

\begin{frame}[fragile]{Análisis del Resultado}
    \begin{block}{Observaciones}
        \begin{itemize}
            \item El texto se ha reducido a sus palabras más significativas.
            \item Las palabras están en su forma raíz, lo que facilita el análisis posterior.
            \item ¡Este texto ''limpio'' ya está listo para ser convertido en números para un modelo de IA!
        \end{itemize}
    \end{block}
\end{frame}

%----------------------------------------------------------------------------------------
\section{Conclusiones y Próximos Pasos}
%----------------------------------------------------------------------------------------

\begin{frame}{Resumen y Conclusiones}
    \begin{itemize}
        \item \textbf{¿Qué es el NLP?} Es el campo que enseña a las máquinas a entender el lenguaje humano.
        \vfill
        \item \textbf{El Preprocesamiento es Clave:} No podemos trabajar con texto ''crudo''. Limpiar y estandarizar los datos es el primer paso y el más importante.
        \vfill
        \item \textbf{Nuestra Caja de Herramientas:} Hemos aprendido 4 técnicas fundamentales:
            \begin{enumerate}
                \item Tokenización
                \item Normalización a minúsculas
                \item Eliminación de Stop Words
                \item Stemming
            \end{enumerate}
    \end{itemize}
\end{frame}

%-------------------------------------------------
\begin{frame}{Próximos Pasos}
    
    \begin{block}{La Próxima Frontera: De Palabras a Números}
        Hemos limpiado nuestras palabras, pero los modelos de Machine Learning no entienden ''alumn'' o ''aprend''. Entienden números.
        
        En la próxima clase, aprenderemos a convertir nuestro texto procesado en \textbf{vectores numéricos} usando técnicas como \textbf{Bag-of-Words} y \textbf{TF-IDF}.
    \end{block}
    
    \vfill
    \begin{center}
        \textit{¡Será el puente definitivo entre el lenguaje y las matemáticas!}
    \end{center}
\end{frame}

\end{document}