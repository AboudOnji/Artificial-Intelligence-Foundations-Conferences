\documentclass[aspectratio=169,xcolor=dvipsnames]{beamer}
\usetheme{SimpleDarkBlue}

\usepackage[spanish]{babel}
\usepackage{hyperref}
\usepackage{graphicx}
\usepackage{booktabs}
\usepackage{amsmath}
\usepackage{lettrine}
\setbeamertemplate{caption}[numbered]
\usepackage[dvipsnames,svgnames,x11names]{xcolor}
\usepackage{xurl}
\usepackage{hyperref}
\usepackage{algorithm}
\usepackage{algorithmicx}
\usepackage{algpseudocode}
\usepackage{adjustbox}
\hypersetup{
    colorlinks=true,
    linkcolor=blue,
    filecolor=blue,
    urlcolor=blue,
    citecolor=blue,
}
%----------------------------------------------------------------------------------------
\usepackage{listings}
\usepackage{xcolor}

\definecolor{codegreen}{rgb}{0,0.6,0}
\definecolor{codegray}{rgb}{0.5,0.5,0.5}
\definecolor{codepurple}{rgb}{0.58,0,0.82}
\definecolor{backcolour}{rgb}{0.97,0.97,0.99}

\lstdefinestyle{MATLABStyle}{
  language=Matlab,
  basicstyle=\ttfamily\footnotesize,
  keywordstyle=\color{blue}\bfseries,
  commentstyle=\color{codegreen},
  stringstyle=\color{violet},
  numberstyle=\tiny\color{gray},
  breakatwhitespace=false,
  breaklines=true,
  captionpos=b,
  keepspaces=true,
  numbers=left,
  numbersep=5pt,
  showspaces=false,
  showstringspaces=false,
  showtabs=false,
  tabsize=2,
  frame=lines,
  framerule=0.4pt,
  backgroundcolor=\color{backcolour}
}
\lstset{style=MATLABStyle}
%----------------------------------------------------------------------------------------
%   TITLE PAGE
%----------------------------------------------------------------------------------------
\title{Procesamiento de Lenguaje Natural (NLP)} 
\subtitle{Clase 2: Vectorización de Texto} 
\author{Prof. D.Sc. BARSEKH-ONJI Aboud}
\institute{
    Faculty of Engineering \\
    Universidad Anáhuac México Sur
}
\date{\today}
%----------------------------------------------------------------------------------------
\AtBeginSection[]
{
  \begin{frame}{Agenda}
    \tableofcontents[currentsection]
  \end{frame}
}
%----------------------------------------------------------------------------------------
\begin{document}

%-------------------------------------------------
% DIAPOSITIVA DE TÍTULO
%-------------------------------------------------
\begin{frame}[plain]
    \titlepage
\end{frame}

%-------------------------------------------------
% DIAPOSITIVA DE AGENDA GENERAL
%-------------------------------------------------
\begin{frame}{Agenda de la Clase}
    \tableofcontents
\end{frame}

%----------------------------------------------------------------------------------------
\section{Repaso y El Siguiente Paso}
%----------------------------------------------------------------------------------------

\begin{frame}{¿Dónde nos quedamos?}
    \begin{block}{Resumen de la Clase 1}
        Aprendimos a tomar texto ''sucio'' y, mediante el \textbf{preprocesamiento}, lo convertimos en una lista de tokens limpios y estandarizados.
    \end{block}
    
    \begin{center}
        \textbf{Texto Crudo} \\
        \textit{''Los alumnos estan aprendiendo...''}
        \vspace{0.5cm}
        $\downarrow$ \quad \textit{Limpieza} \quad $\downarrow$
        \vspace{0.5cm}
        \textbf{Tokens Procesados} \\
        \texttt{'alumn', 'aprend', 'procesamiento', ...}
    \end{center}
    
    \vfill
    \begin{alertblock}{El Problema Actual}
        Nuestros tokens son limpios, ¡pero siguen siendo palabras! Los modelos de Machine Learning no entienden palabras, solo entienden \textbf{números}.
    \end{alertblock}
\end{frame}

%-------------------------------------------------
\begin{frame}{El Puente entre Palabras y Matemáticas}
    \begin{block}{Objetivo de Hoy: La Vectorización}
        El proceso de convertir texto en una representación numérica (vectores o matrices) se llama \textbf{vectorización} o \textit{feature extraction}.
    \end{block}
    
    \begin{center}
        \LARGE
        Tokens Limpios \quad $\xrightarrow{\text{Vectorización}}$ \quad Matriz Numérica
    \end{center}
    
    \vfill
    Hoy aprenderemos dos técnicas fundamentales para lograr esto: \textbf{Bag-of-Words} y \textbf{TF-IDF}.
\end{frame}

%----------------------------------------------------------------------------------------
\section{Modelo Bag-of-Words (Bolsa de Palabras)}
%----------------------------------------------------------------------------------------

\begin{frame}{Bag-of-Words (BoW): Una Idea Sencilla}
    \begin{block}{Analogía}
        Imagina que tomas todas las palabras de un documento, las metes en una bolsa, las revuelves y luego cuentas cuántas veces aparece cada una. \textbf{El orden no importa, solo la frecuencia.}
    \end{block}

    \textbf{El proceso tiene dos pasos clave:}
    \begin{enumerate}
        \item \textbf{Crear un Vocabulario:} Construir una lista de todas las palabras únicas que aparecen en \textit{todos} nuestros documentos.
        \pause
        \item \textbf{Contar Frecuencias:} Para cada documento, contar cuántas veces aparece cada palabra del vocabulario.
    \end{enumerate}
\end{frame}

%-------------------------------------------------
\begin{frame}{Bag-of-Words: Ejemplo Visual}
    Supongamos que tenemos dos documentos (después de preprocesar):
    \begin{itemize}
        \item \textbf{Doc 1:} ''gato persigue raton''
        \item \textbf{Doc 2:} ''perro persigue gato''
    \end{itemize}
    \pause
    
    \begin{enumerate}
        \item \textbf{Vocabulario único:} \{gato, persigue, raton, perro\}
        \pause
        \item \textbf{Contar frecuencias para crear vectores:}
    \end{enumerate}
    
    \begin{table}
    \centering
    \begin{tabular}{l c c c c}
        \toprule
        & \textbf{gato} & \textbf{persigue} & \textbf{raton} & \textbf{perro} \\
        \midrule
        \textbf{Doc 1} & 1 & 1 & 1 & 0 \\
        \textbf{Doc 2} & 1 & 1 & 0 & 1 \\
        \bottomrule
    \end{tabular}
    \end{table}
    
    \begin{alertblock}{}
        ¡Hemos convertido el texto en una matriz numérica! Cada fila es un documento.
    \end{alertblock}
\end{frame}

%-------------------------------------------------
\begin{frame}{Limitaciones del Modelo BoW}
    \begin{block}{El Problema: Todas las palabras pesan lo mismo}
        En el modelo BoW simple, la palabra ''excelente'' tiene la misma importancia que la palabra ''producto'' si ambas aparecen una vez.
    \end{block}
    
    \textbf{Consideremos estas frases:}
    \begin{itemize}
        \item ''El producto es \textbf{excelente}.''
        \item ''El producto es \textbf{terrible}.''
    \end{itemize}
    
    \vfill
    Intuitivamente, sabemos que ''excelente'' y ''terrible'' son más importantes para determinar la opinión que ''producto''.
    
    \begin{alertblock}{¿Cómo podemos dar más peso a las palabras más significativas?}
        La respuesta es \textbf{TF-IDF}.
    \end{alertblock}
\end{frame}

%----------------------------------------------------------------------------------------
\section{Ponderación TF-IDF}
%----------------------------------------------------------------------------------------

\begin{frame}{TF-IDF: Dando Peso a lo Importante}
    \begin{block}{Definición}
        \textbf{TF-IDF} significa \textit{Term Frequency - Inverse Document Frequency}. Es una puntuación numérica que refleja qué tan importante es una palabra para un documento dentro de una colección de documentos.
    \end{block}
    
    La puntuación se compone de dos partes:
    
    \begin{columns}[T]
        \begin{column}{0.5\textwidth}
            \textbf{1. TF (Frecuencia del Término):}
            \begin{itemize}
                \item ¿Qué tan seguido aparece la palabra en \textbf{un documento}?
                \item Si una palabra aparece mucho, es importante \textit{para ese documento}.
            \end{itemize}
        \end{column}
        \begin{column}{0.5\textwidth}
            \textbf{2. IDF (Frecuencia Inversa de Documento):}
            \begin{itemize}
                \item ¿Qué tan rara es la palabra en \textbf{toda la colección}?
                \item Si una palabra aparece en muchos documentos (como ''producto''), es menos informativa y recibe un peso bajo.
            \end{itemize}
        \end{column}
    \end{columns}
\end{frame}

%-------------------------------------------------
\begin{frame}{La Intuición del TF-IDF}
    \begin{center}
        \LARGE
        Puntuación TF-IDF = TF $\times$ IDF
    \end{center}
    
    \vfill
    
    \begin{alertblock}{La Puntuación TF-IDF es alta cuando...}
        Una palabra aparece \textbf{muchas veces} dentro de un documento (\textit{TF alto}), pero \textbf{pocas veces} en el resto de los documentos de la colección (\textit{IDF alto}).
    \end{alertblock}

    \vfill
    
    \begin{itemize}
        \item Esto resalta las palabras que son \textbf{distintivas y características} de un documento en particular.
        \item Por ejemplo, en un artículo sobre "Inteligencia Artificial", la palabra "neuronal" tendrá un TF-IDF alto. En un conjunto de noticias, la palabra ''el'' tendrá un TF-IDF muy bajo.
    \end{itemize}
\end{frame}


%----------------------------------------------------------------------------------------
\section{Ejemplo Práctico en MATLAB}
%----------------------------------------------------------------------------------------

\begin{frame}[fragile]{Creando una Matriz TF-IDF en MATLAB}
    \begin{block}{Objetivo}
    Convertir un conjunto de comentarios de texto en una matriz numérica TF-IDF, lista para ser usada en un modelo de IA.
    \end{block}
\end{frame}
\begin{frame}[fragile]{Creando una Matriz TF-IDF en MATLAB}

\begin{lstlisting}[language=Matlab]
% 1. Datos de ejemplo: 3 comentarios de un producto
textos = [
    "La bateria dura mucho, excelente producto.",
    "La pantalla es grande y la camara es buena.",
    "La bateria no dura nada, un producto terrible."
];

% 2. Preprocesamos los textos (usando los pasos de la Clase 1)
documentos = tokenizedDocument(lower(textos));
documentos = erasePunctuation(documentos);
listaStopWords = ["y", "la", "el", "es", "un", "una"];
documentos = removeWords(documentos, listaStopWords);
documentos = normalizeWords(documentos, 'Style', 'stem');

% 3. Crear el modelo de Bag-of-Words
bolsa = bagOfWords(documentos);

% 4. Convertir los documentos a una matriz TF-IDF
matrizTfidf = tfidf(bolsa);
\end{lstlisting}
\end{frame}

%-------------------------------------------------
\begin{frame}[fragile]{Análisis del Resultado}
    \begin{block}{Paso 1: El Vocabulario Creado}
    La función `bagOfWords` crea el vocabulario automáticamente.
\begin{lstlisting}[language=Matlab]
>> disp(bolsa.Vocabulary)

'bateri'    'buen'    'camar'    'dur'    'excelent' ... 
'grand'    'much'    'nad'    'pantall'   'product'    'terribl'
\end{lstlisting}
    \end{block}
\end{frame}

\begin{frame}[fragile]{Análisis del Resultado}
    \begin{alertblock}{Paso 2: La Matriz Numérica Final}
    La función `tfidf` genera la matriz, donde cada fila es un comentario y cada columna es una palabra del vocabulario.
\begin{lstlisting}[language=Matlab]
>> disp(full(matrizTfidf))

%       bateri   buen    camar    dur    ...
ans =
   0.2588     0      0    0.2588   ...  (Comentario 1)
        0  0.3536 0.3536       0   ...  (Comentario 2)
   0.2182     0      0    0.2182   ...  (Comentario 3)
\end{lstlisting}
    \end{alertblock}
\end{frame}
\begin{frame}{Interpretando los Números de la Matriz}
    \begin{block}{¿Qué significa cada número?}
        Cada valor en la matriz representa la \textbf{puntuación TF-IDF} de una palabra (columna) para un documento específico (fila). Un número más alto significa que la palabra es más importante o distintiva para ese documento.
    \end{block}

    \textbf{Vamos a analizar un ejemplo:}
    \begin{itemize}
        \item La palabra \texttt{'bateri'} tiene un valor alto en la Fila 1 y Fila 3, pero \textbf{cero} en la Fila 2.
        \begin{itemize}
            \item \textbf{¿Por qué?} Porque la palabra ''batería'' solo aparece en el primer y tercer comentario. El cero indica su ausencia en el segundo.
        \end{itemize}
        \vfill
        \item La palabra \texttt{'excelent'} (en la Fila 1) probablemente tendrá una puntuación TF-IDF más alta que \texttt{'product'}.
        \begin{itemize}
            \item \textbf{¿Por qué?} Aunque ambas aparecen una vez (TF similar), \texttt{'product'} aparece en dos documentos, haciéndola menos única (IDF más bajo). En cambio, \texttt{'excelent'} es una palabra muy distintiva de ese primer comentario.
        \end{itemize}
    \end{itemize}
\end{frame}
\begin{frame}{Interpretando los Números de la Matriz}

    \begin{alertblock}{La Gran Idea}
        No solo contamos palabras; hemos calculado su \textbf{relevancia ponderada}. Esto es lo que permite a un modelo de IA enfocarse en las palabras que realmente definen el significado de un texto.
    \end{alertblock}
\end{frame}
%----------------------------------------------------------------------------------------
\section{Conclusiones y Próximos Pasos}
%----------------------------------------------------------------------------------------

\begin{frame}{Resumen y Conclusiones}
    \begin{itemize}
        \item \textbf{Vectorizar es Esencial:} Es el proceso de convertir palabras en números, un paso obligatorio para aplicar Machine Learning al texto.
        \vfill
        \item \textbf{Bag-of-Words (BoW):} Es un modelo simple y efectivo que representa documentos basándose en la frecuencia de las palabras.
        \vfill
        \item \textbf{TF-IDF:} Es una técnica de ponderación que mejora a BoW al dar más importancia a las palabras que son realmente distintivas de un documento.
        \vfill
        \item \textbf{Logro de Hoy:} ¡Ya sabemos cómo transformar texto crudo en una \textbf{matriz numérica significativa}!
    \end{itemize}
\end{frame}

%-------------------------------------------------
\begin{frame}{Próximos Pasos}

    
    \begin{block}{La Próxima Frontera: Aplicaciones Reales}
        Ahora que nuestros datos de texto están en un formato que una máquina puede ''entender'', podemos empezar a hacer tareas útiles con ellos.
        
        En la próxima etapa, se dará un primer paso en las aplicaciones de NLP: realizando un \textbf{Análisis de Sentimientos} para clasificar automáticamente si un comentario es positivo, negativo o neutral.
    \end{block}
\end{frame}
%----------------------------------------------------------------------------------------

\end{document}