
\documentclass[aspectratio=169,xcolor=dvipsnames]{beamer}
\usetheme{SimpleDarkBlue}

\usepackage[english]{babel}
\usepackage{hyperref}
\usepackage{graphicx} % Allows including images
\usepackage{booktabs} % Allows the use of \toprule, \midrule and \bottomrule in tables
\usepackage{amsmath}
\usepackage{lettrine}
\setbeamertemplate{caption}[numbered]
\usepackage[dvipsnames,svgnames,x11names]{xcolor}% Para definir y usar colores (ej. en hipervínculos)
\usepackage{xurl}
\usepackage{hyperref}       % Para crear hipervínculos internos y externos
\usepackage{algorithm}
\usepackage{algorithmicx}
\usepackage{algpseudocode}
\usepackage{adjustbox}
\hypersetup{
    colorlinks=true,        % Colorear los enlaces en lugar de usar recuadros
    linkcolor=blue,     % Color para enlaces internos (índice, referencias cruzadas)
    filecolor=blue, % Color para enlaces a archivos locales
    urlcolor=blue,      % Color para URLs
    citecolor=blue,     % Color para citas bibliográficas
}
%----------------------------------------------------------------------------------------

\usepackage{listings}
\usepackage{xcolor} % Para colores en listings
 \definecolor{codegreen}{rgb}{0,0.6,0}
 \definecolor{codegray}{rgb}{0.5,0.5,0.5}
 \definecolor{codepurple}{rgb}{0.58,0,0.82}
 \definecolor{backcolour}{rgb}{0.97,0.97,0.99}

\lstdefinestyle{MATLABStyle}{
  language=Matlab,
  basicstyle=\ttfamily\footnotesize,
  keywordstyle=\color{blue}\bfseries,
  commentstyle=\color{codegreen},
  stringstyle=\color{violet},
  numberstyle=\tiny\color{gray},
  breakatwhitespace=false,
  breaklines=true,
  captionpos=b,
  keepspaces=true,
  numbers=left,
  numbersep=5pt,
  showspaces=false,
  showstringspaces=false,
  showtabs=false,
  tabsize=2,
  frame=lines, % Añade un marco alrededor del código
  framerule=0.4pt, % Grosor del marco
  backgroundcolor=\color{backcolour} % Color de fondo suave
}
\lstset{style=MATLABStyle}

\title{Data-Driven Design of a Tree Fuzzy Inference System}
\subtitle{A Case Study in Mortgage Approval}

\author{Prof. D.Sc. BARSEKH-ONJI Aboud}

\institute
{
    Faculty of Engineering \\
    Universidad Anáhuac México Sur % Your institution for the title page
}
\date{\today} % Date, can be changed to a custom date

%----------------------------------------------------------------------------------------
%   PRESENTATION SLIDES
%----------------------------------------------------------------------------------------
\AtBeginSection[]
{
  \begin{frame}{Agenda}
    \tableofcontents[currentsection]
  \end{frame}
}
\begin{document}

\begin{frame}
    % Print the title page as the first slide
    \titlepage
\end{frame}

\begin{frame}{Agenda}
    \tableofcontents
\end{frame}

%----------------------------------------------------------------------------------------
\section{Project Overview}
%----------------------------------------------------------------------------------------

\begin{frame}{The Challenge: Complex, Human-like Decisions}
    \begin{block}{The Problem}
        How can we model a complex decision-making process, like mortgage approval, that relies on imprecise information and expert knowledge?
    \end{block}

    \begin{columns}[T]
        \begin{column}{0.48\textwidth}
            \textbf{Inputs are often "fuzzy":}
            \begin{itemize}
                \item What is a "good" credit score?
                \item What constitutes a "high" income?
                \item When is a down payment "adequate"?
            \end{itemize}
            
        \end{column}
        \begin{column}{0.48\textwidth}
            \begin{alertblock}{The Problem with Single FIS}
                A single FIS with many inputs becomes unmanageable. The number of rules can grow exponentially! ($3^4 = 81$ rules for our case).
            \end{alertblock}
            
        \end{column}
    \end{columns}
\end{frame}

\begin{frame}{The Challenge: Complex, Human-like Decisions}
    \begin{block}{The Problem}
        How can we model a complex decision-making process, like mortgage approval, that relies on imprecise information and expert knowledge?
    \end{block}

    \begin{columns}[T]
        \begin{column}{0.48\textwidth}
           
            \textbf{Solution: Fuzzy Inference Systems (FIS)}
            \begin{itemize}
                \item Use linguistic variables and IF-THEN rules.
                \item Manage uncertainty and complexity.
            \end{itemize}
        \end{column}
        \begin{column}{0.48\textwidth}
            \begin{exampleblock}{The FIS Tree Approach}
                Break the problem down into smaller, interconnected modules. This is a modular, hierarchical approach that is easier to design and understand.
            \end{exampleblock}
        \end{column}
    \end{columns}
\end{frame}

\begin{frame}{System Architecture: A Two-Level FIS Tree}
    \begin{figure}
        \centering
    \includegraphics[width=0.9\textwidth]{CreditFIS_Tree.png} % Replace with a diagram of your FIS tree
    \caption{FIS Tree for Mortgage Approval}
    \end{figure}
    
    % The hierarchical structure of the mortgage approval system.
\end{frame}

%----------------------------------------------------------------------------------------
\section{System Design & Data Generation}
%----------------------------------------------------------------------------------------

\begin{frame}{Building the Initial FIS Tree}
    \begin{block}{Goal}
        Programmatically create the initial FIS tree structure in MATLAB based on expert knowledge. This serves as our baseline model.
    \end{block}
    
    The project includes three main scripts:
    \begin{enumerate}
        \item \textbf{generateData.m}: Creates a synthetic dataset for training and validation.
        \item \textbf{buildFisTree.m}: Constructs the initial tree structure shown previously.
        \item \textbf{learnFis.m}: Uses the data to automatically learn rules.
        \item \textbf{tueFis.m}: Uses the data to automatically tune parameters of the learned system.
    \end{enumerate}
    \begin{exampleblock}{GitHub Repository}
        \url{https://github.com/AboudOnji/FUZZY-LOGIC-CREDIT-APPROVAL-SYSTEM}
    \end{exampleblock}
\end{frame}

\begin{frame}{Data Generation: `generateData.m`}
    \begin{block}{Why Generate Data?}
        Machine learning requires data. We create a synthetic dataset that mimics real-world scenarios to train and validate our system.
    \end{block}

    \begin{itemize}
        \item \textbf{Method:} We define 5 "applicant archetypes" (e.g., Ideal Applicant, High-Risk, Young Professional) to create realistic and diverse data points.
        \item \textbf{Output:} A matrix named \texttt{trainingData} with 200 samples and 5 columns:
    \end{itemize}

    \begin{center}
    \begin{tabular}{lllll}
        \toprule
        \textbf{Col 1} & \textbf{Col 2} & \textbf{Col 3} & \textbf{Col 4} & \textbf{Col 5} \\
        \midrule
        MonthlyIncome & CreditScore & PropertyValue & DownPayment\% & ExpectedDecision \\
        \bottomrule
    \end{tabular}
    \end{center}
    
    \begin{alertblock}{Key Idea}
        The first four columns are the system's inputs. The fifth column is the "correct answer" or target output we want the system to learn.
    \end{alertblock}
\end{frame}


\begin{frame}[fragile]{Building the Tree: \texttt{buildFisTree.m}}
    \begin{block}{Process}
    We create the three FIS components (\texttt{fis1\_Health}, \texttt{fis2\_Risk}, \texttt{fis3\_FinalDecision}) and connect them using the modern \texttt{fistree} function.
    \end{block}
    
    This approach ensures a clear, modular, and verifiable structure.
    
    
\end{frame}


%----------------------------------------------------------------------------------------
\section{Automated Learning and Tuning}
%----------------------------------------------------------------------------------------

\begin{frame}{Why Automate? From Expert Knowledge to Data-Driven Design}
    \begin{columns}[T]
    \begin{column}{0.48\textwidth}
        \begin{alertblock}{Limitations of Manual Design}
            \begin{itemize}
                \item Rules can be subjective.
                \item Tedious to define and tune.
                \item May not be optimal.
            \end{itemize}
        \end{alertblock}
    \end{column}
    \begin{column}{0.48\textwidth}
        \begin{exampleblock}{The Data-Driven Solution}
            \begin{itemize}
                \item Use algorithms to \textbf{learn} the best rules from data.
                \item Automatically \textbf{tune} membership function parameters to minimize error.
            \end{itemize}
        \end{exampleblock}
    \end{column}
    \end{columns}
    \vspace{1cm}
    \begin{block}{Our Tools}
        We use the \texttt{tunefis} function in MATLAB with Particle Swarm Optimization for both learning and tuning.
    \end{block}
\end{frame}


\begin{frame}[fragile]{Stage 1: Learning the Rules (`learnFisTree.m`)}
    \begin{block}{Goal}
    Automatically generate a small, effective set of fuzzy rules from the data, replacing our initial manual rules.
    \end{block}

    
\end{frame}

\begin{frame}{Stage 2: Tuning the Parameters (`tuneFisTree.m`)}
    \begin{block}{Goal}
    Take the system with the newly learned rules and fine-tune all of its parameters (membership functions) to maximize accuracy.
    \end{block}

   
\end{frame}

%----------------------------------------------------------------------------------------
\section{Results and Conclusion}
%----------------------------------------------------------------------------------------

\begin{frame}{Evaluating Performance: Root Mean Squared Error (RMSE)}
    \begin{block}{How do we measure success?}
    We calculate the Root Mean Squared Error (RMSE) between the system's predictions and the expected outputs from our dataset. \textbf{A lower RMSE is better.}
    \end{block}
    \begin{center}
    \begin{tabular}{lc}
        \toprule
        \textbf{System Stage} & \textbf{RMSE} \\
        \midrule
        Initial Manual FIS & $\sim 15.8$ \\
        After Rule Learning & $\sim 9.2$ \\
        \textbf{After Full Tuning} & \textbf{$\sim 4.5$} \\
        \bottomrule
    \end{tabular}
    \end{center}
    \begin{alertblock}{Result}
    The fully automated, data-driven approach significantly improved the model's accuracy compared to the initial manual design.
    \end{alertblock}
\end{frame}

\begin{frame}{Visualizing the Tuning Process}
    \begin{block}{What does "tuning" look like?}
        The optimization algorithm physically shifts and reshapes the membership functions to better fit the patterns in the data.
    \end{block}
\end{frame}



\begin{frame}{Conclusion and Key Takeaways}
    \begin{exampleblock}{What We Accomplished}
        \begin{itemize}
            \item We modeled a complex problem using a modular FIS Tree.
            \item We created a synthetic dataset to train and validate our model.
            \item We used modern MATLAB tools (\texttt{fistree}, \texttt{tunefis}) to automatically learn rules and tune parameters from data.
            \item The data-driven approach resulted in a significantly more accurate system.
        \end{itemize}
    \end{exampleblock}
\end{frame}
\begin{frame}{Conclusion and Key Takeaways}

    \begin{alertblock}{Broader Applications}
        This workflow of combining expert knowledge (initial structure) with data-driven optimization can be applied to many other fields, including:
        \begin{itemize}
            \item Medical diagnosis
            \item Financial risk analysis
            \item Process control and automation
            \item Customer churn prediction
        \end{itemize}
    \end{alertblock}
\end{frame}


\end{document}

