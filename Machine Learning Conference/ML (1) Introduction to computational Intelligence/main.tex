\documentclass{beamer}
\usepackage{ragged2e}
\usepackage{times}
\usepackage{array}
\usepackage{amsmath}  % Paquete principal para ecuaciones avanzadas
\usepackage{amssymb}  % Símbolos matemáticos adicionales
\usepackage{amsfonts} % Fuentes matemáticas adicionales
\usepackage{mathtools}
\usepackage{booktabs} % Para formato APA en tablas
\newcolumntype{L}[1]{>{\raggedright\arraybackslash}p{#1}}
%\usecolortheme{dolphin}%crane, beaver,dolphin,wolverine,seagull,fly,albatross
\setbeamertemplate{caption}[numbered]
\usepackage[english]{babel}
\usepackage{caption}
\usepackage{setspace}
\usepackage{changepage}

\captionsetup{
    font=footnotesize, % Tamaño de la letra para las captions
    labelfont=bf,      % Hace "Figura X" en negrita
    labelsep=colon     % Cambia el separador después de "Figura X"
}
\usepackage[style=apa, backend=biber]{biblatex} % Configuración para APA o el estilo deseado
\addbibresource{sample.bib} % Archivo .bib


\justifying

% Tema para la presentación
\usetheme{Berkeley} %  "Madrid"  "Berlin", "CambridgeUS","Warsaw" etc.
\setbeamertemplate{caption}[numbered]
\addtolength{\topskip}{-1cm} % Reduce el espacio superior
\usefonttheme{professionalfonts}
% Personalizar el footline
\setbeamertemplate{footline}{%
  \hbox{%
    \begin{beamercolorbox}[wd=0.8\paperwidth,ht=2.25ex,dp=1ex,leftskip=1em]{author in head/foot}%
      \usebeamerfont{author in head/foot}BARSEKH-ONJI Aboud, ORCID: 0009-0004-5440-8092
    \end{beamercolorbox}%
    \begin{beamercolorbox}[wd=0.2\paperwidth,ht=2.25ex,dp=1ex,rightskip=1em]{page number in head/foot}%
      \usebeamerfont{ page number in head/foot}\insertframenumber{} / \inserttotalframenumber
    \end{beamercolorbox}%
  }%
}
% Información del título
\institute{\textbf{Universidad Anáhuac México}\\ {Facultad de Ingeniería}}
\title{\textbf {{Introduction to Computational Intelligence}}}
\author{\textbf{Prof. DSc. BARSEKH-ONJI Aboud}}
\date \today


\begin{document}
\justifying

\begin{frame}{}
    \maketitle
    \textbf{\textit{Machine Learning}}
\end{frame}

\begin{frame}{\textbf{What is Machine Learning?}}
\begin{block}{}
\justifying
Machine learning teaches computers to do what comes naturally to humans and animals:\textbf{ learn from experience}. Machine learning algorithms use computational methods to “learn” information directly from data without relying on a predetermined equation as a model. The algorithms adaptively improve their performance as the number of samples available for learning increases.
\end{block}  
\end{frame}

\begin{frame}{\textbf{When we use machine learning?}}
\begin{block}{}
\justifying
\begin{itemize}
\justifying
    \item When a system is too complex for handwritten rules, like in face and speech recognition.
    \item When the rules of a task are constantly changing, as in fraud detection.
    \item And when the nature of the data itself keeps changing, like in automated trading, energy demand forecasting, and predicting shopping trends.
\end{itemize}
\end{block}  
\end{frame}

\begin{frame}{\textbf{How Machine Learning works?}}
\begin{block}{Supervised Learning}
\justifying
which trains a model on known input and output data so that it can predict future outputs.
\end{block}  
\begin{block}{Unsupervised Learning}
\justifying
which finds hidden patterns or intrinsic structures in input data.
\end{block}  
\end{frame}

\begin{frame}{\textbf{How Machine Learning works?}}
\begin{figure}
    \centering
    \includegraphics[width=1\linewidth]{img/MLClss.png}
    \caption{Machine Learning Classification}
    \label{fig:enter-label}
\end{figure}
\end{frame}

\begin{frame}{\textbf{Supervised Learning}}
    \begin{block}
    \justifying
        The aim of supervised machine learning is to build a model that makes predictions based on evidence in the presence of uncertainty. \textbf{A supervised learning algorithm takes a known set of input data and known responses to the data (output)} and trains a model to generate reasonable predictions for the response to new data.
    \end{block}
\end{frame}

\begin{frame}{\textbf{Supervised Learning}}
    \begin{block}{Classification techniques}
    \justifying
        Predict discrete responses, e.g. whether an email is genuine or spam, or whether a tumor is cancerous or benign. Classification models classify input data into categories. Typical applications include medical imaging, speech recognition, and credit scoring.
    \end{block}
    \begin{block}{Regression Techniques}
    \justifying
        Predict continuous responses. e.g., changes in temperature or fluctuations in power demand. Typical applications include electricity load forecasting and algorithmic trading.
    \end{block}
\end{frame}

\begin{frame}{\textbf{Unsupervised Learning}}
    \begin{block}
    \justifying
        Unsupervised learning draws \textbf{inferences} from datasets that do not have labeled responses associated with the input data.
        
        \textbf{Clustering} is the most common unsupervised learning technique. It puts data into different groups based on the shared characteristics of the data.
        
        Clustering is used to find \textbf{hidden groups} in applications such as gene sequence analysis, market research, and object recognition among others.
    \end{block}
\end{frame}

\begin{frame}{\textbf{Unsupervised Learning}}
    \begin{figure}
        \centering
        \includegraphics[width=\linewidth]{img/Clustering.png}
        \caption{Clustering based on patterns}
        \label{fig:enter-label}
    \end{figure}
\end{frame}

\begin{frame}{\textbf{Machine learning algorithms}}
    \begin{figure}
        \centering
        \includegraphics[width=0.7\linewidth]{img/classML.jpg}
        \caption{Machine larning (supervised \& unsupervised algorithms)}
        \label{fig:enter-label}
    \end{figure}
\end{frame}

\begin{frame}{\textbf{Tuning Inference System}}
    \begin{figure}
        \centering
        \includegraphics[width=\linewidth]{img/Blackbox.png}
        \caption{Black Box Model}
        \label{fig:enter-label}
    \end{figure}
\end{frame}

\begin{frame}{\textbf{Black Box System and AI}}
    \begin{figure}
        \centering
        \includegraphics[width=\linewidth]{img/TuningInference.png}
        \caption{Tuning Process}
        \label{fig:enter-label}
    \end{figure}
\end{frame}

\begin{frame}{\textbf{Steps to build an AI model}}
    \begin{figure}
        \centering
        \includegraphics[width=\linewidth]{img/steps.jpg}
        \caption{Steps to build an AI mode}
        \label{fig:enter-label}
    \end{figure}
\end{frame}

\begin{frame}[allowframebreaks]{\textbf{Bibliography}}
    \justifying
    \footnotesize
    \setbeamertemplate{bibliography item}{}
    \nocite{*} 
    \printbibliography
\end{frame}
\end{document}

