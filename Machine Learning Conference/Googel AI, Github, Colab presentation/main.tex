\documentclass{beamer}
\usepackage[utf8]{inputenc}
\usepackage[spanish]{babel}
\usepackage{graphicx}
\usepackage{hyperref}

% Theme - puedes elegir otro si prefieres
\usetheme{Berkeley}
\usecolortheme{default}

\title{Introducción a Google AI, GitHub y Colab}
\author{Prof. DSc. BARSEKH-ONJI Aboud}
\institute{{Universidad Anáhuac México}\\{Facultad de Ingeniería}}
\date{\today}

\begin{document}

% --- Diapositiva de Título ---
\begin{frame}
    \titlepage
\end{frame}

% --- Agenda ---
\begin{frame}
    \frametitle{Agenda}
    \tableofcontents
\end{frame}

% --- Sección 1: Introducción ---
\section{Introducción al Ecosistema}

\begin{frame}
    \frametitle{El Mundo de la IA y el Desarrollo Colaborativo}
    \begin{itemize}
        \item La Inteligencia Artificial (IA) está transformando industrias.
        \item El desarrollo moderno de software, incluida la IA, es altamente colaborativo.
        \item Plataformas como GitHub y herramientas como Google Colab son esenciales.
        \item Google ofrece un amplio conjunto de APIs y herramientas de IA.
    \end{itemize}
    \begin{center}
        % Puedes añadir un logo de Google AI o similar aquí si quieres
        % \includegraphics[width=0.3\textwidth]{logo_google_ai.png}
        \textit{Explorando las herramientas clave para el desarrollo de IA.}
    \end{center}
\end{frame}

% --- Sección 2: GitHub ---
\section{GitHub: El Hogar del Código}

\begin{frame}
    \frametitle{¿Qué es GitHub?}
    \begin{figure}
        % Reemplaza con una imagen relevante si lo deseas
        % \includegraphics[width=0.4\textwidth]{github_logo.png}
        \caption{Logo de GitHub}
    \end{figure}
    \begin{itemize}
        \item<1-> Plataforma de desarrollo colaborativo basada en la web.
        \item<2-> Utiliza \textbf{Git}, un sistema de control de versiones distribuido.
        \item<3-> Permite alojar repositorios de código (públicos y privados).
        \item<4-> Facilita la colaboración:
              \begin{itemize}
                  \item Forks (copias personales de repositorios).
                  \item Pull Requests (propuestas de cambios).
                  \item Issues (seguimiento de tareas y errores).
                  \item Gestión de proyectos.
              \end{itemize}
        \item<5-> Es fundamental para el software de código abierto (\textit{open source}), incluyendo muchos proyectos de IA.
    \end{itemize}
\end{frame}

\begin{frame}
    \frametitle{GitHub en el Mundo de la IA}
    \begin{itemize}
        \item \textbf{Repositorios de Modelos:} Muchos modelos de IA (ej. de TensorFlow Hub, Hugging Face) tienen su código fuente o ejemplos en GitHub.
        \item \textbf{Proyectos de Investigación:} Investigadores publican el código de sus artículos en GitHub para reproducibilidad.
        \item \textbf{Librerías y Frameworks:} El código fuente de librerías como TensorFlow, PyTorch, Keras, Scikit-learn está en GitHub.
        \item \textbf{Ejemplos y Tutoriales:} Abundan repositorios con ejemplos prácticos y tutoriales de IA.
    \end{itemize}
    \begin{alertblock}{Importancia}
        Saber navegar y utilizar GitHub es crucial para cualquier desarrollador o investigador de IA.
    \end{alertblock}
\end{frame}

% --- Sección 3: Google Colaboratory (Colab) ---
\section{Google Colab: Tu Entorno de IA en la Nube}

\begin{frame}
    \frametitle{¿Qué es Google Colaboratory (Colab)?}
     \begin{figure}
        % Reemplaza con una imagen relevante si lo deseas
        % \includegraphics[width=0.3\textwidth]{colab_logo.png}
        \caption{Logo de Google Colab}
    \end{figure}
    \begin{itemize}
        \item<1-> Servicio \textbf{gratuito} de Google Research.
        \item<2-> Permite escribir y ejecutar código Python a través del navegador.
        \item<3-> Basado en \textbf{Jupyter Notebooks}, ideal para ciencia de datos y Machine Learning.
        \item<4-> Ofrece acceso gratuito a recursos computacionales:
            \begin{itemize}
                \item GPUs (Unidades de Procesamiento Gráfico).
                \item TPUs (Unidades de Procesamiento Tensorial).
            \end{itemize}
        \item<5-> No requiere configuración inicial.
        \item<6-> Facilita la colaboración y el compartir notebooks.
    \end{itemize}
\end{frame}

\begin{frame}
    \frametitle{Colab y su Integración con GitHub}
    \begin{itemize}
        \item \textbf{Abrir Notebooks desde GitHub:} Puedes abrir directamente archivos `.ipynb` alojados en repositorios de GitHub dentro de Colab.
            \begin{itemize}
                \item Simplemente reemplaza `github.com` por `colab.research.google.com/github/` en la URL del repositorio o archivo.
            \end{itemize}
        \item \textbf{Guardar Copias en GitHub:} Puedes guardar una copia de tu notebook de Colab directamente en un repositorio de GitHub (requiere autenticación).
        \item \textbf{Clonar Repositorios:} Puedes usar comandos de Git (`!git clone ...`) dentro de una celda de código de Colab para descargar repositorios completos.
    \end{itemize}
    \begin{exampleblock}{Caso de Uso Típico}
        Encontrar un proyecto interesante en GitHub -> Abrir su notebook de ejemplo en Colab -> Ejecutarlo y experimentar con él usando GPUs gratuitas -> Guardar tus modificaciones en tu propio fork de GitHub.
    \end{exampleblock}
\end{frame}

% --- Sección 4: Lenguajes de Programación ---
\section{Lenguajes de Programación Clave}

\begin{frame}
    \frametitle{Python: El Rey de la IA}
    \begin{columns}[T] % Divide la diapositiva en columnas
        \begin{column}{0.6\textwidth}
            \begin{itemize}
                \item<1-> Lenguaje \textbf{dominante} en IA, Machine Learning y Ciencia de Datos.
                \item<2-> Sintaxis clara y legible.
                \item<3-> Vasto ecosistema de librerías:
                    \begin{itemize}
                        \item \textbf{NumPy:} Computación numérica.
                        \item \textbf{Pandas:} Manipulación y análisis de datos.
                        \item \textbf{Matplotlib/Seaborn:} Visualización de datos.
                        \item \textbf{Scikit-learn:} Machine Learning tradicional.
                        \item \textbf{TensorFlow/Keras/PyTorch:} Deep Learning.
                    \end{itemize}
                \item<4-> Es el lenguaje principal usado en Google Colab.
                \item<5-> La mayoría de ejemplos y SDKs de Google AI están en Python.
            \end{itemize}
        \end{column}
        \begin{column}{0.4\textwidth} % Columna para imagen/logo
            \begin{figure}
                 % Reemplaza con una imagen relevante si lo deseas
                 % \includegraphics[width=\textwidth]{python_logo.png}
                 \caption{Logo de Python}
            \end{figure}
        \end{column}
    \end{columns}
\end{frame}

\begin{frame}
    \frametitle{Otros Lenguajes Relevantes}
    \begin{itemize}
        \item \textbf{R:} Popular en estadística y visualización de datos. Menos común para Deep Learning que Python, pero con fuerte presencia académica.
        \item \textbf{JavaScript:} Usado para desplegar modelos de IA en la web (TensorFlow.js) y para interactuar con APIs desde el frontend.
        \item \textbf{Java/Kotlin:} Relevante para aplicaciones Android que integran IA (ML Kit) y para usar APIs de Google Cloud desde backend.
        \item \textbf{Swift:} Para integrar IA en aplicaciones iOS (Core ML, ML Kit).
        \item \textbf{C++:} Usado para optimizar el rendimiento de bajo nivel en librerías de IA y en sistemas embebidos.
    \end{itemize}
    \begin{block}{Enfoque Principal}
        Aunque existen otros lenguajes, \textbf{Python} es el punto de partida esencial para la mayoría de las tareas de IA, especialmente al usar Colab y explorar repositorios de GitHub.
    \end{block}
\end{frame}

% --- Sección 5: Herramientas Google AI/API y GitHub ---
\section{Herramientas Google AI/API y su Conexión con GitHub}

\begin{frame}
    \frametitle{TensorFlow y Keras}
     \begin{figure}
        % Reemplaza con una imagen relevante si lo deseas
        % \includegraphics[width=0.3\textwidth]{tensorflow_logo.png}
        \caption{Logo de TensorFlow}
    \end{figure}
    \begin{itemize}
        \item \textbf{TensorFlow:} Framework open-source líder para Machine Learning y Deep Learning, desarrollado por Google.
        \item \textbf{Keras:} API de alto nivel para construir y entrenar modelos de redes neuronales, integrada en TensorFlow.
        \item \textbf{Conexión con GitHub:}
            \begin{itemize}
                \item El código fuente de TensorFlow y Keras está en GitHub.
                \item \href{https://github.com/tensorflow/models}{TensorFlow Models Zoo}: Repositorio oficial con implementaciones de modelos estado-del-arte.
                \item \href{https://github.com/keras-team/keras-io}{Keras.io Examples}: Repositorio con numerosos ejemplos y tutoriales.
                \item Muchos proyectos de investigación y aplicaciones usan TF/Keras y publican su código en GitHub.
            \end{itemize}
        \item \textbf{Uso en Colab:} TensorFlow y Keras vienen preinstalados en Colab, listos para usar con GPUs/TPUs.
    \end{itemize}
\end{frame}

\begin{frame}
    \frametitle{Vertex AI}
    \begin{itemize}
        \item Plataforma unificada de Machine Learning de Google Cloud.
        \item Cubre todo el ciclo de vida del ML: preparación de datos, entrenamiento, evaluación, despliegue y monitorización de modelos.
        \item Ofrece modelos pre-entrenados (APIs) y la capacidad de entrenar modelos personalizados (AutoML, Custom Training).
        \item \textbf{Conexión con GitHub:}
            \begin{itemize}
                \item \href{https://github.com/GoogleCloudPlatform/vertex-ai-samples}{Vertex AI Samples}: Repositorio oficial con notebooks y código de ejemplo para diversas tareas en Vertex AI.
                \item Muchos ejemplos muestran cómo integrar código de GitHub en pipelines de entrenamiento o despliegue en Vertex AI.
                \item Las SDKs de Vertex AI (principalmente Python) permiten interactuar con la plataforma desde código (que puede estar en GitHub o Colab).
            \end{itemize}
        \item \textbf{Uso en Colab:} Puedes usar la SDK de Vertex AI desde Colab para gestionar recursos y lanzar trabajos en Google Cloud.
    \end{itemize}
\end{frame}

\begin{frame}
    \frametitle{APIs de IA de Google Cloud}
    \begin{itemize}
        \item Google ofrece APIs pre-entrenadas para tareas específicas de IA:
            \begin{itemize}
                \item \textbf{Vision AI:} Análisis de imágenes (detección de objetos, OCR, etc.).
                \item \textbf{Natural Language AI:} Análisis de texto (sentimiento, entidades, sintaxis).
                \item \textbf{Speech-to-Text / Text-to-Speech:} Transcripción de audio y síntesis de voz.
                \item \textbf{Translation AI:} Traducción automática.
                \item \textbf{Video Intelligence AI:} Análisis de video.
            \end{itemize}
        \item \textbf{Conexión con GitHub:}
            \begin{itemize}
                \item \href{https://github.com/googleapis/google-cloud-python}{Google Cloud Client Libraries (Python)}: Repositorio con las librerías cliente para interactuar con estas APIs (y otros servicios de GCP) desde Python. Hay repositorios similares para otros lenguajes (Java, Node.js, Go, etc.).
                \item Abundantes ejemplos en GitHub que muestran cómo llamar a estas APIs desde diferentes lenguajes y aplicaciones.
            \end{itemize}
        \item \textbf{Uso en Colab:} Fácil de usar las librerías cliente de Python desde Colab para experimentar con las APIs (requiere autenticación a Google Cloud).
    \end{itemize}
\end{frame}

\begin{frame}
    \frametitle{Otros Recursos Relevantes}
    \begin{itemize}
        \item \textbf{ML Kit:} SDK para llevar modelos de Machine Learning a aplicaciones móviles (Android/iOS). Los ejemplos y algunos componentes pueden encontrarse en GitHub.
        \item \textbf{TensorFlow Hub (\href{https://tfhub.dev/}{tfhub.dev}):} Repositorio de modelos pre-entrenados listos para usar. Muchos enlazan a su código fuente o ejemplos en GitHub.
        \item \textbf{Google AI Blog y Research Publications:} A menudo enlazan a repositorios de GitHub con el código asociado a nuevas investigaciones o modelos.
    \end{itemize}
\end{frame}

% --- Sección 6: Conclusión ---
\section{Conclusión}

\begin{frame}
    \frametitle{Resumen y Próximos Pasos}
    \begin{itemize}
        \item \textbf{GitHub} es esencial para la colaboración y el acceso al código fuente en IA.
        \item \textbf{Google Colab} proporciona un entorno accesible y potente (con GPUs/TPUs gratuitas) para ejecutar código de IA, especialmente Python.
        \item \textbf{Python} es el lenguaje clave en el ecosistema de IA actual.
        \item \textbf{Google} ofrece herramientas potentes (TensorFlow, Vertex AI, APIs) que a menudo se encuentran, se comparten y se ejemplifican a través de \textbf{GitHub} y se pueden experimentar fácilmente en \textbf{Colab}.
    \end{itemize}
    \begin{alertblock}{Recomendación}
        ¡Exploren! Busquen proyectos de IA en GitHub, ábranlos en Colab, ejecuten el código y empiecen a experimentar. Es la mejor manera de aprender.
    \end{alertblock}
    \begin{center}
        \vspace{0.5cm}
        \textbf{¿Preguntas?}
    \end{center}
\end{frame}

\end{document}
