\documentclass{beamer}
\usepackage[utf8]{inputenc}
\usepackage[spanish]{babel}
\usepackage{graphicx}
\usepackage{hyperref}
\usepackage{amsmath} % Para fórmulas matemáticas
\usepackage{amssymb} % Para símbolos matemáticos
\usepackage{listings}
\usepackage{xcolor} % Necesario para usar colores
% Para MATLAB, también puedes considerar matlab-prettifier
\usepackage{matlab-prettifier}
% No cargamos listings ni verbatim, ni paquetes complejos para cajas

% Theme - puedes elegir otro si prefieres
\usetheme{Berkeley}
\usecolortheme{default}

\title{Introducción Python (for Matlab users)}
\author{Prof. DSc. BARSEKH-ONJI Aboud}
\institute{{Universidad Anáhuac México}\\{Facultad de Ingeniería}}
\date{\today}

\lstset{
    language=Python, % O language=MATLAB si no usas matlab-prettifier
    basicstyle=\small\ttfamily, % Fuente y tamaño del código
    keywordstyle=\color{blue}\bfseries, % Estilo de las palabras clave
    commentstyle=\color{green}\itshape, % Estilo de los comentarios
    stringstyle=\color{red}, % Estilo de las cadenas
    showspaces=false, % No mostrar espacios como caracteres especiales
    showtabs=false, % No mostrar tabulaciones como caracteres especiales
    breaklines=true, % Habilitar saltos de línea automáticos
    numbers=left, % Mostrar números de línea a la izquierda
    numberstyle=\tiny\color{gray}, % Estilo de los números de línea
    frame=topbot, % Poner un marco alrededor del código (otras opciones: leftline, topbot, etc.)
    framesep=5pt, % Espacio entre el marco y el código
    rulesep=5pt, % Espacio entre los números de línea y el código
    backgroundcolor=\color{blue!10}, % Color de fondo (requiere xcolor)
    rulecolor=\color{black} % Color del marco
}
\begin{document}

\begin{frame}
    \titlepage
\end{frame}

\section*{Introducción}
\begin{frame}[fragile]
    \frametitle{Ejemplo de Código Python}
    \begin{lstlisting}
import numpy as np

def factorial(n):
    if n == 0:
        return 1
    else:
        return n * factorial(n-1)

# Calcular el factorial de 5
num = 5
resultado = factorial(num)
print(f"El factorial de {num} es {resultado}")
    \end{lstlisting}
\end{frame}

\begin{frame}[fragile]
    \frametitle{Ejemplo de Código MATLAB}
    \begin{lstlisting}[style=Matlab-editor] % Otros estilos: Matlab-bw, Matlab-Pyglike
% Este es un comentario en MATLAB
a = 10;
b = 20;
c = a + b;
disp(['La suma es: ', num2str(c)]);

if c > 25
    fprintf('La suma es mayor que 25\n');
else
    fprintf('La suma es menor o igual a 25\n');
end
    \end{lstlisting}
\end{frame}

\begin{frame}{¿Por qué considerar Python?}
    \begin{itemize}
        \item \textbf{Gratuito y Código Abierto:} A diferencia de MATLAB, Python y la mayoría de sus librerías científicas son de uso libre.
        \item \textbf{Versatilidad:} Python es un lenguaje de propósito general utilizado en desarrollo web, automatización, machine learning, etc., no solo computación numérica.
        \item \textbf{Gran Comunidad y Ecosistema:} Amplia disponibilidad de librerías (NumPy, SciPy, Matplotlib, Pandas, etc.) y soporte en línea.
        \item \textbf{Legibilidad:} Sintaxis a menudo considerada más clara y fácil de leer.
    \end{itemize}
\end{frame}

\begin{frame}{Objetivo de esta Presentación}
    \begin{itemize}
        \item Identificar las principales diferencias de sintaxis y estructura de código entre MATLAB y Python.
        \item Mostrar equivalencias de operaciones y funciones comunes.
        \item Facilitar la transición y el aprendizaje de Python para quienes ya programan en MATLAB.
    \end{itemize}
\end{frame}

\section*{Diferencias Clave de Sintaxis}

\begin{frame}[fragile]{Comentarios}
    \begin{columns}
        \column{0.5\textwidth}
        \textbf{MATLAB}
        \begin{block}{}
            \texttt{\% Esto es un comentario en MATLAB}
        \end{block}
        \column{0.5\textwidth}
        \textbf{Python}
        \begin{block}{}
        \begin{lstlisting}[language=Python]
print("Hola, mundo!")
        \end{lstlisting}
        \end{block}
    \end{columns}
\end{frame}

\begin{frame}{Fin de Sentencia}
    \begin{columns}
        \column{0.5\textwidth}
        \textbf{MATLAB}
        \begin{block}{}
            \texttt{a = 10; \% El punto y coma suprime la salida}\\
            \texttt{b = 20  \% Sin punto y coma, muestra la salida}
        \end{block}
        \column{0.5\textwidth}
        \textbf{Python}
        \begin{block}{}
            \texttt{\# No se usa punto y coma para}\\
            \texttt{\# finalizar sentencias.}\\
            \texttt{\# La salida se controla con print()}\\
            \texttt{a = 10}\\
            \texttt{b = 20}\\
            \texttt{print(b)}
        \end{block}
    \end{columns}
\end{frame}

\begin{frame}{Bloques de Código}
    \begin{columns}
        \column{0.5\textwidth}
        \textbf{MATLAB}
        \begin{block}{}
            \texttt{if (condicion)}\\
            \texttt{    \% Código si es verdadero}\\
            \texttt{end}\\
            \texttt{}\\
            \texttt{for i = 1:10}\\
            \texttt{    \% Código del bucle}\\
            \texttt{end}
        \end{block}
        \column{0.5\textwidth}
        \textbf{Python}
        \begin{block}{}
            \texttt{\# Se usan dos puntos (:) e INDENTACIÓN}\\
            \texttt{if condicion:}\\
            \texttt{    \# Código si es verdadero}\\
            \texttt{    pass \# Ejemplo de indentación}\\
            \texttt{}\\
            \texttt{for i in range(10):}\\
            \texttt{    \# Código del bucle}\\
            \texttt{    pass \# Ejemplo de indentación}
        \end{block}
    \end{columns}
    \textbf{¡Importante!} La indentación define los bloques de código en Python.
\end{frame}

\begin{frame}{Definición de Funciones}
    \begin{columns}
        \column{0.5\textwidth}
        \textbf{MATLAB}
        \begin{block}{}
            \texttt{function [output1, output2] = myFunction(input1, input2)}\\
            \texttt{    \% Cuerpo de la función}\\
            \texttt{    output1 = input1 + input2;}\\
            \texttt{    output2 = input1 * input2;}\\
            \texttt{end}
        \end{block}
        \column{0.5\textwidth}
        \textbf{Python}
        \begin{block}{}
            \texttt{\# Se usa 'def' y dos puntos (:)}\\
            \texttt{def my\_function(input1, input2):}\\
            \texttt{    \# Cuerpo de la función}\\
            \texttt{    output1 = input1 + input2}\\
            \texttt{    output2 = input1 * input2}\\
            \texttt{    return output1, output2 \# Retorna una tupla}
        \end{block}
    \end{columns}
\end{frame}

\section*{Tipos de Datos y Estructuras}

\begin{frame}{Arrays vs NumPy Arrays}
    \begin{columns}
        \column{0.5\textwidth}
        \textbf{MATLAB (Arrays)}
        \begin{itemize}
            \item El tipo de dato fundamental es la matriz.
            \item Indexación basada en 1.
        \end{itemize}
        \begin{block}{}
            \texttt{A = [1 2; 3 4];}\\
            \texttt{elemento = A(2, 1); \% Elemento en fila 2, columna 1}
        \end{block}
        \column{0.5\textwidth}
        \textbf{Python (NumPy)}
        \begin{itemize}
            \item Se usa la librería NumPy para trabajar con arrays eficientemente.
            \item Indexación basada en 0.
        \end{itemize}
        \begin{block}{}
            \texttt{import numpy as np}\\
            \texttt{A = np.array([[1, 2], [3, 4]])}\\
            \texttt{elemento = A[1, 0] \# Elemento en fila 2, columna 1}
        \end{block}
    \end{columns}
\end{frame}

\begin{frame}{Indexación y Slicing}
    \begin{columns}
        \column{0.5\textwidth}
        \textbf{MATLAB}
        \begin{block}{}
            \texttt{v = [10 20 30 40 50];}\\
            \texttt{primer\_elemento = v(1);}\\
            \texttt{slice = v(2:4); \% Elementos 2, 3, 4}
        \end{block}
        \column{0.5\textwidth}
        \textbf{Python (NumPy)}
        \begin{block}{}
            \texttt{import numpy as np}\\
            \texttt{v = np.array([10, 20, 30, 40, 50])}\\
            \texttt{primer\_elemento = v[0] \# Índice 0}\\
            \texttt{slice = v[1:4] \# Elementos en índices 1, 2, 3}
        \end{block}
    \end{columns}
    \textbf{¡Ojo con la indexación base 0 en Python!} El slice \texttt{a[inicio:fin]} incluye el \texttt{inicio} pero excluye el \texttt{fin}.
\end{frame}

\begin{frame}{Otras Estructuras de Datos en Python}
    \begin{itemize}
        \item \textbf{Listas:} Colecciones ordenadas y mutables (ej: \texttt{[1, 'hola', 3.14]}). Similares a los cell arrays de MATLAB, pero más flexibles.
        \item \textbf{Tuplas:} Colecciones ordenadas e inmutables (ej: \texttt{(1, 2, 3)}). Útiles para agrupar datos relacionados.
        \item \textbf{Diccionarios:} Colecciones desordenadas de pares clave-valor (ej: \texttt{\{'nombre': 'Juan', 'edad': 30\}}). Similares a las estructuras de MATLAB.
        \item \textbf{Sets:} Colecciones desordenadas de elementos únicos.
    \end{itemize}
\end{frame}

\section*{Operaciones Comunes: Equivalencias}

\begin{frame}{Operaciones Elemento a Elemento}
    \begin{columns}
        \column{0.5\textwidth}
      
        \begin{block}{MATLAB}
            \texttt{A = [1 2; 3 4];}\\
            \texttt{B = [5 6; 7 8];}\\
            \texttt{C = A .* B; \% Multiplicación elemento a elemento}\\
            \texttt{D = A ./ B; \% División elemento a elemento}\\
            \texttt{E = A .^ 2; \% Potencia elemento a elemento}
        \end{block}
        \column{0.5\textwidth}
       
        \begin{block}{Python (NumPy)}
            \texttt{import numpy as np}\\
            \texttt{A = np.array([[1, 2], [3, 4]])}\\
            \texttt{B = np.array([[5, 6], [7, 8]])}\\
            \texttt{C = A * B \# Multiplicación elemento a elemento}\\
            \texttt{D = A / B \# División elemento a elemento}\\
            \texttt{E = A ** 2 \# Potencia elemento a elemento}
        \end{block}
    \end{columns}
    En NumPy, los operadores aritméticos estándar ($\texttt{*}$, $\texttt{/}$, $\texttt{**}$) realizan operaciones elemento a elemento en arrays.
\end{frame}

\begin{frame}{Multiplicación Matricial}
    \begin{columns}
        \column{0.5\textwidth}
        \textbf{MATLAB}
        \begin{block}{}
            \texttt{A = [1 2; 3 4];}\\
            \texttt{B = [5 6; 7 8];}\\
            \texttt{C = A * B; \% Multiplicación matricial}
        \end{block}
        \column{0.5\textwidth}
        \textbf{Python (NumPy)}
        \begin{block}{}
            \texttt{import numpy as np}\\
            \texttt{A = np.array([[1, 2], [3, 4]])}\\
            \texttt{B = np.array([[5, 6], [7, 8]])}\\
            \texttt{C = A @ B \# O np.dot(A, B)}
        \end{block}
    \end{columns}
    Python 3.5+ introdujo el operador \texttt{@} para multiplicación matricial. \texttt{np.dot()} también funciona y es compatible con versiones anteriores.
\end{frame}

\begin{frame}{Concatenación de Arrays}
    \begin{columns}
        \column{0.5\textwidth}
        \textbf{MATLAB}
        \footnotesize
        \begin{block}{}
            \texttt{A = [1 2];}\\
            \texttt{B = [3 4];}\\
            \texttt{C = [A B];    \% Concatenar horizontalmente}\\
            \texttt{D = [A; B];   \% Concatenar verticalmente}
        \end{block}
        \column{0.5\textwidth}
        \textbf{Python (NumPy)}
        \footnotesize
        \begin{exampleblock}{}
            \texttt{import numpy as np}\\
            \texttt{A = np.array([1, 2])}\\
            \texttt{B = np.array([3, 4])}\\
            \texttt{C = np.hstack((A, B)) \# Horizontal}\\
            \texttt{D = np.vstack((A, B)) \# Vertical}\\
            \texttt{\# O np.concatenate((A, B), axis=0) para vertical}\\
            \texttt{\# O np.concatenate((A, B), axis=1) para horizontal (si son 2D)}
        \end{exampleblock}
    \end{columns}
\end{frame}

\begin{frame}{Funciones Matemáticas Comunes}
    La mayoría de las funciones matemáticas de MATLAB tienen equivalentes en NumPy.
    \footnotesize
    \begin{itemize}
        \item \texttt{sin(x)} $\rightarrow$ \texttt{np.sin(x)}
        \item \texttt{cos(x)} $\rightarrow$ \texttt{np.cos(x)}
        \item \texttt{sqrt(x)} $\rightarrow$ \texttt{np.sqrt(x)}
        \item \texttt{exp(x)} $\rightarrow$ \texttt{np.exp(x)}
        \item \texttt{log(x)} (base e) $\rightarrow$ \texttt{np.log(x)}
        \item \texttt{log10(x)} $\rightarrow$ \texttt{np.log10(x)}
        \item \texttt{abs(x)} $\rightarrow$ \texttt{np.abs(x)} o \texttt{abs(x)} (para escalares)
        \item \texttt{sum(A)} $\rightarrow$ \texttt{np.sum(A)} (suma todos los elementos por defecto)
        \item \texttt{sum(A, 1)} (suma por columnas en MATLAB) $\rightarrow$ \texttt{np.sum(A, axis=0)}
        \item \texttt{sum(A, 2)} (suma por filas en MATLAB) $\rightarrow$ \texttt{np.sum(A, axis=1)}
    \end{itemize}
\end{frame}

\begin{frame}{Generación de Secuencias y Matrices Especiales}
    \begin{columns}
        \column{0.5\textwidth}
        \textbf{MATLAB}
        \footnotesize
        \begin{block}{}
            \texttt{x = 1:5; \% [1 2 3 4 5]}\\
            \texttt{y = 1:2:10; \% [1 3 5 7 9]}\\
            \texttt{zeros(2, 3);}\\
            \texttt{ones(3, 1);}\\
            \texttt{eye(4);}\\
            \texttt{linspace(0, 10, 50);}\\
            \texttt{logspace(1, 3, 10);}
        \end{block}
        \column{0.5\textwidth}
        \textbf{Python (NumPy)}
        \footnotesize
        \begin{block}{}
            \texttt{import numpy as np}\\
            \texttt{x = np.arange(1, 6) \# [1 2 3 4 5]}\\
            \texttt{y = np.arange(1, 11, 2) \# [1 3 5 7 9]}\\
            \texttt{np.zeros((2, 3)) \# Nota: tupla para dimensiones}\\
            \texttt{np.ones((3, 1))}\\
            \texttt{np.eye(4)}\\
            \texttt{np.linspace(0, 10, 50)}\\
            \texttt{np.logspace(1, 3, 10)}
        \end{block}
    \end{columns}
    \texttt{np.arange} es similar al operador \texttt{:} de MATLAB, pero el final es exclusivo.
\end{frame}

\section*{Gráficos (Matplotlib)}

\begin{frame}{Gráficos Básicos}
    MATLAB tiene funciones de graficado integradas, mientras que en Python se usa la librería Matplotlib (similar a la sintaxis de MATLAB).
    \begin{columns}
        \column{0.5\textwidth}
        \textbf{MATLAB}
        \footnotesize
        \begin{block}{}
            \texttt{x = 0:pi/10:2*pi;}\\
            \texttt{y = sin(x);}\\
            \texttt{plot(x, y);}\\
            \texttt{title('Seno');}\\
            \texttt{xlabel('x');}\\
            \texttt{ylabel('sin(x)');}\\
            \texttt{grid on;}
        \end{block}
        \column{0.5\textwidth}
        \textbf{Python (Matplotlib)}
        \footnotesize
        \begin{block}{}
            \texttt{import matplotlib.pyplot as plt}\\
            \texttt{import numpy as np}\\
            \texttt{x = np.arange(0, 2*np.pi, np.pi/10)}\\
            \texttt{y = np.sin(x)}\\
            \texttt{plt.plot(x, y)}\\
            \texttt{plt.title('Seno')}\\
            \texttt{plt.xlabel('x')}\\
            \texttt{plt.ylabel('sin(x)')}\\
            \texttt{plt.grid(True)}\\
            \texttt{plt.show() \# Para mostrar la gráfica}
        \end{block}
    \end{columns}
    Se recomienda importar \texttt{matplotlib.pyplot} como \texttt{plt}.
\end{frame}

\begin{frame}{Subplots}
    \begin{columns}
        \column{0.5\textwidth}
        \textbf{MATLAB}
        \footnotesize
        \begin{block}{}
            \texttt{t = 0:0.1:10;}\\
            \texttt{y1 = sin(t);}\\
            \texttt{y2 = cos(t);}\\
            \texttt{subplot(2, 1, 1); \% Filas, Cols, Panel}\\
            \texttt{plot(t, y1);}\\
            \texttt{title('Seno');}\\
            \texttt{subplot(2, 1, 2);}\\
            \texttt{plot(t, y2);}\\
            \texttt{title('Coseno');}
        \end{block}
        \column{0.5\textwidth}
        \textbf{Python (Matplotlib)}
        \footnotesize
        \begin{block}{}
            \texttt{import matplotlib.pyplot as plt}\\
            \texttt{import numpy as np}\\
            \texttt{t = np.arange(0, 10, 0.1)}\\
            \texttt{y1 = np.sin(t)}\\
            \texttt{y2 = np.cos(t)}\\
            \texttt{plt.subplot(2, 1, 1) \# Filas, Cols, Panel (base 1)}\\
            \texttt{plt.plot(t, y1)}\\
            \texttt{plt.title('Seno')}\\
            \texttt{plt.subplot(2, 1, 2)}\\
            \texttt{plt.plot(t, y2)}\\
            \texttt{plt.title('Coseno')}\\
            \texttt{plt.tight\_layout() \# Ajusta el espacio}\\
            \texttt{plt.show()}
        \end{block}
    \end{columns}
    La función \texttt{subplot} en Matplotlib sigue la convención de MATLAB (base 1 para el número de panel).
\end{frame}

\section*{Consideraciones Adicionales}

\begin{frame}{Entornos de Desarrollo Integrado (IDE)}
    \begin{itemize}
        \item \textbf{MATLAB:} Viene con su propio entorno robusto.
        \item \textbf{Python:} Existen varias opciones populares:
        \begin{itemize}
            \item \textbf{Spyder:} Muy similar a MATLAB, ideal para la transición.
            \item \textbf{Jupyter Notebook/Lab:} Excelente para análisis interactivo y documentación.
            \item \textbf{VS Code, PyCharm:} IDEs más generales y potentes.
        \end{itemize}
    \end{itemize}
\end{frame}

\begin{frame}{Manejo de Archivos .m}
    No hay una conversión automática perfecta de código MATLAB a Python. A menudo es necesario reescribir el código, aprovechando las capacidades y librerías de Python.
\end{frame}

\begin{frame}{Rendimiento}
    \begin{itemize}
        \item Para operaciones vectorizadas y matriciales, MATLAB a menudo es muy rápido debido a su optimización interna y el uso de librerías matemáticas de bajo nivel.
        \item En Python, el uso eficiente de NumPy es crucial para obtener buen rendimiento en operaciones numéricas.
        \item Para bucles explícitos, Python puro puede ser más lento que MATLAB, pero se pueden usar herramientas como Numba o Cython para acelerar secciones críticas.
    \end{itemize}
\end{frame}

\section*{Recursos Adicionales}

\begin{frame}{Recursos para la Transición}
    \begin{itemize}
        \item \textbf{Documentación de NumPy:} Excelente referencia para funciones matemáticas y manipulación de arrays.
        \item \textbf{Documentación de Matplotlib:} Guías y ejemplos para crear visualizaciones.
        \item \textbf{Guías de transición en línea:} Muchos sitios web ofrecen comparaciones detalladas y tutoriales (buscar "MATLAB to Python migration guide").
        \item \textbf{Cursos en línea:} Plataformas como Coursera, edX, Codecademy, etc., ofrecen cursos de Python, NumPy y Matplotlib.
        \item \textbf{Comunidades en línea:} Stack Overflow, foros de Python y comunidades de librerías específicas.
    \end{itemize}
\end{frame}

\section*{Conclusión}

\begin{frame}{Resumen}
    \begin{itemize}
        \item Python es una alternativa potente y versátil a MATLAB, con un gran ecosistema de librerías para computación científica.
        \item Las principales diferencias radican en la sintaxis (indentación, fin de sentencia, definición de bloques) y la indexación (base 0 en Python).
        \item NumPy proporciona la funcionalidad de arrays y operaciones numéricas similar a MATLAB.
        \item Matplotlib permite crear gráficos de manera similar a MATLAB.
        \item La transición requiere adaptarse a la sintaxis de Python y familiarizarse con las librerías clave.
    \end{itemize}
\end{frame}

\begin{frame}{¡Gracias!}
    \begin{center}
        % Asegúrate de tener el logo en la misma carpeta o especifica la ruta
        % \includegraphics[width=0.3\textwidth]{logo_anahuac.png}
        % Si no tienes la imagen del logo, puedes quitar la línea anterior.
    \end{center}
    \begin{center}
        Preguntas?
    \end{center}
\end{frame}

\end{document}