\documentclass[aspectratio=169,xcolor=dvipsnames]{beamer}
\usetheme{Berlin}

\usepackage[spanish]{babel} % Cambiado a español para acentos y textos automáticos
\usepackage{hyperref}
\usepackage{graphicx}
\usepackage{booktabs}
\usepackage{amsmath}
\usepackage{lettrine}
\setbeamertemplate{caption}[numbered]
\usepackage[dvipsnames,svgnames,x11names]{xcolor}
\usepackage{xurl}
\usepackage{hyperref}
\usepackage{algorithm}
\usepackage{algorithmicx}
\usepackage{algpseudocode}
\usepackage{adjustbox}
\hypersetup{
    colorlinks=true,
    linkcolor=cyan, % Color más visible en temas oscuros
    filecolor=blue,
    urlcolor=blue,
    citecolor=blue,
}
%----------------------------------------------------------------------------------------
\usepackage{listings}
\usepackage{xcolor}

\definecolor{codegreen}{rgb}{0,0.6,0}
\definecolor{codegray}{rgb}{0.5,0.5,0.5}
\definecolor{codepurple}{rgb}{0.58,0,0.82}
\definecolor{backcolour}{rgb}{0.97,0.97,0.99}

\lstdefinestyle{MATLABStyle}{
  language=Matlab,
  basicstyle=\ttfamily\footnotesize,
  keywordstyle=\color{blue}\bfseries,
  commentstyle=\color{codegreen},
  stringstyle=\color{violet},
  numberstyle=\tiny\color{gray},
  breakatwhitespace=false,
  breaklines=true,
  captionpos=b,
  keepspaces=true,
  numbers=left,
  numbersep=5pt,
  showspaces=false,
  showstringspaces=false,
  showtabs=false,
  tabsize=2,
  frame=lines,
  framerule=0.4pt,
  backgroundcolor=\color{backcolour}
}
\lstset{style=MATLABStyle}
%----------------------------------------------------------------------------------------
%   TITLE PAGE
%----------------------------------------------------------------------------------------
\title{Particle Swarm Optimization Algorithm}
\author{Prof. DSc. BARSEKH-ONJI Aboud}
\institute{Universidad Anáhuac México\\ Facultad de Ingeniería}
\date{\today}

\begin{document}

\frame{\titlepage}

% Índice
\begin{frame}{Contenido}
    \tableofcontents
\end{frame}



\section{Introduction}

\begin{frame}{What is PSO?}
    \begin{itemize}
        \item PSO is a population-based stochastic optimization algorithm inspired by social behavior of bird flocking.
        \item Each individual (called a particle) represents a potential solution.
        \item Particles adjust their positions in the search space by learning from their own experience and that of their neighbors.
        \item Developed originally to simulate the choreography of birds.
    \end{itemize}
\end{frame}

\section{Basic PSO Algorithm}

\begin{frame}{Particle Representation and Movement}
    \begin{itemize}
        \item A swarm is a collection of particles.
        \item Each particle has a position $x_i$ and velocity $v_i$.
        \item Position update:
        \begin{equation*}
            x_i(t+1) = x_i(t) + v_i(t+1)
        \end{equation*}
    \end{itemize}
\end{frame}

\begin{frame}{Velocity Update (gbest model)}
    \begin{equation*}
        v_{ij}(t+1) =  v_{ij}(t) + c_1 r_1 (y_{ij}(t) - x_{ij}(t)) + c_2 r_2 (\hat{y}_j(t) - x_{ij}(t))
    \end{equation*}
    \begin{itemize}
        \item $y_{ij}(t)$: personal best
        \item $\hat{y}_j(t)$: global best
        \item $r_1$, $r_2$: random numbers in $[0, 1]$
    \end{itemize}
\end{frame}

\begin{frame}{Global vs. Local Best}
    \begin{itemize}
        \item \textbf{gbest:} global best PSO, fast convergence, less diversity
        \item \textbf{lbest:} local best PSO, slower convergence, more robust
        \item Neighborhoods overlap to share information
    \end{itemize}
    
\end{frame}

\begin{frame}{Inertia Weight in PSO}
    \begin{itemize}
        \item Introduced by Shi and Eberhart to control the balance between exploration and exploitation.
        \item Modifies the velocity equation:
        \[
            v_{ij}(t+1) = w v_{ij}(t) + c_1 r_1 (y_{ij}(t) - x_{ij}(t)) + c_2 r_2 (\hat{y}_j(t) - x_{ij}(t))
        \]
        \item $w$ is the inertia weight. High $w$ encourages exploration; low $w$ promotes exploitation.
    \end{itemize}
\end{frame}

\begin{frame}{Effects and Tuning of Inertia Weight}
    \begin{itemize}
        \item If $w \geq 1$: velocity may diverge, particles can overshoot.\newline
        \item If $w < 1$: velocities decay over time, enabling convergence.\newline
        \item Typically $w \in [0.4, 0.9]$.
        \item Must be chosen in conjunction with $c_1$, $c_2$ to avoid divergence:
        \[
            w > \frac{1}{2}(c_1 + c_2) - 1
        \]
    \end{itemize}
\end{frame}

\begin{frame}{Dynamic Inertia Weight Strategies}
    \begin{itemize}
        \item \textbf{Linear Decrease}:
        \[
            w(t) = w_{\text{max}} - \left(\frac{w_{\text{max}} - w_{\text{min}}}{t_{\text{max}}}\right) t
        \]
        \item \textbf{Nonlinear Decay} (e.g., exponential):
        \[
            w(t) = \alpha w(t-1) \quad \text{with } \alpha \in (0, 1)
        \]
        \item \textbf{Fuzzy Adaptive} and \textbf{Stochastic} adjustments also explored.
        \item Goal: allow wide exploration initially, focus on refinement in later iterations.
    \end{itemize}
\end{frame}


\section{PSO Dynamics}

\begin{frame}{Velocity Components}
    \begin{itemize}
        \item \textbf{Inertia:} memory of previous direction
        \item \textbf{Cognitive:} tendency to return to personal best
        \item \textbf{Social:} tendency to follow best in neighborhood
    \end{itemize}
    \centering\includegraphics[width=0.7\textwidth]{lbest_vs_gbest.png}
\end{frame}

\begin{frame}{Geometric Interpretation-gbest}
    \begin{itemize}
        \item Motion guided by vector components
        \item Balance between exploration and exploitation
    \end{itemize}
    \centering \includegraphics[width=0.7\textwidth]{geometric.png}
\end{frame}

\begin{frame}{Geometric Interpretation-lbest}
    \begin{itemize}
        \item Motion guided by vector components
        \item Balance between exploration and exploitation
    \end{itemize}
    \centering \includegraphics[width=0.7\textwidth]{lbest.png}
\end{frame}

\section{PSO Parameters and Termination}

\begin{frame}{Key Parameters}
    \begin{itemize}
        \item Swarm size $n_s$
        \item Cognitive and social coefficients $c_1$, $c_2$
        \item Inertia weight $w$
        \item Maximum velocity $v_{max}$
    \end{itemize}
\end{frame}

\begin{frame}{Termination Criteria}
    \begin{itemize}
        \item Maximum number of iterations
        \item Acceptable fitness threshold
        \item No improvement over iterations
        \item Normalized swarm radius
        \item Objective function slope \textasciitilde 0
    \end{itemize}
\end{frame}

\section{PSO Variants and Enhancements}

\begin{frame}{Common Variants}
    \begin{itemize}
        \item Inertia weight PSO
        \item Constriction factor PSO
        \item Barebones PSO
        \item Fully Informed PSO (FIPS)
        \item Charged PSO
    \end{itemize}
\end{frame}

\begin{frame}{Hybrid PSO Models}
    \begin{itemize}
        \item PSO with mutation (Gaussian, Cauchy)
        \item PSO with DE (Differential Evolution) crossover
        \item PSO + local search (GCPSO)
        \item PSO with genetic selection pressure
    \end{itemize}
\end{frame}

\begin{frame}{Multi-Swarm and Niching PSOs}
    \begin{itemize}
        \item Multi-swarm: cooperative or competitive subgroups
        \item Predator-prey PSO
        \item Self-organizing and adaptive networks
        \item Split dimension optimization
    \end{itemize}
\end{frame}



\section{PSO in Matlab}
\begin{frame}{PSO in Matlab}
\begin{itemize}
    \item \texttt{particleswarm} is a solver in MATLAB's Global Optimization Toolbox.
    \item It implements Particle Swarm Optimization (PSO) for solving bound-constrained optimization problems.
    \item Useful for problems where the objective function is non-differentiable, noisy, or has multiple local minima.
\end{itemize}
\end{frame}

\begin{frame}{Basic Syntax}
\begin{itemize}
    \item \texttt{x = particleswarm(fun, nvars)} \\
    Minimizes the objective function \texttt{fun} with \texttt{nvars} variables.
    \item \texttt{x = particleswarm(fun, nvars, lb, ub)} \\
    Adds lower (\texttt{lb}) and upper (\texttt{ub}) bounds to the variables.
    \item \texttt{x = particleswarm(fun, nvars, lb, ub, options)} \\
    Allows customization through the \texttt{options} structure.
\end{itemize}
\end{frame}

\begin{frame}{Key Parameters (1/2)}
\begin{itemize}
    \item \textbf{SwarmSize} \\
    Number of particles in the swarm. Default is \texttt{100}.
    \item \textbf{MaxIterations} \\
    Maximum number of iterations allowed. Default is \texttt{100}.
    \item \textbf{FunctionTolerance} \\
    Termination tolerance on the function value. Default is \texttt{1e-6}.
    \item \textbf{InertiaRange} \\
    Range for inertia weight. Controls exploration vs. exploitation.
\end{itemize}
\end{frame}

\begin{frame}{Key Parameters (2/2)}
\begin{itemize}
    \item \textbf{SelfAdjustmentWeight} \\
    Weight for the particle's own best position influence.
    \item \textbf{SocialAdjustmentWeight} \\
    Weight for the swarm's best position influence.
    \item \textbf{HybridFcn} \\
    Function handle for a hybrid function to refine the solution.
    \item \textbf{Display} \\
    Level of display output: \texttt{'off'}, \texttt{'iter'}, or \texttt{'final'}.
\end{itemize}
\end{frame}

\begin{frame}{Velocity Update Equation}
\begin{equation*}
v = W \cdot v + y_1 \cdot u_1 \cdot (p - x) + y_2 \cdot u_2 \cdot (g - x)
\end{equation*}
\begin{itemize}
    \item $v$: Current velocity
    \item $W$: Inertia weight
    \item $y_1$: SelfAdjustmentWeight
    \item $y_2$: SocialAdjustmentWeight
    \item $u_1$, $u_2$: Random vectors in [0,1]
    \item $p$: Particle's best-known position
    \item $g$: Global best-known position
\end{itemize}
\end{frame}

\begin{frame}{Termination Criteria}
\begin{itemize}
    \item Maximum number of iterations reached.
    \item Change in the best function value is less than \texttt{FunctionTolerance}.
    \item No improvement in the best function value for \texttt{MaxStallIterations}.
\end{itemize}
\end{frame}

\begin{frame}{PSO Parameters: Exploration vs. Exploitation}
\begin{table}[ht]
\centering
\footnotesize
\begin{tabular}{p{1.5cm}p{2cm}p{2.5cm}p{2cm}}
\toprule
\textbf{Parameter} & \textbf{Typical Range} & \textbf{Exploration} & \textbf{Exploitation} \\
\midrule
$w$ & 0.4 -- 0.9 & High $w$ (e.g., 0.9) & Low $w$ (e.g., 0.4) \\
$c_1$ & 1.5 -- 2.5 & High $c_1$ & Low $c_1$ \\
$c_2$ & 1.5 -- 2.5 & Low $c_2$ & High $c_2$ \\
\bottomrule
\end{tabular}
\caption{Impact of PSO parameters on exploration and exploitation}
\end{table}
\end{frame}

\begin{frame}{PSO Parameters: Exploration vs. Exploitation}
\begin{itemize}
    \item \textbf{Inertia Weight ($w$)}: Controls the momentum of particles. Higher values encourage global exploration, while lower values promote local exploitation.
    \item \textbf{Cognitive Coefficient ($c_1$)}: Represents the particle's tendency to return to its own best position. Higher values increase individual exploration.
    \item \textbf{Social Coefficient ($c_2$)}: Represents the particle's tendency to move towards the swarm's best position. Higher values increase convergence and exploitation.
\end{itemize}
\end{frame}

\begin{frame}{Best Practices}
\begin{itemize}
    \item Start with default parameters; adjust based on problem complexity.
    \item Use bounds (\texttt{lb}, \texttt{ub}) to confine the search space.
    \item Consider using a hybrid function for fine-tuning the solution.
    \item Monitor convergence using the \texttt{Display} option.
\end{itemize}
\end{frame}

\section{Conclusion}

\begin{frame}{Conclusion}
    \begin{itemize}
        \item PSO is an efficient population-based optimization algorithm
        \item Many variants have improved its robustness and flexibility
        \item Understanding social structure and velocity dynamics is key
        \item Suitable for continuous, discrete, and dynamic environments
    \end{itemize}
\end{frame}

\begin{frame}{References}
\begin{itemize}
    \item MATLAB Documentation: \href{https://www.mathworks.com/help/gads/particleswarm.html}{particleswarm}
    \item MATLAB Documentation: \href{https://www.mathworks.com/help/gads/particle-swarm-options.html}{Particle Swarm Options}
    \item  Engelbrecht, A. P. (2007). \textit{Computational Intelligence: An Introduction}. John Wiley \& Sons
\end{itemize}
\end{frame}



\end{document}
