%----------------------------------------------------------------------------------------
%	PACKAGES AND THEMES
%----------------------------------------------------------------------------------------

\documentclass[aspectratio=169,xcolor=dvipsnames]{beamer}
\usetheme{SimpleDarkBlue}

\usepackage[english]{babel}
\usepackage{hyperref}
\usepackage{graphicx} % Allows including images
\usepackage{booktabs} % Allows the use of \toprule, \midrule and \bottomrule in tables
\usepackage{amsmath}
\usepackage{lettrine}
\setbeamertemplate{caption}[numbered]
\usepackage[dvipsnames,svgnames,x11names]{xcolor}% Para definir y usar colores (ej. en hipervínculos)
\usepackage{xurl}
\usepackage{hyperref}       % Para crear hipervínculos internos y externos
\usepackage{algorithm}
\usepackage{algorithmicx}
\usepackage{algpseudocode}
\usepackage{adjustbox}
\hypersetup{
    colorlinks=true,        % Colorear los enlaces en lugar de usar recuadros
    linkcolor=blue,     % Color para enlaces internos (índice, referencias cruzadas)
    filecolor=blue, % Color para enlaces a archivos locales
    urlcolor=blue,      % Color para URLs
    citecolor=blue,     % Color para citas bibliográficas
}
%----------------------------------------------------------------------------------------

\usepackage{listings}
\usepackage{xcolor} % Para colores en listings
 \definecolor{codegreen}{rgb}{0,0.6,0}
 \definecolor{codegray}{rgb}{0.5,0.5,0.5}
 \definecolor{codepurple}{rgb}{0.58,0,0.82}
 \definecolor{backcolour}{rgb}{0.97,0.97,0.99}

\lstdefinestyle{MATLABStyle}{
  language=Matlab,
  basicstyle=\ttfamily\footnotesize,
  keywordstyle=\color{blue}\bfseries,
  commentstyle=\color{codegreen},
  stringstyle=\color{violet},
  numberstyle=\tiny\color{gray},
  breakatwhitespace=false,
  breaklines=true,
  captionpos=b,
  keepspaces=true,
  numbers=left,
  numbersep=5pt,
  showspaces=false,
  showstringspaces=false,
  showtabs=false,
  tabsize=2,
  frame=lines, % Añade un marco alrededor del código
  framerule=0.4pt, % Grosor del marco
  backgroundcolor=\color{backcolour} % Color de fondo suave
}
\lstset{style=MATLABStyle}
%----------------------------------------------------------------------------------------
%	TITLE PAGE
%----------------------------------------------------------------------------------------

\title{Fuzzy Logic Foundations}
\subtitle{Fuzzy Expert Systems}

\author{Prof. D.Sc. BARSEKH-ONJI Aboud}

\institute
{
    Facultad de Ingeniería \\
    Universidad Anáhuac México % Your institution for the title page
}
\date{\today} % Date, can be changed to a custom date

%----------------------------------------------------------------------------------------
%	PRESENTATION SLIDES
%----------------------------------------------------------------------------------------
% Poner esto en el preámbulo
\AtBeginSection[]
{
  \begin{frame}{Agenda}
    \tableofcontents[currentsection]
  \end{frame}
}
\begin{document}

\begin{frame}
    % Print the title page as the first slide
    \titlepage
\end{frame}
%----------------------------------------------------------------------------------------
%   SECTION: Mamdani vs. Sugeno Models
%----------------------------------------------------------------------------------------
\section{Intelligent Fuzzy Systems Process}
\begin{frame}
 \frametitle{Intelligent Fuzzy Systems Process}

    \begin{columns}
        \column[t]{0.5\textwidth}
         \begin{figure}
        \centering
        \includegraphics[width=0.7\textwidth]{Process.png} 
    \end{figure}  
        \column[t]{0.5\textwidth}
         \begin{block}{Note:}
        The FIS process is the same for both Mamdani and Sugeno models. The key difference lies in the \textbf{rule output} (consequent) and how the final output is computed.
    \end{block}  
    \end{columns}
   
   
   
\end{frame}
\section{Fuzzy Models: Mamdani vs. Sugeno}

\begin{frame}{The Mamdani Model: Intuitive \& Linguistic}
    \begin{block}{Key Concept: The output of each rule is a FUZZY SET.}
        This model is excellent at capturing expert knowledge in a human-like, linguistic way.
    \end{block}

            \textbf{Rule Structure:}
            \begin{alertblock}{IF-THEN}
                IF \textit{service} is \textbf{good} \\
                AND \textit{food} is \textbf{tasty} \\
                THEN \textit{tip} is \textbf{generous}.
            \end{alertblock}
            \textbf{Process:}
            \begin{enumerate}
                \item Evaluate the "IF" part (antecedent).
                \item "Clip" or "scale" the output fuzzy set (\textit{generous}).
                \item Aggregate all output fuzzy sets from all rules.
                \item \textbf{Defuzzify} the final aggregate shape to get a single crisp number (e.g., using Centroid).
            \end{enumerate}
            
\end{frame}

\begin{frame}{The Mamdani Model: Intuitive \& Linguistic}

\begin{figure}
                \centering
                \includegraphics[width=0.6\textwidth]{mamdani.png} 
            \end{figure}
            \begin{block}{Note:}
                The defuzzification step (e.g., Centroid) is computationally expensive, especially with many rules or complex shapes.
            \end{block}
\end{frame} 
%----------------------------------------------------------------------------------------

\begin{frame}{The Sugeno (TSK) Model: Computational \& Precise}
    \begin{block}{Key Concept: The output of each rule is a MATH FUNCTION (a constant or linear polynomial).}
        This model is computationally efficient and works very well for control systems and data-driven modeling (like ANFIS).
    \end{block}

            
            \begin{exampleblock}{IF-THEN \textbf{Rule Structure (Zero-Order):}}
                 IF \textit{service} is \textbf{good} \\
                 AND \textit{food} is \textbf{tasty} \\
                 THEN \textit{tip} = \textbf{20\%}.
            \end{exampleblock}
            
           
            \begin{exampleblock}{IF-THEN  \textbf{Rule Structure (First-Order):}}
                 IF \textit{service} is \textbf{good} \\
                 AND \textit{food} is \textbf{tasty} \\
                 THEN \textit{tip} = \textbf{c1*service + c2*food + k}.
            \end{exampleblock}
            
\end{frame}

\begin{frame}{The Sugeno (TSK) Model: Computational \& Precise}

\begin{figure}
                \centering
                \includegraphics[width=0.6\textwidth]{sugeno.png} % Replace with an image illustrating the Sugeno process
            \end{figure}
            \begin{block}{Note:}
                The Sugeno model is very efficient and fast, making it ideal for real-time applications.
            \end{block}
\end{frame} 

%----------------------------------------------------------------------------------------

\begin{frame}{Mamdani vs. Sugeno}
    \begin{table}
    \centering
    \footnotesize
    \begin{tabular}{p{0.45\textwidth} p{0.45\textwidth}}
        \toprule
        \textbf{\textcolor{blue}{Mamdani}} & \textbf{\textcolor{red}{Sugeno (TSK)}} \\
        \midrule
        \end{tabular}
        \begin{adjustbox}{width=\textwidth, valign=t}
        \begin{tabular}{p{0.45\textwidth} p{0.45\textwidth}}
        \textbf{Rule Output (Consequent)} & \\
        A Fuzzy Set. & A mathematical function (constant or linear). \\
        \textit{"tip is \textbf{generous}"} & \textit{"tip = \textbf{20}" or "tip = f(x,y)"} \\
        \midrule
        \textbf{Final Output Aggregation} & \\
        The aggregate of all output fuzzy sets. & A weighted average of each rule's crisp output. \\
        \midrule
        \textbf{Final Output Type} & \\
        Fuzzy. Requires \textbf{defuzzification} (e.g., Centroid) to get a crisp number. & Crisp. No defuzzification needed. \\
        \midrule
        \textbf{Best For} & \\
        Capturing human expert knowledge. Systems where the output needs to be linguistically interpretable. & Control systems, data-driven modeling, optimization problems (\texttt{ANFIS}, \texttt{tunefis}). \\
        \midrule
        \textbf{Computational Cost} & \\
        Higher. The defuzzification step is computationally expensive. & Lower. Very efficient and fast. \\
        \bottomrule
    \end{tabular}
    \end{adjustbox}
    \caption{Key differences between the two main FIS models.}
    \end{table}
\end{frame}
\end{document}

