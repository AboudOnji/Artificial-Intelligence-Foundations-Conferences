\documentclass[aspectratio=169,xcolor=dvipsnames]{beamer}
\usetheme{Berlin}

\usepackage[spanish]{babel} % Cambiado a español para acentos y textos automáticos
\usepackage{hyperref}
\usepackage{graphicx}
\usepackage{booktabs}
\usepackage{amsmath}
\usepackage{lettrine}
\setbeamertemplate{caption}[numbered]
\usepackage[dvipsnames,svgnames,x11names]{xcolor}
\usepackage{xurl}
\usepackage{hyperref}
\usepackage{algorithm}
\usepackage{algorithmicx}
\usepackage{algpseudocode}
\usepackage{adjustbox}
\hypersetup{
    colorlinks=true,
    linkcolor=cyan, % Color más visible en temas oscuros
    filecolor=blue,
    urlcolor=blue,
    citecolor=blue,
}
%----------------------------------------------------------------------------------------
\usepackage{listings}
\usepackage{xcolor}

\definecolor{codegreen}{rgb}{0,0.6,0}
\definecolor{codegray}{rgb}{0.5,0.5,0.5}
\definecolor{codepurple}{rgb}{0.58,0,0.82}
\definecolor{backcolour}{rgb}{0.97,0.97,0.99}

\lstdefinestyle{MATLABStyle}{
  language=Matlab,
  basicstyle=\ttfamily\footnotesize,
  keywordstyle=\color{blue}\bfseries,
  commentstyle=\color{codegreen},
  stringstyle=\color{violet},
  numberstyle=\tiny\color{gray},
  breakatwhitespace=false,
  breaklines=true,
  captionpos=b,
  keepspaces=true,
  numbers=left,
  numbersep=5pt,
  showspaces=false,
  showstringspaces=false,
  showtabs=false,
  tabsize=2,
  frame=lines,
  framerule=0.4pt,
  backgroundcolor=\color{backcolour}
}
\lstset{style=MATLABStyle}

\lstset{
language=MATLAB, % Cambiado a MATLAB
basicstyle=\small\ttfamily, % Fuente y tamaño del código
keywordstyle=\color{blue}\bfseries, % Estilo de las palabras clave
commentstyle=\color{codegreen}\itshape, % Estilo de los comentarios
stringstyle=\color{red}, % Estilo de las cadenas
showspaces=false, % No mostrar espacios como caracteres especiales
showtabs=false, % No mostrar tabulaciones como caracteres especiales
 breaklines=true, % Habilitar saltos de línea automáticos
numbers=left, % Mostrar números de línea a la izquierda
numberstyle=\tiny\color{gray}, % Estilo de los números de línea
frame=topbot, % Poner un marco alrededor del código (otras opciones: leftline, topbot, etc.)
framesep=5pt, % Espacio entre el marco y el código
rulesep=5pt, % Espacio entre los números de línea y el código
backgroundcolor=\color{blue!10}, % Color de fondo (requiere xcolor)
rulecolor=\color{black} % Color del marco
}

\title{Introducción a la Inteligencia Computacional}
\subtitle{Materia: Fundamentos de Inteligencia Artificial}


\author{Prof. D.Sc. BARSEKH-ONJI Aboud}

\institute
{
    Facultad de Ingeniería \\
    Universidad Anáhuac México % Your institution for the title page
}
\date{\today} % Date, can be changed to a custom date

%----------------------------------------------------------------------------------------
%	PRESENTATION SLIDES
%----------------------------------------------------------------------------------------
% Poner esto en el preámbulo
\AtBeginSection[]
{
  \begin{frame}{Agenda}
    \tableofcontents[currentsection]
  \end{frame}
}
\begin{document}

\begin{frame}
    % Print the title page as the first slide
    \titlepage
\end{frame}

%------------------------------------------------
\section{Inteligencia Artificial e Inteligencia Computacional}
%------------------------------------------------

\begin{frame}{Definiendo la Inteligencia}
    \begin{block}{Punto de Partida: La Inteligencia Humana}
    La inteligencia humana es un espectro de capacidades cognitivas: razonamiento, aprendizaje, resolución de problemas, percepción, lenguaje y creatividad. Trasladar esto a una máquina es un desafío complejo.
    \end{block}
    
    \begin{alertblock}{Definiciones de Inteligencia Artificial (IA)}
    Existen diferentes enfoques para definir la IA:
        \begin{itemize}
            \item \textbf{Enfoque Humano:} Sistemas que piensan o actúan como humanos.
            \item \textbf{Enfoque Racional:} Sistemas que piensan o actúan racionalmente. El enfoque del \textbf{agente racional}, que toma la mejor decisión posible dada la información disponible, ha ganado gran tracción en la investigación.
        \end{itemize}
    \end{alertblock}
\end{frame}

%------------------------------------------------

\begin{frame}{La Prueba de Turing (1950)}
    \begin{block}{Un Punto de Inflexión Histórico}
    Propuesta por Alan Turing, busca responder a la pregunta: ¿pueden pensar las máquinas?
    \begin{itemize}
        \item \textbf{El Juego:} Un evaluador humano se comunica por texto con dos entidades ocultas: una máquina y un humano.
        \item \textbf{El Veredicto:} Si el evaluador no puede distinguir de manera fiable cuál es la máquina, se dice que esta ha superado la prueba y exhibe un comportamiento inteligente.
    \end{itemize}
    \end{block}
    
    \begin{alertblock}{Críticas y Debate}
    Aunque es un referente cultural, la prueba ha sido muy debatida. Argumentos como la 'Habitación China' de John Searle cuestionan si realmente mide la \textbf{comprensión genuina} o solo una \textbf{simulación sofisticada} del lenguaje.
    \end{alertblock}
\end{frame}

%------------------------------------------------

\begin{frame}{Inteligencia Computacional (IC)}
    \begin{block}{Definición}
    La IC es 'el estudio de mecanismos \textbf{adaptativos} para permitir o facilitar el comportamiento inteligente en entornos complejos y cambiantes'. El énfasis está en la \textbf{adaptación} y el \textbf{aprendizaje}.
    \end{block}
    
    \begin{alertblock}{Habilidades Centrales de los Sistemas de IC}
    Estos sistemas demuestran una habilidad intrínseca para:
        \begin{itemize}
            \item Aprender de la experiencia.
            \item Adaptarse a nuevas situaciones.
            \item Generalizar el conocimiento a casos no vistos.
            \item Descubrir patrones ocultos en los datos.
        \end{itemize}
    \end{alertblock}
\end{frame}

%------------------------------------------------

\begin{frame}{IA vs. IC: Un Enfoque Distintivo}
    \begin{columns}[t]
        \column{.48\textwidth}
            \begin{block}{Inteligencia Artificial (General)}
                Es un campo amplio que también abarca enfoques simbólicos, lógicos y basados en el conocimiento explícito, como los sistemas expertos clásicos.
            \end{block}

        \column{.48\textwidth}
            \begin{alertblock}{Inteligencia Computacional (Específico)}
                 Es una rama de la IA con un enfoque distintivo en algoritmos a menudo \textbf{inspirados en la naturaleza}. Se centra en paradigmas como:
                 \begin{itemize}
                    \item Redes Neuronales Artificiales
                    \item Computación Evolutiva
                    \item Lógica Difusa
                    \item Inteligencia de Enjambre
                \end{itemize}
            \end{alertblock}
    \end{columns}
\end{frame}
%------------------------------------------------
\section{Ramas Principales de la Inteligencia Artificial}
%------------------------------------------------

\begin{frame}{Aprendizaje Automático de Máquina (Machine Learning)}
    \begin{block}{Definición}
    El Aprendizaje Automático (ML) es una rama de la IA cuyo objetivo es desarrollar modelos que permitan a los sistemas informáticos mejorar su rendimiento en una tarea a través de la experiencia, es decir, \textbf{aprendiendo de los datos} sin necesidad de ser programados explícitamente para cada detalle.
    \end{block}
    
    \begin{alertblock}{Concepto Clave: Generalización}
    Es la capacidad de un modelo de ML para funcionar bien y hacer predicciones precisas con \textbf{datos no vistos previamente}. Es el objetivo crucial del proceso de aprendizaje.
    \end{alertblock}
\end{frame}

%------------------------------------------------
\subsection{Aprendizaje Automático de Máquina}
%------------------------------------------------

\begin{frame}{Principales Categorías de Aprendizaje Automático}
    \begin{columns}[t]
        \column{.32\textwidth}
            \begin{block}{Aprendizaje Supervisado}
            El algoritmo aprende de datos \textbf{etiquetados}, donde cada entrada tiene una salida correcta conocida. El objetivo es predecir la salida para nuevas entradas.
            \end{block}
        \column{.32\textwidth}
            \begin{alertblock}{Aprendizaje No Supervisado}
            El algoritmo trabaja con datos \textbf{sin etiquetar}. El objetivo es descubrir patrones, estructuras o agrupaciones inherentes en los datos.
            \end{alertblock}
        \column{.32\textwidth}
            \begin{block}{Aprendizaje por Refuerzo}
            Un agente aprende a tomar decisiones interactuando con un \textbf{entorno}. El objetivo es aprender una estrategia que maximice una \textbf{recompensa} acumulada.
            \end{block}
    \end{columns}
    
   
\end{frame}
\begin{frame}{Principales Categorías de Aprendizaje Automático}
 \begin{figure}
        \centering
        \includegraphics[width=0.8\linewidth]{Figuras/Cap1/fig1.png}
        \caption{Paradigmas del Aprendizaje Automático.}
        \label{fig:1}
    \end{figure}
\end{frame}
%------------------------------------------------

\begin{frame}{Aprendizaje Supervisado: Aprender con un Maestro}
    \begin{block}{Analogía}
    Es como enseñarle a un niño la diferencia entre un perro y un gato: le muestras fotos etiquetadas ('esto es un perro', 'esto es un gato') y supervisas su aprendizaje. El modelo aprende a mapear una entrada (imagen) a una salida (etiqueta).
    \end{block}
    
    \begin{alertblock}{Tareas Típicas}
        \begin{itemize}
            \item \textbf{Clasificación:} Predecir una etiqueta categórica.
                \begin{itemize}
                    \item Ej: 'spam' o 'no spam', tipo de animal en una foto.
                \end{itemize}
            \item \textbf{Regresión:} Predecir un valor continuo.
                \begin{itemize}
                    \item Ej: precio de una casa, temperatura de mañana.
                \end{itemize}
        \end{itemize}
    \end{alertblock}
\end{frame}

%------------------------------------------------

\begin{frame}{Aprendizaje No Supervisado y por Refuerzo}
    \begin{columns}[t]
        \column{.48\textwidth}
            \begin{block}{Aprendizaje No Supervisado}
            \frametitle{Encontrar Estructura Oculta}
                El algoritmo explora datos sin etiquetar para encontrar patrones por sí mismo.
                \begin{itemize}
                    \item \textbf{Agrupamiento (Clustering):} Descubrir grupos naturales en los datos (e.g., segmentar clientes por comportamiento de compra).
                    \item \textbf{Reducción de Dimensionalidad:} Encontrar representaciones más compactas de los datos.
                \end{itemize}
            \end{block}

        \column{.48\textwidth}
            \begin{alertblock}{Aprendizaje por Refuerzo}
            \frametitle{Aprender por Prueba y Error}
                Un agente aprende interactuando con un entorno. No recibe 'respuestas correctas', sino señales de \textbf{recompensa} o \textbf{castigo} por sus acciones, buscando maximizar la recompensa a largo plazo.
                \begin{itemize}
                    \item \textbf{Ejemplo Clásico:} Un programa que aprende a jugar ajedrez o Go.
                \end{itemize}
            \end{alertblock}
    \end{columns}
\end{frame}

%------------------------------------------------
\subsection{Aprendizaje Profundo}
%------------------------------------------------

\begin{frame}{Aprendizaje Profundo (Deep Learning)}
    \begin{block}{Definición}
    El Aprendizaje Profundo (DL) es una rama del Aprendizaje Automático caracterizada por el uso de \textbf{Redes Neuronales Artificiales (ANN)} con múltiples capas de procesamiento, conocidas como arquitecturas 'profundas'.
    \end{block}
    
    \begin{alertblock}{La Fuerza de la Profundidad: Aprendizaje Jerárquico}
    La 'profundidad' se refiere al número de capas a través de las cuales se transforman los datos. Esta estructura jerárquica permite que el modelo aprenda características complejas de forma incremental: las primeras capas aprenden rasgos simples (como bordes en una imagen) y las capas posteriores los combinan para formar conceptos más abstractos (como rostros).
    \end{alertblock}
\end{frame}

%------------------------------------------------

\begin{frame}{Aplicaciones y Desafíos del Aprendizaje Profundo}

            \begin{block}{Avances Espectaculares}
            El DL ha sido el motor de revoluciones en dominios como:
                \begin{itemize}
                    \item \textbf{Visión por Computadora:} Clasificación de imágenes y detección de objetos con Redes Neuronales Convolucionales (CNNs).
                    \item \textbf{Procesamiento del Lenguaje Natural (NLP):} Traducción y generación de texto con \textit{Transformers}.
                    \item \textbf{Reconocimiento de Voz.}
                    \item \textbf{Juegos y Control:} Agentes con nivel sobrehumano en juegos como Go (AlphaGo).
                \end{itemize}
            \end{block}


           

\end{frame}
\begin{frame}{Aplicaciones y Desafíos del Aprendizaje Profundo}

 \begin{alertblock}{El Gran Desafío: La 'Caja Negra'}
                A pesar de su efectividad, los modelos de DL a menudo son 'cajas negras' imposibles de interpretar. No podemos saber con certeza por qué toman una decisión específica.
                
                \vspace{1em}
                Esta falta de interpretabilidad es un limitante en áreas críticas y ha dado origen a un nuevo campo de investigación: la \textbf{Inteligencia Artificial Explicable (XAI)}.
            \end{alertblock}
\end{frame}
%------------------------------------------------
\subsection{Modelos de Lenguaje y Procesamiento del Lenguaje Natural (NLP)}
%------------------------------------------------

\begin{frame}{NLP y Modelos de Lenguaje}
    \begin{columns}[t]
        \column{.48\textwidth}
            \begin{block}{Procesamiento del Lenguaje Natural (NLP) ��️}
                Es una rama de la IA que se ocupa de la interacción entre las computadoras y el lenguaje humano.
                
                \vspace{1em}
                \textbf{Objetivo:} Permitir que las máquinas procesen, comprendan, generen y respondan al lenguaje humano de manera útil y significativa.
            \end{block}

        \column{.48\textwidth}
            \begin{alertblock}{Modelos de Lenguaje (LMs)}
                 Son el componente central del NLP. Un LM es un modelo probabilístico que, dado un contexto, predice cuál es la siguiente palabra más probable en una secuencia.
                 
                 \vspace{1em}
                 Pensemos en la función de autocompletar de nuestros teclados o editores de código.
            \end{alertblock}
    \end{columns}
\end{frame}

%------------------------------------------------

\begin{frame}{La Revolución del Aprendizaje Profundo en NLP}
    \begin{block}{Evolución de los Modelos de Lenguaje}
        \begin{itemize}
            \item \textbf{LMs Tradicionales (n-gramas):} Capturan dependencias locales, pero fallan con el contexto a largo plazo. \pause
            \item \textbf{Redes Neuronales Recurrentes (RNNs):} Primer gran avance para modelar secuencias y dependencias a más largo plazo. \pause
            \item \textbf{Arquitectura Transformer:} Una revolución gracias a su mecanismo de 'auto-atención', que le permite ponderar la importancia de todas las palabras en una secuencia a la vez.
        \end{itemize}
    \end{block}
    
    \begin{alertblock}{Grandes Modelos de Lenguaje (LLMs)}
    Son una manifestación de los LMs basados en Transformers, como \textbf{GPT} o \textbf{Gemini}. Se caracterizan por tener miles de millones de parámetros y ser entrenados en cantidades masivas de texto, lo que les permite aprender una comprensión general del lenguaje, el conocimiento del mundo y patrones de razonamiento.
    \end{alertblock}
\end{frame}

%------------------------------------------------

\begin{frame}{Usar un LLM vs. 'Hacer' Inteligencia Artificial}
    \begin{block}{Uso de LLMs}
    Estos modelos pueden ser ajustados para tareas específicas (\textit{fine-tuning}) o utilizados directamente mediante \textbf{ingeniería de prompts} (\textit{prompt engineering}), demostrando capacidades asombrosas en generación de texto, traducción, resumen y hasta programación.
    \end{block}
    
    \begin{alertblock}{Una Distinción Importante}
        \begin{itemize}
            \item \textbf{Usar un LLM:} Es ser un usuario de un producto final desarrollado con IA.
            \item \textbf{'Hacer' IA / Inteligencia Computacional:} Implica aplicar directamente un algoritmo (ML o DL) sobre una base de datos para crear un nuevo modelo inteligente. Significa tener contacto directo tanto con los datos como con los algoritmos.
        \end{itemize}
    \end{alertblock}
\end{frame}

%------------------------------------------------
\section{Tipos de Inteligencia Computacional}
%------------------------------------------------

\begin{frame}{Paradigmas de la Inteligencia Computacional (IC)}
    \begin{block}{Enfoque Principal}
    La Inteligencia Computacional se distingue por su enfoque en paradigmas que simulan procesos o estrategias que existen en la \textbf{naturaleza} con la finalidad de lograr \textbf{adaptación y aprendizaje}.
    \end{block}
    
    \begin{alertblock}{Características Comunes de los Algoritmos de IC}
    Todos los algoritmos que pertenecen a la IC son:
        \begin{itemize}
            \item \textbf{Iterativos:} Mejoran su solución a lo largo de múltiples pasos o 'generaciones'.
            \item \textbf{Optimizadores:} Buscan resolver un problema de optimización, como minimizar un error de predicción.
        \end{itemize}
    \end{alertblock}
\end{frame}

%------------------------------------------------
\subsection{Redes Neuronales Artificiales}
%------------------------------------------------

\begin{frame}{Redes Neuronales Artificiales (ANN)}
    \frametitle{Inspiración: El Cerebro Humano}
    \begin{block}{Concepto}
    Las Redes Neuronales Artificiales (ANNs) son uno de los pilares de la IC. Tratan de simular la red neuronal del cerebro humano, representando redes de neuronas artificiales interconectadas y organizadas en capas.
    \end{block}
    
    \begin{columns}[c]
        \column{.48\textwidth}
            \begin{alertblock}{Funcionamiento Básico de una Neurona}
                 \begin{enumerate}
                    \item Recibe señales de entrada.
                    \item Calcula una suma ponderada de estas entradas.
                    \item Aplica una \textbf{función de activación} no lineal para producir una señal de salida.
                \end{enumerate}
            \end{alertblock}
        \column{.48\textwidth}
             \begin{block}{Proceso de Aprendizaje}
                El 'aprendizaje' consiste en un algoritmo iterativo que \textbf{ajusta los pesos} de las conexiones para minimizar el error entre la salida producida por la red y la salida deseada.
            \end{block}
    \end{columns}
\end{frame}

%------------------------------------------------
\subsection{Computación Evolutiva}
%------------------------------------------------

\begin{frame}{Computación Evolutiva (CE)}
    \frametitle{Inspiración: La Evolución Biológica}
    \begin{block}{Concepto}
    La Computación Evolutiva es una familia de algoritmos de optimización que se inspiran directamente en la teoría de la evolución de Charles Darwin: \textbf{selección natural, herencia genética y mutación}.
    \end{block}
    
    \begin{alertblock}{Funcionamiento General}
    Estos algoritmos operan sobre una \textbf{'población'} de posibles soluciones y buscan mejorar su calidad de forma iterativa a lo largo de múltiples \textbf{'generaciones'}.
    \end{alertblock}
\end{frame}

%------------------------------------------------

\begin{frame}{El Ciclo de un Algoritmo Evolutivo}
    \begin{enumerate}
        \item \textbf{Población Inicial:} Se genera un conjunto inicial de posibles soluciones ('cromosomas'), a menudo de forma aleatoria. \pause
        
        \item \textbf{Evaluación de Aptitud (\textit{Fitness}):} Cada solución es evaluada con una 'función de aptitud' que cuantifica qué tan buena es. \pause
        
        \item \textbf{Selección:} Las mejores soluciones ('los más aptos') son seleccionadas para ser 'padres' de la siguiente generación. \pause
        
        \item \textbf{Operadores Genéticos:} Se crean nuevas soluciones ('descendencia') combinando a los padres (\textbf{cruce}) e introduciendo pequeñas variaciones (\textbf{mutación}). \pause
        
        \item \textbf{Reemplazo:} La nueva generación de soluciones reemplaza a la anterior. \pause
        
        \item El ciclo se repite hasta que se cumple un criterio de terminación (e.g., número de generaciones, calidad de la solución).
    \end{enumerate}
\end{frame}

%------------------------------------------------

\begin{frame}{Principales Paradigmas de la Computación Evolutiva}
    \begin{itemize}
        \item \textbf{Algoritmos Genéticos (GA):} Típicamente utilizan representaciones binarias de las soluciones y enfatizan el operador de cruce. \pause
        
        \item \textbf{Programación Genética (GP):} Evolucionan directamente programas de computadora o expresiones matemáticas, a menudo representados como árboles. \pause
        
        \item \textbf{Estrategias Evolutivas (ES):} Se centran en la optimización de parámetros de valor real y a menudo auto-adaptan sus propias estrategias de mutación. \pause
        
        \item \textbf{Programación Evolutiva (EP):} Pone un mayor énfasis en la mutación como principal motor de la evolución, modelando la evolución a nivel de especie.
    \end{itemize}
\end{frame}

%------------------------------------------------
\subsection{Lógica Difusa y Sistemas Difusos}
%------------------------------------------------

\begin{frame}{Lógica Difusa: Más Allá del Verdadero y Falso}
    \begin{columns}[t]
        \column{.48\textwidth}
            \begin{block}{Lógica Clásica (Booleana)}
                \begin{itemize}
                    \item Opera con proposiciones que son estrictamente \textbf{verdaderas} o \textbf{falsas}.
                    \item Utiliza valores binarios (0 o 1).
                \end{itemize}
            \end{block}

        \column{.48\textwidth}
            \begin{alertblock}{Lógica Difusa (Fuzzy Logic)}
                 \begin{itemize}
                    \item Permite hablar de \textbf{'grados de verdad'}.
                    \item Una proposición puede ser parcialmente verdadera.
                    \item Trata con la imprecisión y vaguedad del razonamiento humano.
                \end{itemize}
            \end{alertblock}
    \end{columns}
    
    \begin{block}{Concepto Central: El Conjunto Difuso}
    A diferencia de un conjunto clásico, en un \textbf{conjunto difuso} los elementos tienen un \textbf{grado de pertenencia} que varía en el intervalo continuo [0, 1]. Por ejemplo, una temperatura de 18°C puede pertenecer un 70\% al conjunto 'templado' y un 30\% al conjunto 'frío'.
    \end{block}
\end{frame}

%------------------------------------------------

\begin{frame}{Sistemas de Inferencia Difusa}
    \begin{block}{¿Cómo funciona un Sistema Difuso?}
    Utilizan reglas lingüísticas para modelar sistemas y tomar decisiones. Constan de cuatro componentes:
    \end{block}
    
    \begin{enumerate}
        \item \textbf{Fuzificación:} Convierte las entradas numéricas ('nítidas') en grados de pertenencia a conjuntos difusos. \pause
        \item \textbf{Base de Reglas Difusas:} Un conjunto de reglas del tipo \textbf{'SI-ENTONCES'}. Por ejemplo: 'SI la temperatura ES alta ENTONCES la velocidad del ventilador ES rápida'. \pause
        \item \textbf{Motor de Inferencia:} Combina las reglas para derivar las conclusiones difusas. \pause
        \item \textbf{Defuzificación:} Convierte la conclusión difusa de nuevo en una salida numérica y accionable.
    \end{enumerate}
    
    \begin{alertblock}{Aplicaciones}
    Son especialmente valorados en el diseño de \textbf{sistemas de control}, donde el conocimiento de un experto puede ser fácilmente traducido a reglas lingüísticas.
    \end{alertblock}
\end{frame}

%------------------------------------------------
\subsection{Inteligencia de Enjambre}
%------------------------------------------------

\begin{frame}{Inteligencia de Enjambre (Swarm Intelligence)}
    \frametitle{Inspiración: Comportamiento Colectivo}
    \begin{block}{Concepto}
    Es un paradigma de la IC que se inspira en el comportamiento colectivo de sistemas naturales como colonias de hormigas, bandadas de pájaros o bancos de peces.
    \end{block}
    
    \begin{alertblock}{Funcionamiento}
    Estos sistemas están compuestos por un gran número de agentes individuales relativamente simples que siguen reglas básicas. De sus interacciones locales emerge un comportamiento global inteligente y autoorganizado, sin necesidad de un control centralizado.
    \end{alertblock}
\end{frame}

%------------------------------------------------

\begin{frame}{Características Fundamentales de los Enjambres}
    \begin{itemize}
        \item \textbf{Control descentralizado:} No hay una entidad única que dirija el comportamiento del enjambre. \pause
        \item \textbf{Autoorganización:} Los patrones y estructuras globales surgen espontáneamente de las interacciones locales entre agentes. \pause
        \item \textbf{Robustez:} El sistema puede tolerar fallos de algunos individuos sin que el rendimiento global se vea afectado significativamente. \pause
        \item \textbf{Escalabilidad:} El comportamiento del sistema puede mantenerse efectivo incluso si aumenta el número de agentes. \pause
        \item \textbf{Comunicación indirecta:} Los agentes a menudo interactúan modificando su entorno (e.g., el rastro de feromonas de las hormigas), y estas modificaciones sirven como señales para otros.
    \end{itemize}
\end{frame}

%------------------------------------------------

\begin{frame}{Algoritmos Prominentes de Inteligencia de Enjambre}
    \begin{columns}[t]
        \column{.48\textwidth}
            \begin{block}{Optimización por Enjambre de Partículas (PSO)}
                \begin{itemize}
                    \item \textbf{Inspiración:} El movimiento coordinado de las bandadas de pájaros o bancos de peces buscando alimento.
                    \item \textbf{Mecanismo:} Una población de 'partículas' (soluciones) 'vuela' a través del espacio de búsqueda, ajustando su trayectoria en base a su propia mejor experiencia y la mejor experiencia de todo el enjambre.
                \end{itemize}
            \end{block}

        \column{.48\textwidth}
            \begin{alertblock}{Optimización por Colonia de Hormigas (ACO)}
                 \begin{itemize}
                    \item \textbf{Inspiración:} El comportamiento de las hormigas para encontrar los caminos más cortos hacia una fuente de alimento.
                    \item \textbf{Mecanismo:} 'Hormigas' artificiales construyen soluciones y depositan 'feromona' en los caminos que eligen. Los caminos con más feromona son más atractivos, guiando al enjambre hacia buenas soluciones.
                \end{itemize}
            \end{alertblock}
    \end{columns}
\end{frame}

%------------------------------------------------
\subsection{Sistemas Inmunes Artificiales}
%------------------------------------------------

\begin{frame}{Sistemas Inmunes Artificiales (AIS)}
    \frametitle{Inspiración: El Sistema Inmunológico}
    \begin{block}{Concepto}
    Los Sistemas Inmunes Artificiales (AIS) son una clase de sistemas computacionales inspirados en los principios y mecanismos del sistema inmunológico de los vertebrados.
    \end{block}
    
    \begin{alertblock}{Capacidades del Sistema Inmunológico a Emular}
    El sistema inmunológico natural es un sistema complejo, distribuido y altamente adaptativo con notables capacidades de:
        \begin{itemize}
            \item Aprendizaje y memoria.
            \item Reconocimiento de patrones (distinguiendo entre 'lo propio' y 'lo no propio').
            \item Autoorganización y robustez.
        \end{itemize}
    \end{alertblock}
\end{frame}

%------------------------------------------------

\begin{frame}{Conceptos Inmunológicos Clave y Aplicaciones}
  
            \begin{block}{Conceptos que Inspiran Algoritmos}
                \begin{itemize}
                    \item \textbf{Selección Clonal:} Clonar y mutar las mejores soluciones (para optimización).
                    \item \textbf{Selección Negativa:} Generar detectores que reconozcan patrones anómalos o 'no propios' (para detección de anomalías).
                    \item \textbf{Redes Inmunes:} Modelar el sistema como una red auto-regulada (para clustering).
                    \item \textbf{Teoría del Peligro:} Reaccionar a señales de 'peligro' en lugar de solo a lo 'no propio' (para detección de intrusiones).
                \end{itemize}
            \end{block}

           
\end{frame}
\begin{frame}{Conceptos Inmunológicos Clave y Aplicaciones}

 \begin{alertblock}{Aplicaciones}
                 \begin{itemize}
                    \item Detección de intrusiones en redes informáticas.
                    \item Detección de virus y spam.
                    \item Clasificación de datos.
                    \item Optimización de funciones.
                    \item Robótica.
                \end{itemize}
            \end{alertblock}
\end{frame}

\begin{frame}[allowframebreaks]{Referencias}
    \nocite{*} % Este comando incluye todas las referencias
    
    \bibliographystyle{apalike} % Define el estilo visual de la bibliografía
    \bibliography{references} % Apunta a tu archivo references.bib
\end{frame}
\end{document}