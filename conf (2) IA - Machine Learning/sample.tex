\documentclass[aspectratio=169,xcolor=dvipsnames]{beamer}
\usetheme{Berlin}

\usepackage[spanish]{babel} % Cambiado a español para acentos y textos automáticos
\usepackage{hyperref}
\usepackage{graphicx}
\usepackage{booktabs}
\usepackage{amsmath}
\usepackage{lettrine}
\setbeamertemplate{caption}[numbered]
\usepackage[dvipsnames,svgnames,x11names]{xcolor}
\usepackage{xurl}
\usepackage{hyperref}
\usepackage{algorithm}
\usepackage{algorithmicx}
\usepackage{algpseudocode}
\usepackage{adjustbox}
\hypersetup{
    colorlinks=true,
    linkcolor=cyan, % Color más visible en temas oscuros
    filecolor=blue,
    urlcolor=blue,
    citecolor=blue,
}
%----------------------------------------------------------------------------------------
\usepackage{listings}
\usepackage{xcolor}

\definecolor{codegreen}{rgb}{0,0.6,0}
\definecolor{codegray}{rgb}{0.5,0.5,0.5}
\definecolor{codepurple}{rgb}{0.58,0,0.82}
\definecolor{backcolour}{rgb}{0.97,0.97,0.99}

\lstdefinestyle{MATLABStyle}{
  language=Matlab,
  basicstyle=\ttfamily\footnotesize,
  keywordstyle=\color{blue}\bfseries,
  commentstyle=\color{codegreen},
  stringstyle=\color{violet},
  numberstyle=\tiny\color{gray},
  breakatwhitespace=false,
  breaklines=true,
  captionpos=b,
  keepspaces=true,
  numbers=left,
  numbersep=5pt,
  showspaces=false,
  showstringspaces=false,
  showtabs=false,
  tabsize=2,
  frame=lines,
  framerule=0.4pt,
  backgroundcolor=\color{backcolour}
}
\lstset{style=MATLABStyle}

\title{Principios de Machine Learning}
\subtitle{Materia: Fundamentos de Inteligencia Artificial}


\author{Prof. D.Sc. BARSEKH-ONJI Aboud}

\institute
{
    Facultad de Ingeniería \\
    Universidad Anáhuac México % Your institution for the title page
}
\date{\today} % Date, can be changed to a custom date
\lstset{
language=MATLAB, % Cambiado a MATLAB
basicstyle=\small\ttfamily, % Fuente y tamaño del código
keywordstyle=\color{blue}\bfseries, % Estilo de las palabras clave
commentstyle=\color{codegreen}\itshape, % Estilo de los comentarios
stringstyle=\color{red}, % Estilo de las cadenas
showspaces=false, % No mostrar espacios como caracteres especiales
showtabs=false, % No mostrar tabulaciones como caracteres especiales
 breaklines=true, % Habilitar saltos de línea automáticos
numbers=left, % Mostrar números de línea a la izquierda
numberstyle=\tiny\color{gray}, % Estilo de los números de línea
frame=topbot, % Poner un marco alrededor del código (otras opciones: leftline, topbot, etc.)
framesep=5pt, % Espacio entre el marco y el código
rulesep=5pt, % Espacio entre los números de línea y el código
backgroundcolor=\color{blue!10}, % Color de fondo (requiere xcolor)
rulecolor=\color{black} % Color del marco
}

%----------------------------------------------------------------------------------------
%	PRESENTATION SLIDES
%----------------------------------------------------------------------------------------
% Poner esto en el preámbulo
% Poner esto en el preámbulo
\AtBeginSection[]
{
  \begin{frame}{Agenda}
    \tableofcontents[currentsection]
  \end{frame}
}
\begin{document}

\begin{frame}
    % Print the title page as the first slide
    \titlepage
\end{frame}

%------------------------------------------------
\section{Introducción al Aprendizaje Automático}
%------------------------------------------------

\begin{frame}{¿Qué es el Aprendizaje Automático (Machine Learning)?}
    \begin{block}{Definición}
    Es una rama de la Inteligencia Artificial que se enfoca en el desarrollo de sistemas capaces de \textbf{aprender y mejorar a partir de la experiencia}, sin ser programados explícitamente para cada tarea.
    \end{block}
    
    \begin{alertblock}{El Proceso de Aprendizaje}
    En lugar de seguir instrucciones fijas, los algoritmos de ML construyen un \textbf{modelo matemático} basado en \textbf{datos de entrenamiento} para realizar predicciones o tomar decisiones.
    \end{alertblock}
\end{frame}

%------------------------------------------------

\begin{frame}{El Objetivo Clave: Generalización}
    \begin{block}{Generalización}
    Se dice que un modelo \textbf{generaliza} bien si, después de aprender de un conjunto de datos, es capaz de realizar \textbf{predicciones correctas sobre datos nuevos y no vistos}.
    \end{block}
    
    \begin{alertblock}{La Ciencia del Aprendizaje Automático}
    El objetivo fundamental no es crear un modelo que memorice los datos de entrenamiento, sino aprender un modelo que \textbf{generalice bien}.
    \end{alertblock}
\end{frame}

%------------------------------------------------

\begin{frame}{Principales Categorías del Aprendizaje Automático}
    \begin{itemize}
        \item<1-> \textbf{Aprendizaje Supervisado:}
            \begin{itemize}
                \item El modelo aprende de datos \textbf{etiquetados} (cada entrada tiene una salida correcta conocida).
                \item \textbf{Objetivo:} Predecir una salida.
            \end{itemize}
        \pause
        \item<2-> \textbf{Aprendizaje No Supervisado:}
            \begin{itemize}
                \item El modelo aprende de datos \textbf{sin etiquetar}.
                \item \textbf{Objetivo:} Encontrar patrones o estructuras ocultas en los datos.
            \end{itemize}
        \pause
        \item<3-> \textbf{Aprendizaje por Refuerzo:}
            \begin{itemize}
                \item Un \textbf{agente} aprende interactuando con un \textbf{entorno}.
                \item \textbf{Objetivo:} Aprender una estrategia para maximizar una \textbf{recompensa} a largo plazo.
            \end{itemize}
    \end{itemize}
\end{frame}


\begin{frame}{Principales Categorías del Aprendizaje Automático}
\begin{figure}
    \centering
    \includegraphics[width=0.75\linewidth]{figg1.png}
\end{figure}
\end{frame}
%------------------------------------------------
\section{Aprendizaje Supervisado}
%------------------------------------------------

\begin{frame}{Aprendizaje Supervisado: Aprender de Ejemplos Etiquetados}
    \begin{block}{Definición}
    En el aprendizaje supervisado, el algoritmo aprende a partir de un conjunto de datos de entrenamiento que consiste en pares de entrada-salida. Cada ejemplo de entrenamiento $(\mathbf{x}_i, y_i)$ incluye un vector de características de entrada y una \textbf{etiqueta de salida o valor objetivo conocido}.
    \end{block}
    
    \begin{alertblock}{Objetivo}
    Aprender una función de mapeo $f: X \to Y$ tal que $f(\mathbf{x}) \approx y$. Una vez entrenado, el modelo puede \textbf{predecir la salida} para nuevas entradas para las cuales la salida es desconocida.
    \end{alertblock}
\end{frame}

%------------------------------------------------

\begin{frame}{Tareas del Aprendizaje Supervisado: Regresión vs. Clasificación}
    \begin{columns}[t]
        \column{.48\textwidth}
            \begin{block}{Regresión}
                \begin{itemize}
                    \item La variable de salida $y$ es un valor \textbf{continuo} o \textbf{cuantitativo}.
                    \item El objetivo es predecir un valor numérico.
                    \item \textbf{Ejemplos:}
                        \begin{itemize}
                            \item Predecir el precio de una casa.
                            \item Estimar la temperatura de mañana.
                        \end{itemize}
                \end{itemize}
            \end{block}

        \column{.48\textwidth}
            \begin{alertblock}{Clasificación}
                 \begin{itemize}
                    \item La variable de salida $y$ es una etiqueta \textbf{categórica} o \textbf{cualitativa}.
                    \item El objetivo es asignar una etiqueta de clase.
                    \item \textbf{Ejemplos:}
                        \begin{itemize}
                            \item Clasificar un correo como 'spam' o 'no spam'.
                            \item Identificar si una imagen contiene un gato, un perro o un pájaro.
                        \end{itemize}
                \end{itemize}
            \end{alertblock}
    \end{columns}
\end{frame}

%------------------------------------------------
\section{Aprendizaje No Supervisado}
%------------------------------------------------

\begin{frame}{Aprendizaje No Supervisado}
    \frametitle{Encontrando Patrones en Datos sin Etiquetar}
    \begin{block}{Definición}
    A diferencia del aprendizaje supervisado, en el aprendizaje no supervisado el algoritmo recibe datos de entrada $\mathbf{x}_i$ \textbf{sin ninguna etiqueta de salida} explícita $y_i$.
    \end{block}
    
    \begin{alertblock}{Objetivo}
    Descubrir patrones, estructura o conocimiento inherente directamente de los datos no etiquetados. El modelo debe encontrar por sí mismo las relaciones en los datos.
    \end{alertblock}
\end{frame}

%------------------------------------------------

\begin{frame}{Tareas Comunes del Aprendizaje No Supervisado}
    \begin{itemize}
        \item<1-> \textbf{Agrupamiento (Clustering):}
            \begin{itemize}
                \item Consiste en agrupar los datos en clústeres, de modo que los puntos dentro de un mismo clúster sean muy similares entre sí.
                \item \textit{Ejemplo:} Segmentar clientes en grupos con comportamientos de compra parecidos.
            \end{itemize}
        \pause
        \item<2-> \textbf{Reducción de Dimensionalidad:}
            \begin{itemize}
                \item Busca reducir el número de variables (características) de un conjunto de datos, conservando la mayor cantidad de información relevante posible.
                \item \textit{Ejemplo:} Simplificar un modelo o facilitar la visualización de datos complejos.
            \end{itemize}
        \pause
        \item<3-> \textbf{Aprendizaje de Reglas de Asociación:}
             \begin{itemize}
                \item Descubre relaciones interesantes entre variables en grandes conjuntos de datos.
                \item \textit{Ejemplo:} "Los clientes que compran pañales también tienden a comprar cerveza".
            \end{itemize}
    \end{itemize}
\end{frame}

%------------------------------------------------
\section{Aprendizaje por Refuerzo}
%------------------------------------------------

\begin{frame}{Aprendizaje por Refuerzo (Reinforcement Learning)}
    \frametitle{Aprender a través de la Interacción ��}
    \begin{block}{Definición}
    Es un paradigma donde un \textbf{agente} aprende a tomar una secuencia de \textbf{acciones} en un \textbf{entorno} para maximizar una noción de \textbf{recompensa} acumulada a lo largo del tiempo.
    \end{block}
    
    \begin{alertblock}{Aprendizaje por Prueba y Error}
    El agente no recibe ejemplos de "entrada-salida correcta". En su lugar, aprende a través de la interacción y la retroalimentación (recompensas o castigos) de sus acciones para desarrollar una estrategia ganadora.
    \end{alertblock}
\end{frame}

%------------------------------------------------

\begin{frame}{Componentes Clave del Aprendizaje por Refuerzo}
    \begin{itemize}
        \item \textbf{Agente:} La entidad que aprende y toma decisiones. \pause
        \item \textbf{Entorno:} El mundo (real o simulado) con el que interactúa el agente. \pause
        \item \textbf{Estado ($s$):} Una descripción de la situación actual del entorno. \pause
        \item \textbf{Acción ($a$):} Una elección que el agente puede tomar en un estado dado. \pause
        \item \textbf{Recompensa ($r$):} Una señal numérica que el entorno proporciona al agente después de cada acción. \pause
        \item \textbf{Política ($\pi$):} La estrategia que utiliza el agente para seleccionar acciones. \textbf{El objetivo del RL es aprender una política óptima.}
    \end{itemize}
\end{frame}

%------------------------------------------------

\begin{frame}{El Dilema Central: Exploración vs. Explotación}
    \begin{columns}[t]
        \column{.48\textwidth}
            \begin{block}{Exploración}
                \begin{itemize}
                    \item Consiste en probar nuevas acciones para descubrir qué tan efectivas son.
                    \item Es necesario para encontrar mejores estrategias y no quedarse atascado en una solución subóptima.
                    \item Implica un riesgo a corto plazo con la esperanza de una mayor recompensa a largo plazo.
                \end{itemize}
            \end{block}

        \column{.48\textwidth}
            \begin{alertblock}{Explotación}
                 \begin{itemize}
                    \item Consiste en utilizar el conocimiento actual para tomar las acciones que se sabe que son las mejores.
                    \item Maximiza la recompensa a corto plazo basándose en la experiencia pasada.
                    \item Si se explota demasiado pronto, el agente podría nunca descubrir acciones mucho mejores.
                \end{itemize}
            \end{alertblock}
    \end{columns}
    
    
\end{frame}

\begin{frame}{El Dilema Central: Exploración vs. Explotación}

\begin{block}{El Equilibrio}
    Un agente de RL exitoso debe encontrar un equilibrio inteligente entre explorar su entorno para adquirir nuevo conocimiento y explotar su conocimiento actual para obtener recompensas.
    \end{block}
    \end{frame}

%------------------------------------------------
\section{Tratamiento de Datos}
%------------------------------------------------

\begin{frame}{Tratamiento de Datos: La Base del Éxito}
    \begin{block}{División de los Datos}
    La calidad y la gestión adecuada de los datos son fundamentales para el éxito. Un paso crítico es dividir el conjunto de datos total en tres subconjuntos independientes:
    \end{block}
    
    \begin{alertblock}{Entrenamiento, Validación y Prueba}
    \begin{center}
    \begin{tabular}{|c|c|c|}
    \hline
    \textbf{Datos de Entrenamiento} & \textbf{Datos de Validación} & \textbf{Datos de Prueba} \\
    (p.ej., 60\%-70\%) & (p.ej., 15\%-20\%) & (p.ej., 15\%-20\%) \\ \hline
    \end{tabular}
    \end{center}
    \end{alertblock}
\end{frame}

%------------------------------------------------
\subsection{Datos de Entrenamiento, Validación y Prueba}
%------------------------------------------------

\begin{frame}{Definición de los Conjuntos de Datos}
    \begin{columns}[t]
        \column{.48\textwidth}
            \begin{block}{Datos de Entrenamiento (Training)}
                \begin{itemize}
                    \item Es la porción de datos que se utiliza para "enseñar" o ajustar los parámetros del modelo.
                    \item El modelo aprende los patrones y relaciones subyacentes presentes en estos datos.
                \end{itemize}
            \end{block}

        \column{.48\textwidth}
            \begin{alertblock}{Datos de Validación (Validation)}
                 \begin{itemize}
                    \item Se utiliza para el \textbf{ajuste de hiperparámetros} (e.g., el número de vecinos $k$ en k-NN).
                    \item Sirve para la \textbf{selección de modelos} (comparar un árbol de decisión vs. una SVM).
                    \item Ayuda a detectar el \textbf{sobreajuste} (overfitting).
                \end{itemize}
            \end{alertblock}
    \end{columns}
    
    \begin{examples}{Datos de Prueba (Testing)}
    Se mantiene completamente separado y se utiliza \textbf{una sola vez} al final del proceso para proporcionar una evaluación final e \textbf{insesgada} del rendimiento del modelo en datos del mundo real.
    \end{examples}
\end{frame}

%------------------------------------------------
\subsection{Validación Cruzada}
%------------------------------------------------

\begin{frame}{Técnica Robusta: Validación Cruzada (k-Fold CV)}
    \begin{block}{¿Qué es la Validación Cruzada?}
    Es una técnica para obtener una estimación más estable del rendimiento del modelo. En lugar de una única división, los datos se dividen en $k$ "pliegues" (folds), y el modelo se entrena y valida $k$ veces.
    \end{block}
\end{frame}
\begin{frame}{Técnica Robusta: Validación Cruzada (k-Fold CV)}

    \begin{alertblock}{Proceso de k-Fold Cross-Validation}
        \begin{itemize}
            \item Dividir los datos de entrenamiento en $k$ subconjuntos (pliegues).
            \item En cada una de las $k$ iteraciones:
                \begin{itemize}
                    \item Se utiliza un pliegue diferente para la \textbf{validación}.
                    \item Se utilizan los $k-1$ pliegues restantes para el \textbf{entrenamiento}.
                \end{itemize}
            \item Se promedian los $k$ resultados de validación para obtener la estimación final del rendimiento.
        \end{itemize}
    \end{alertblock}
\end{frame}

%------------------------------------------------
\begin{frame}[fragile]{Ejemplo de Código: División de Datos}
    \begin{columns}[t]
        \column{.48\textwidth}
            \begin{block}{Python (Scikit-learn)}
                \begin{lstlisting}
from sklearn.model_selection import train_test_split

# Dividir en 80% entrenamiento y 20% prueba
X_train, X_test, y_train, y_test = train_test_split(
    X, y, test_size=0.20, random_state=42
)
                \end{lstlisting}
            \end{block}

        \column{.48\textwidth}
            \begin{alertblock}{MATLAB}
                 \begin{lstlisting}
% Crear una particion para 80% entrenamiento y 20% prueba
cvp = cvpartition(Y, 'Holdout', 0.20);

X_train = X(cvp.training,:);
Y_train = Y(cvp.training);
X_test = X(cvp.test,:);
Y_test = Y(cvp.test);
                \end{lstlisting}
            \end{alertblock}
    \end{columns}
\end{frame}
\end{document}